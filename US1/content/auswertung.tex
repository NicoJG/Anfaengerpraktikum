\section{Auswertung}
\label{sec:Auswertung}

% Messwerte: Alle gemessenen physikalischen Größen sind übersichtlich darzustellen.

% Auswertung:
% Berechnung der geforderten Endergebnisse
% mit allen Zwischenrechnungen und Fehlerformeln, sodass die Rechnung nachvollziehbar ist.
% Eine kurze Erläuterung der Rechnungen (z.B. verwendete Programme)
% Graphische Darstellung der Ergebnisse

Die Ergebnisse der Messungen für das Durchschallungsverfahren sind in \autoref{tab:durchschallung} notiert und die jeweilige Messung für das Impuls-Echo-Verfahren in \autoref{tab:impulsecho}.

\begin{table}
    \centering
    \caption{Messergebnisse des Durchschallungsverfahren.}
    \begin{tabular}{S[table-format=3.1] S[table-format=1.2] S[table-format=1.1] S[table-format=1.2] S[table-format=2.1]}
        \toprule
        \tableSI{l}{\milli\meter} & \tableSI{U_\text{A}}{\volt} &\tableSI{t_\text{A}}{\micro\second} & \tableSI{U_\text{R}}{\volt} &\tableSI{t_\text{R}}{\micro\second}  \\
        \midrule
         31.1 &  0.79&    0.3&     1.35&    12.8\\
         40.5 &  0.79&    0.3&     1.36&    15.8\\
         80.6 &  0.79&    0.3&     1.35&    31.2\\
        102.1 &  0.79&    0.3&     1.34&    39.2\\
        121.7 &  0.79&    0.3&     1.34&    45.4\\
        \bottomrule
    \end{tabular}
    \label{tab:durchschallung}
\end{table}

\begin{table}
    \centering
    \caption{Messerggebnisse des Impuls-Echo-Verfahrens.}
    \begin{tabular}{S[table-format=3.1] S[table-format=1.2] S[table-format=1.1] S[table-format=1.2] S[table-format=2.1]}
        \toprule
        \tableSI{l}{\milli\meter} & \tableSI{U_\text{A}}{\volt} &\tableSI{t_\text{A}}{\micro\second} & \tableSI{U_\text{R}}{\volt} &\tableSI{t_\text{R}}{\micro\second}  \\
        \midrule
         31.1&   1.29&    6.0&     1.36&    23.7\\
         40.5&   1.30&    6.5&     1.33&    30.5\\
         80.6&   1.27&    5.6&     1.32&    59.8\\
        102.1&   1.29&    6.0&     1.18&    76.9\\
        121.7&   1.29&    6.0&     0.33&    89.1\\ 
        \bottomrule
    \end{tabular}
    \label{tab:impulsecho}
\end{table}

Zunächst wird aus diesen Messwerten die Schallgeschwindigkeit $c$ bestimmt, dies gelingt über ein Ausgleichsrechnung.
Also werden die Messdaten geplottet und eine Ausgleichsgerade durch die Punkte gelegt.
Bei dem Impuls-Echo-Verfahren wird die Laufzeit $t$ halbiert, da der Impuls die Strecke doppelt durchläuft.

\begin{figure}
    \centering
    \includegraphics[width=\textwidth]{build/plot_durchschallung.pdf}
    \caption{Plot der Messwerte zum Durchschallungsverfahren}
    \label{fig:durch}
\end{figure}

\begin{figure}
    \centering
    \includegraphics[width=\textwidth]{build/plot_impulsecho.pdf}
    \caption{Plot der Messwerte zum Impuls-Echo-Verfahren}
    \label{fig:impuls}
\end{figure}

Dabei wurde eine lineare Funktion der Art 

\begin{equation}
    f(x) = a \cdot x + b 
\end{equation}

verwendet.
Für die Durchschallung gilt

\begin{equation}
    t = \frac{1}{c} \cdot s + b 
\end{equation}

und für das Impuls-Echo-Verfahren gilt nach \eqref{eq:strecke}

\begin{equation}
    \frac{t}{2} = \frac{1}{c} \cdot s + b.
\end{equation}

Dann ergeben sich folgende Ausgleichsparameter für das Durchschallungsverfahren als

\begin{align}
    a_\text{D} &= \SI{366(6)e-6}{\second\per\meter}\\
    b_\text{D} &= \SI{13(5)e-7}{\second}.
\end{align}

Auf dem gleichen Weg werden ebenso die Parameter für das Impuls-Echo-Verfahren 

\begin{align}
    a_\text{I} &= \SI{365(6)e-6}{\second\per\meter}\\
    b_\text{I} &= \SI{5(5)e-7}{\second}.
\end{align}

bestimmt.
Hierbei ist wichtig anzumerken, dass der Parameeter $b$ in beiden Fällen einen systematischen Fehler angibt.
Dieser kann allerdings hier ignoriert werden, da dieser für die weitere Berechnung nicht relevant ist.
Über 

\begin{equation}
    c = \frac{1}{a}
\end{equation}

kann dann die Schallgeschwindigkeit bestimmt werden.
Für beide Berechnugsmethoden ergeben sich

\begin{align}
    c_\text{D} &= \SI{2730(50)}{\meter\per\second}\\
    c_\text{I} &= \SI{2737(40)}{\meter\per\second}.
\end{align}

Die Berechnung des Absorptionskoeffizienten gelingt über \eqref{eq:absorption}, wird der Zusammenhang umgeformt, ergibt sich 

\begin{equation}
    \ln{\frac{I(l)}{I_0}} = - \alpha \cdot l.
    \label{eq:geradengl}
\end{equation}

Dabei ist $l$ die Länge des Weges durch die Zylinder und $I_0$ die Startamplitude.
Die Messwerte für $I(l)$ werden aus \autoref{tab:durchschallung} und \autoref{tab:impulsecho} genommen und geben die Amplitude nach einer durchlaufenen Länge $l$ an.
$\alpha$ ist der Absorptionskoeffizient und ihn gilt es zu bestimmen. 
Bereits in \eqref{eq:geradengl} ist eine Gerade der Form 

\begin{equation}
    f(x) = a \cdot x.
    \label{eq:geradengl2}
\end{equation}

zu erkennen.
Die Messwerte werden also für beide Messverfahren geplottet und dazu wird eine Ausgleichsgerade durch die Messwerte gelegt.

\begin{figure}
    \centering
    \includegraphics[width=\textwidth]{build/plot_durchschallung2.pdf}
    \caption{Plot der Abhängigkeit von Amplitude und Weglänge zum Durchschallungsverfahren}
    \label{fig:durch2}
\end{figure}

\begin{figure}
    \centering
    \includegraphics[width=\textwidth]{build/plot_impuls2.pdf}
    \caption{Plot der Abhängigkeit von Amplitude und Weglänge zum Impuls-Echo-Verfahren}
    \label{fig:impuls2}
\end{figure}

Die Parameter $a$ von \autoref{fig:durch2} und \autoref{fig:impuls2} sind dann

\begin{align}
    a_\text{D} &= \SI{5.8(14)}{\per\meter}\\
    a_\text{I} &= \SI{-4.9(27)}{\per\meter}.
\end{align}

Wird \eqref{eq:geradengl} mit \eqref{eq:geradengl2} verglichen ergibt sich der triviale Zusammenhang 

\begin{equation}
    \alpha = - a.
\end{equation}

Darüber werden die Absorptionskoeffiziente der beiden Verfahren zu 

\begin{align}
    \alpha _\text{D} &= \SI{-5.8(14)}{\per\meter}\\
    \alpha _\text{I} &= \SI{4.9(27)}{\per\meter}
\end{align}

bestimmt.