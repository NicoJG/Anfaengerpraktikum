\section{Auswertung}
\label{sec:Auswertung}

% Messwerte: Alle gemessenen physikalischen Größen sind übersichtlich darzustellen.

% Auswertung:
% Berechnung der geforderten Endergebnisse
% mit allen Zwischenrechnungen und Fehlerformeln, sodass die Rechnung nachvollziehbar ist.
% Eine kurze Erläuterung der Rechnungen (z.B. verwendete Programme)
% Graphische Darstellung der Ergebnisse

Die Ergebnisse der Messungen für das Durchschallungsverfahren sind in \autoref{tab:durchschallung} notiert und die jeweilige Messung für das Impuls-Echo-Verfahren in \autoref{tab:impulsecho}.

\begin{table}
    \centering
    \caption{Messergebnisse des Durchschallungsverfahren.}
    \begin{tabular}{S[table-format=3.1] S[table-format=1.2] S[table-format=1.1] S[table-format=1.2] S[table-format=2.1]}
        \toprule
        \tableSI{l}{\centi\meter} & \tableSI{U_\text{A}}{\volt} &\tableSI{t_\text{A}}{\second} & \tableSI{U_\text{R}}{\volt} &\tableSI{t_\text{R}}{\second}  \\
        \midrule
         31.1 &  0.79&    0.3&     1.35&    12.8\\
         40.5 &  0.79&    0.3&     1.36&    15.8\\
         80.6 &  0.79&    0.3&     1.35&    31.2\\
        102.1 &  0.79&    0.3&     1.34&    39.2\\
        121.7 &  0.79&    0.3&     1.34&    45.4\\
        \bottomrule
    \end{tabular}
    \label{tab:durchschallung}
\end{table}

\begin{table}
    \centering
    \caption{Messerggebnisse des Impuls-Echo-Verfahrens.}
    \begin{tabular}{S[table-format=3.1] S[table-format=1.2] S[table-format=1.1] S[table-format=1.2] S[table-format=2.1]}
        \toprule
        \tableSI{l}{\centi\meter} & \tableSI{U_\text{A}}{\volt} &\tableSI{t_\text{A}}{\second} & \tableSI{U_\text{R}}{\volt} &\tableSI{t_\text{R}}{\second}  \\
        \midrule
         31.1&   1.29&    6.0&     1.36&    23.7\\
         40.5&   1.30&    6.5&     1.33&    30.5\\
         80.6&   1.27&    5.6&     1.32&    59.8\\
        102.1&   1.29&    6.0&     1.18&    76.9\\
        121.7&   1.29&    6.0&     0.33&    89.1\\ 
        \bottomrule
    \end{tabular}
    \label{tab:impulsecho}
\end{table}

Zunächst wird aus diesen Messwerten die Schallgeschwindigkeit $c$ bestimmt, dies gelingt über ein Ausgleichsrechnung.
Also werden die Messdaten geplottet und eine Ausgleichsgerade durch die Punkte gelegt.
Bei dem Impuls-Echo-Verfahren wird die Laufzeit $t$ halbiert, da der Impuls die Strecke doppelt durchläuft.

\begin{figure}
    \centering
    \includegraphics[width=\textwidth]{build/plot_durchschallung.pdf}
    \caption{Plot der Messwerte zum Durchschallungsverfahren}
    \label{fig:durch}
\end{figure}

\begin{figure}
    \centering
    \includegraphics[width=\textwidth]{build/plot_impulsecho.pdf}
    \caption{Plot der Messwerte zu Impuls-Echo-Verfahren}
    \label{fig:impuls}
\end{figure}

Dabei wurde eine lineare Funktion der Art 

\begin{equation}
    f(x) = a \cdot x + b 
\end{equation}

verwendet.
Für die Durchschallung gilt

\begin{equation}
    t = \frac{1}{c} \cdot s + b 
\end{equation}

und für das Impuls-Echo-Verfahren gilt nach \eqref{eq:strecke}

\begin{equation}
    \frac{t}{2} = \frac{1}{c} \cdot s + b.
\end{equation}

Dann ergeben sich folgende Ausgleichsparameter für das Durchschallungsverfahren als

\begin{align}
    a_\text{D} &= 36.63 \pm \SI{0.64}{\second\per\meter}\\
    b_\text{D} &= 1.33 \pm \SI{0.53}{\second}.
\end{align}

Auf dem gleichen Weg werden ebenso die Parameter für das Impuls-Echo-Verfahren 

\begin{align}
    a_\text{I} &= 36.54 \pm \SI{0.57}{\second\per\meter}\\
    b_\text{I} &= 0.52 \pm \SI{0.47}{\second}.
\end{align}

bestimmt.
Hierbei ist wichtig anzumerken, dass der Parameeter $b$ in beiden Fällen einen systematischen Fehler angibt.
Dieser kann allerdings hier ignoriert werden, da dieser für die weitere Berechnung nicht relevant ist.
Über 

\begin{equation}
    c = \frac{1}{a}
\end{equation}

kann dann die Schallgeschwindigkeit bestimmt werden.