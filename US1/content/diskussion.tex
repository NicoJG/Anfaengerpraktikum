\section{Diskussion}
\label{sec:Diskussion}

% Kurze Zusammenfassung der Ergebnisse
% -Vergleich mit Literaturwerten
% -Vergleich mit verschiedenen Messverfahren
% -bei Abweichungen mögliche Ursachen finden

Als erstes wurde die Schallgeschwindigkeit in Acrylglas bestimmt.
In diesem Experiment wurde sie auf zwei verschiedene Arten bestimmt.
In \autoref{tab:e_lit} werden beide Werte jeweils mit ihren Theoriewerten verglichen und ihre Abweichung bestimmt.

\begin{table}
  \centering
  \caption{Vergleich der experimentell berechneten und den theoretischen Schallgeschwindigkeiten.\cite{schallgeschw}}
  \label{tab:e_lit}
  \begin{tabular}{c S[table-format=4.0,table-figures-uncertainty = 1] S[table-format=4.0] S[table-format=1.2]}
    \toprule 
   \text{Verfahren} & \tableSI{c_\text{gem}}{\frac{\meter}{\second}} & \tableSI{c_\text{lit}}{\frac{\meter}{\second}} & \text{Abweichungen} \\ 
    \midrule 
    Durchschallung & 2730+-50 & 2760 & 1.09 \% \\
    Impuls-Echo & 2737+-40 & 2760 & 0.83 \% \\
    \bottomrule
  \end{tabular}
\end{table}

Wird die Unsicherheit der Werte mitbetrachet, liegen die Theoriewerte sogar innerhalb des Intervalls.
Das Messergebnis kann also als sehr genau beschrieben werden.

Zuletzt wurde der Absorptionskoeffizent $\alpha$ bestimmt, auch hier über beide Messmethoden

\begin{align}
    \alpha _\text{D} &= \SI{-5.8(14)}{\per\meter}\\
    \alpha _\text{I} &= \SI{4.9(27)}{\per\meter}.
\end{align}

Hier muss gesagt werden, dass die Werte sehr ungenau, bis physikalisch unmöglich sind.
Die Impulsamlitude beim Durchschallungsverfahren ist nach dem Durchlaufen des Zylinders größer geworden.
Dieses Ergebnis ist nicht erklärbar und es konnte keine Lösung gefunden werden.
Sämtliche Amplituden eignen sich also nicht im geringsten für die Messung.
$\alpha _\text{D}$ besitzt daher das falsche Vorzeichen.
Das gleiche Problem haben die Messwerte des Impuls-Echo Verfahrens.
Nur hier ist es nicht mal konsistent. Hier werden manche Amplituden verstärkt und andere abgeschwächt. 
Es erscheint daher nicht wirklich sinnvoll überhaupt einen Vergleich mit der Theorie vorzunehmen, da die berechneten Werte absolut keine Aussage treffen können.
Aufgrund der sehr genauen Werte für $c$ wird der Fehler bei der Amplitudenmessung liegen.
Allerdings kann nachträglich nicht gesagt werden, welche Einstellung nicht korrekt war.
