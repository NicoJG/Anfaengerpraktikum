\section{Durchführung}
\label{sec:Durchführung}

% Was wurde gemessen bzw. welche Größen wurden variiert?

Es werden fünf verschieden hohe Acrylzylinder mithilfe des Durchschallungs- und des Impuls-Echo-Verfahrens vermessen.

Dazu werden als erstes die Zylinder mit einem Messschieber vermessen. 

Die $\SI{2}{\mega\hertz}$ Sonde wird über ein Ultraschallechoskop mit einem Computer verbunden.
Am Computer wird die Scanmethode zu A-Scan gewählt.
Somit kann jeweils die Amplituden und Zeiten der ausgesendeten und eingehenden Impulse auf dem Bildschirm abgelesen werden.

Für das Impuls-Echo-Verfahren wird der Zylinder auf ein Papiertuch gestellt, damit diesem keine Kratzer zugefügt werden.
Die Ultraschallsonde wird mit bidestilliertem Wasser an den Zylinder gekoppelt.
Dann wird am Computer das Impuls-Echo-Verfahren ausgewählt und die entsprechenden Amplituden und Zeiten der Impulse werden notiert.

Für das Durchschallungsverfahren wird der Zylinder horizontal auf eine Halterung gelegt und an beiden Seiten wird eine Sonde mit Kopplungsgel gekoppelt.
Am Computer wird das Durchschallungsverfahren ausgewählt und es werden erneut die Amplituden und Zeiten der Impulse notiert.

Dies wird für fünf verschiedene Zylinder wiederholt.
