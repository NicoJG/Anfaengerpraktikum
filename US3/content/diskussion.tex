\section{Diskussion}
\label{sec:Diskussion}

% Kurze Zusammenfassung der Ergebnisse
% -Vergleich mit Literaturwerten
% -Vergleich mit verschiedenen Messverfahren
% -bei Abweichungen mögliche Ursachen finden

In \autoref{tab:stroemung} sind die berechneten Geschwindigkeiten eingetragen. Zudem befindet sich dort die umgerechnete eingestellte Geschwindigkeit $\dot{v}$.
Diese dient als Referenzwert für alle berechneten Werte.
Nachfolgend werden für alle Werte die Abweichungen bestimmt.
Die Ergebnisse stehen in \autoref{tab:stroemungsdiskussion}.

\begin{table}
    \centering
    \caption{Vergleich der berechneten Strömungsgeschwindigkeiten mit der Eingestellten}
    \begin{tabular}{S[table-format=1.2] S[table-format=1.2] S[table-format=2.2] S[table-format=1.2] S[table-format=2.2] S[table-format=1.2] S[table-format=2.2]}
        \toprule
        \tableSI{\dot{v}}{\frac{\meter}{\second}} & \tableSI{v_{15}}{\frac{\meter}{\second}} & \text{Abw.} & \tableSI{v_{30}}{\frac{\meter}{\second}} & \text{Abw.} & \tableSI{v_{60}}{\frac{\meter}{\second}} & \text{Abw.}\\
        \midrule
        0.64 & 0.52 & 18.75\% &  0.51 & 20.31\% &  0.52 & 18.75\% \\
        0.74 & 0.68 &  8.10\% &  0.62 & 16.21\% &  0.64 & 13.51\% \\
        0.85 & 0.84 &  1.18\% &  0.81 &  4.71\% &  0.81 &  4.71\% \\
        0.95 & 0.99 &  4.21\% &  0.97 &  2.11\% &  0.97 &  2.11\% \\
        1.06 & 1.23 & 16.04\% &  1.15 &  8.49\% &  1.18 & 11.32\% \\
        \bottomrule
    \end{tabular}
    \label{tab:stroemungsdiskussion}
\end{table}

Die Abweichungen varriieren recht stark, aber befinden sich bis auf eine Ausnahme unter den $\SI{20}{\percent}$. 
Es fällt auf, dass die Messung mit der gerinsten Geschwindigkeit am ungenausten ist, für jeden Dopplerwinkel. Die Werte werden dann zunehmend genauer, bis sie dann am Ende wieder teilweise sehr stark schwanken.
Hier ist wichtig zu beachten, dass die aus dem Messprogramm abgelesenen Werte teils sehr stark geschwankt haben. Daher musste ein Kompromiss gefunden werden, bei dem etwa der mittlere Wert genommen wurde, jedoch ist das höchst ungenau, was sich teilweise hier wiederspiegelt.
Die drei Plots \autoref{fig:doppler1}, \autoref{fig:doppler2} und \autoref{fig:doppler3} stellen alle einen linearen Zusammenhang dar. 
Es gibt in keinem der Plots zu große Abweichungen, die lineare Gestalt ist in jedem Fall erkennebar, also kann hier von einem erfolgreichen Ergebnis die Rede sein.

Für den zweiten Teil der Auswertung gibt es größtenteils nur Plots, die diskutiert werden können.
Dabei stellen \autoref{fig:profil1} und \autoref{fig:profil3} in guter Näherung Parabeln dar, allerdings gibt es vor allen im ersten Plot einen starken Ausreißer.
Dieser lässt sich rückblickend nicht erklären, eventuell ist während dieser Messung die Sonde leicht verrutscht. Von einem Fehler der Messung an sich kann nicht ausgegangen werden, da danach alle Werte wieder eine Parabel ergeben.
In \autoref{fig:profil1} fallen die letzten Messwerte aus dem Parabelmuster heraus. Es scheint, als würde dort eine neue Parabel beginnen.
Warum dieser Effekt hier auftritt ist nicht zu erklären, da ein vergleichbarer Knick in \autoref{fig:profil3} nicht auftaucht. 
Dort sind die Werte weniger präszise, als zuvor, aber sie repräsentieren den zu erwartenden Verlauf. 
Zusammenfassend ergibt es Sinn, dass beide Größen in der Mitte des Rohres ihr Maximum haben und an den Rändern jeweils eine Schwächung auftaucht.