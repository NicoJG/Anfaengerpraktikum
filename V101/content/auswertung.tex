\section{Auswertung}
\label{sec:Auswertung}

% Messwerte: Alle gemessenen physikalischen Größen sind übersichtlich darzustellen.

% Auswertung:
% Berechnung der geforderten Endergebnisse
% mit allen Zwischenrechnungen und Fehlerformeln, sodass die Rechnung nachvollziehbar ist.
% Eine kurze Erläuterung der Rechnungen (z.B. verwendete Programme)
% Graphische Darstellung der Ergebnisse

Zunächst muss die Winkelrichtgröße $D$ und das Eigenträgheitsmoment $I_D$ der Drillachse bestimmt werden.

Mit diesen Werten können dann über \autoref{eq:periodendauer} die Trägheitsmomente der verwendeten Körper bestimmt werden. Folgende Körper wurden bei diesem Versuch untersucht:
\begin{enumerate}
    \item Zylinder mit Drehachse entlang der Symmetrieachse
    \item Zylinder mit Drehachse entlang der Querachse
    \item Holzpuppe mit angelegten Armen und Beinen
    \item Holzpuppe mit ausgestreckten Armen und Beinen
\end{enumerate}

Zur Fehlerfortpflanzung wird 
\begin{equation}
    \Delta y = \sum_{i=1}^n \left| \frac{\delta f(x_1,...,x_n)}{\delta x_i} \right| \Delta x_i
    \label{eq:fehlerrechnung}
\end{equation}
bzw. die Python Bibliothek Uncertainties verwendet.\cite{uncertainties}

%%%%%%%%%%%%%%%%%%%%%%%%%%%%%%%%%%%%%%%%%%%%%%%%%%%%%%
\subsection{Bestimmung der Winkelrichtgröße der Drillachse}
\label{sec:winkelrichtgroesse}

Zur Bestimmung der Winkelrichtgröße $D$ wurde die rücktreibende Kraft $F$ im Abstand $R=\SI{28.7}{\centi\meter}$ zur Drehachse in Abhängigkeit der Auslenkung $\varphi$ gemessen. Hiermit kann nun das rücktreibende Drehmoment $M=R \cdot F$ berechnet werden. Diese Werte sind in \autoref{tab:winkelrichtgroesse} aufgelistet und in \autoref{fig:plot_winkelrichtgroesse} dargestellt.

\begin{table}
    \centering
    \begin{tabular}{S[table-format=3.0] S[table-format=1.2] S[table-format=1.2] S[table-format=1.3]}
        \toprule
        \tableSI{\varphi}{\degree} & \tableSI{\varphi}{\radian} & \tableSI{F}{\newton} & \tableSI{M}{\newton\meter} \\
        \midrule
        30 & 0.52 & 0.05 & 0.014 \\
        60 & 1.05 & 0.09 & 0.026 \\
        90 & 1.57 & 0.13 & 0.037 \\
        120 & 2.09 & 0.17 & 0.049 \\
        150 & 2.62 & 0.22 & 0.063 \\
        180 & 3.14 & 0.26 & 0.075 \\
        210 & 3.67 & 0.31 & 0.089 \\
        240 & 4.19 & 0.34 & 0.098 \\
        270 & 4.71 & 0.39 & 0.112 \\
        300 & 5.24 & 0.43 & 0.123 \\
        \bottomrule
    \end{tabular}
    \caption{Messwerte zur Bestimmung der Winkelrichtgröße: Auslenkung $\varphi$, Kraft $F$ und Drehmoment $M=R \cdot F$}
    \label{tab:winkelrichtgroesse}
\end{table}

\begin{figure}
    \centering
    \includegraphics[width=\textwidth]{build/plot_winkelrichtgroesse.pdf}
    \caption{Plot des rücktreibenden Drehmoments $M$ in Abhängigkeit der Auslenkung $\varphi$ aus \autoref{tab:winkelrichtgroesse} mit dazugehöriger Ausgleichsgerade}
    \label{fig:plot_winkelrichtgroesse}
\end{figure}

Um nun die Winkelrichtgröße aus \autoref{eq:winkelrichtgroesse} bestimmen zu können, wurde eine Ausgleichsrechnung mit
\begin{equation}
    M = a \cdot \varphi
\end{equation}
mithilfe der Python Bibliothek SciPy ausgeführt.\cite{scipy} Dadurch ergibt sich der Parameter $a=\SI{0.0237+-0.0001}{\newton\meter}$. Nach \autoref{eq:winkelrichtgroesse} ergibt sich also die Winkelrichtgröße
\begin{equation}
    D = a = \SI{0.0237+-0.0001}{\newton\meter}.
\end{equation}

%%%%%%%%%%%%%%%%%%%%%%%%%%%%%%%%%%%%%%%%%%%%%%%%%%%%%%
\subsection{Bestimmung des Eigenträgheitsmoments der Drillachse}
\label{sec:eigentraegheit}

Zur Bestimmung des Eigenträgheitsmoments $I_\text{D}$ wurden zwei zylinderförmige Massen $m=\SI{222.9}{\gram}$ in verschiedenen Abständen $a$ zur Drehachse befestigt und die Periodendauer $T$ wurde bei einer Initialauslenkung von $\varphi=\SI{10}{\degree}$ gemessen.

Die Massen werden hier als Punktmassen angenommen. Gemessen wurden hierbei allerdings die Abstände von der Drehachse bis zum Anfang des Zylinders. Also lässt sich der gesuchte Abstand mit $a=a_\text{gemessen} + L/2$ berechnen, wobei $L=\SI{3.0}{\centi\meter}$ die Länge des Zylinders ist.

Diese Werte sind in \autoref{tab:eigentraegheit} aufgelistet und in \autoref{fig:plot_eigentraegheit} dargestellt.

\begin{table}
    \centering
    \begin{tabular}{S[table-format=2.1] S[table-format=2.1] S[table-format=1.2]}
        \toprule
        \tableSI{a_\text{gemessen}}{\centi\meter} & \tableSI{a}{\centi\meter} & \tableSI{T}{\second} \\
        \midrule
        26.9 & 28.4 & 7.50 \\
        25.0 & 26.5 & 7.15 \\
        23.0 & 24.5 & 6.58 \\
        21.0 & 22.5 & 6.35 \\
        19.0 & 20.5 & 5.80 \\
        17.0 & 18.5 & 5.32 \\
        15.0 & 16.5 & 4.60 \\
        13.0 & 14.5 & 4.15 \\
        9.0 & 10.5 & 3.32 \\
        4.0 & 5.5 & 2.60 \\
        \bottomrule
    \end{tabular}
    \begin{tabular}{S[table-format=3.1] S[table-format=2.2]}
        \toprule
        \tableSI{a^2}{\centi\meter\squared} & \tableSI{T^2}{\second\squared} \\
        \midrule
        806.6 & 56.25 \\
        702.2 & 51.12 \\
        600.2 & 43.30 \\
        506.2 & 40.32 \\
        420.2 & 33.64 \\
        342.2 & 28.30 \\
        272.2 & 21.16 \\
        210.3 & 17.22 \\
        110.2 & 11.02 \\
        30.2 & 6.76 \\
        \bottomrule
    \end{tabular}
    \caption{Messwerte zur Bestimmung des Eigenträgheitsmoments: Abstände $a_\text{gemessen}$ und $a$, Periodendauer $T$ sowie die geplotteten Werte $a^2$ und $T^2$}
    \label{tab:eigentraegheit}
\end{table}

\begin{figure}
    \centering
    \includegraphics[width=\textwidth]{build/plot_eigentraegheit.pdf}
    \caption{Plot der Quadratperiodendauer $T^2$ in Abhängigkeit des Quadratabstands $a^2$ aus \autoref{tab:eigentraegheit} mit dazugehöriger Ausgleichsgerade}
    \label{fig:plot_eigentraegheit}
\end{figure}

Um nun aus den gemessenen Werten das Eigenträgheitsmoment der Drillachse auszurechnen kann \autoref{eq:periodendauer} verwendet werden. Hierbei ist $I=I_\text{D}+2I_m$ mit dem Trägheitsmoment $I_m=ma^2$ der zusätzlich angehangenen Massen. Somit ergibt sich die Gleichung
\begin{equation}
    T=2\pi \sqrt{\frac{I_\text{D}+2ma^2}{D}}.
    \label{eq:eigentraegheit_theorie}
\end{equation}
Also wird eine Ausgleichsrechnung mit
\begin{equation}
    T^2 = p_1 \cdot a^2 + p_2
    \label{eq:eigentraegheit_fit}
\end{equation}
mithilfe von SciPy ausgeführt.\cite{scipy} Dabei ergeben sich die Parameter $p_1=\SI{660+-18}{\second\squared\per\meter\squared}$ und $p_2=\SI{4.48+-0.86}{\second\squared}$. 

Nach vergleichen der \autoref{eq:eigentraegheit_fit} mit \autoref{eq:eigentraegheit_theorie} lässt sich das Eigenträgheitsmoment mit
\begin{equation}
    I_\text{D} = \frac{D}{(2\pi)^2}p_2
\end{equation}
berechnen. Somit ergibt sich
\begin{equation}
    I_\text{D} = \SI{0.0026+-0.0005}{\kilo\gram\meter\squared}.
\end{equation}

%%%%%%%%%%%%%%%%%%%%%%%%%%%%%%%%%%%%%%%%%%%%%%%%%%%%%%
\subsection{Bestimmung des Trägheitsmoments zweier Zylinder}
\label{sec:zylinder}

Es wurde von zwei Zylindern, die jeweils von der zuvor vermessenen Drillachse zum Schwingen gebracht wurden, die Periodendauer $T$ der Schwingung mehrmals gemessen. Diese Periodendauern werden über
\begin{equation}
    \bar{x} = \frac{1}{n} \sum_{i=1}^n x_i
\end{equation}
gemittelt.
Und über 
\begin{equation}
    \Delta\bar{x} = \sqrt{\frac{1}{n(n-1)}\sum_{i=1}^n (x_i - \bar{x})^2}
\end{equation}
wird der entsprechende Fehler des Mittelwerts berechnet.
Durch diese gemittelte Periodendauer wird im Folgenden über \autoref{eq:periodendauer} das Trägheitsmoment bestimmt.

%%%%%%%%%%%%%%%%%%%%%%%%%%%%%%%%%%%%%%%%%%%%%%%%%%%%%%
\subsubsection{Zylinder um seine Symmetrieachse drehend}
\label{sec:brauner_zylinder}

\begin{table}
    \centering
    \begin{tabular}{S S S S S S S S S S}
        \toprule
        \multicolumn{10}{c}{\tableSI{T_\text{gemessen}}{\second}} \\
        \midrule
        0.73 & 0.92 & 0.87 & 0.78 & 0.93 & 0.96 & 0.89 & 0.96 & 0.89 & 0.87 \\
        \bottomrule
    \end{tabular}
    \caption{Gemessene Periodendauern des Zylinders, welcher um seine Symmetrieachse dreht}
    \label{tab:brauner_zylinder}
\end{table}

Die Mittelwertsbildung der Werte aus \autoref{tab:brauner_zylinder} ergibt eine Periodendauer 
\begin{equation}
    T = \SI{0.88+-0.02}{\second}.
\end{equation}
Daraus ergibt sich mit \autoref{eq:periodendauer} das Trägheitsmoment
\begin{equation}
    I_\text{gemessen} = \SI{-0.0022+-0.0005}{\kilo\gram\meter\squared}.
\end{equation}
Diese errechnete Größe ist jedoch nicht sinnvoll, da ein negatives Trägheitsmoment nicht physikalisch zu erklären ist.

Auf die Ursachen der Bestimmung dieses offensichtlich fehlerhaften Trägheitsmoments wird näher in der Diskussion eingegangen.

Um die berechnete Größe vergleichen zu können wird außerdem mithilfe der gemessenen Masse $m=\SI{1119.3}{\gram}$ und dem Durchmesser $d=\SI{7.495}{\centi\meter}$ des Zylinders das theoretische Trägheitsmoment über
\begin{equation}
    I_\text{Theorie} = \frac{1}{2}m\left(\frac{d}{2}\right)^2
\end{equation}
bestimmt. (siehe \autoref{fig:zylinder}) Damit ergibt sich der Theoriewert
\begin{equation}
    I_\text{Theorie} = \SI{0.00079}{\kilo\gram\meter\squared}.
\end{equation}

%%%%%%%%%%%%%%%%%%%%%%%%%%%%%%%%%%%%%%%%%%%%%%%%%%%%%%
\subsubsection{Zylinder um seine Querachse drehend}
\label{sec:weisser_zylinder}

\begin{table}
    \centering
    \begin{tabular}{S S S S S S S S S S}
        \toprule
        \multicolumn{10}{c}{\tableSI{T_\text{gemessen}}{\second}} \\
        \midrule
        2.36 & 2.03 & 2.04 & 2.06 & 2.16 & 2.23 & 2.13 & 2.20 & 2.18 & 2.16 \\
        \bottomrule
    \end{tabular}
    \caption{gemessene Periodendauern des Zylinders, welcher um seine Querachse dreht}
    \label{tab:weisser_zylinder}
\end{table}

Die Mittelwertsbildung der Werte aus \autoref{tab:weisser_zylinder} ergibt eine Periodendauer 
\begin{equation}
    T = \SI{2.15+-0.03}{\second}.
\end{equation}
Daraus ergibt sich mit \autoref{eq:periodendauer} das Trägheitsmoment
\begin{equation}
    I_\text{gemessen} = \SI{0.0001+-0.0005}{\kilo\gram\meter\squared}.
\end{equation}
Auch hier lässt sich ein nicht sinnvolles Ergebnis beobachten, da der berechnete Fehler größer ist als das berechnete Trägheitsmoment.

Um die berechnete Größe vergleichen zu können wird außerdem mithilfe der gemessenen Masse $m=\SI{1546.6}{\gram}$, dem Durchmesser $d=\SI{8.00}{\centi\meter}$ und der Länge $l=\SI{13.93}{\centi\meter}$ des Zylinders das theoretische Trägheitsmoment über
\begin{equation}
    I_\text{Theorie} = \frac{1}{4}m\left(\frac{d}{2}\right)^2 + \frac{1}{12}ml^2
\end{equation}
bestimmt.(siehe \autoref{fig:zylinder}) Damit ergibt sich der Theoriewert
\begin{equation}
    I_\text{Theorie} = \SI{0.0031}{\kilo\gram\meter\squared}.
\end{equation}

%%%%%%%%%%%%%%%%%%%%%%%%%%%%%%%%%%%%%%%%%%%%%%%%%%%%%%
\subsection{Bestimmung des Trägheitsmoments einer Puppe in zwei Stellungen}
\label{sec:puppe}

Es wurde eine Holzpuppe in zwei verschiedenen Stellungen, wie in \autoref{sec:puppe} beschrieben, zum Schwingen gebracht. 
Durch die Periodendauer dieser Schwingung kann das jeweilige Trägheitsmoment bestimmt werden.

Um Theoriewerte für das Trägheitsmoment der Puppe berechnen zu können wurde diese zunächst vermessen. 
Die Messergebnisse sind in \autoref{tab:puppe_maße} aufgelistet. 
Hierbei wurden alle Körperteile als Zylinder angenähert.

Außerdem wurden die Massen der einzelnen Körperteile berechnet. Hier wurde davon ausgegangen, dass die Puppe aus Tannenholz ist, welches eine Dichte von $\rho = \SI{430}{\kilo\gram\per\meter\cubed}$ aufweist.\cite{holzarten}
Mithilfe dieser Dichte, der \autoref{eq:dichte} und der Gleichung vom Volumen
\begin{equation}
    V_\text{Zylinder} = \pi \left(\frac{d}{2}\right)^2 l 
\end{equation}
wurden die jeweiligen Massen berechnet.

Das jeweilige Trägheitsmoment bei Rotation um den Schwerpunkt wurde wie in \autoref{fig:zylinder} berechnet. 

\begin{table}
    \centering
    \begin{tabular}{c S[table-format=1.1] S[table-format=2.1] S[table-format=2.1] S[table-format=1.2e-1] S[table-format=1.2e-1]}
        \toprule
        Körperteil & \tableSI{d}{\centi\meter} & \tableSI{l}{\centi\meter} & \tableSI{m}{\gram} & \tableSI{I_1}{\kilo\gram\meter\squared} & \tableSI{I_2}{\kilo\gram\meter\squared} \\
        \midrule
        Kopf & 3.0 & 6.0 & 18.2 & \multicolumn{2}{S[table-format=1.2e-1]}{2.05e-6} \\
        Torso & 4.0 & 10.0 & 54.0 & \multicolumn{2}{S[table-format=1.2e-1]}{10.8e-6} \\
        Arm & 1.3 & 13.7 & 7.8 & 0.165e-6 & 12.3e-6 \\
        Bein & 1.5 & 14.5 & 11.0 & 0.310e-6 & 19.5e-6 \\
        \bottomrule
    \end{tabular}
    \caption{Messergebnisse des Vermessens der einzelnen Körperteile der Puppe: Durchmesser $d$, Länge $l$, Masse $m$, Trägheitsmoment der jeweiligen Stellung $I_1$,$I_2$}
    \label{tab:puppe_maße}
\end{table}

Im Folgenden wird aus den gemessenen Periodendauern das jeweilige Trägheitsmoment bestimmt.

\FloatBarrier
%%%%%%%%%%%%%%%%%%%%%%%%%%%%%%%%%%%%%%%%%%%%%%%%%%%%%%
\subsubsection{Puppe mit angelegten Armen und Beinen}
\label{sec:puppe_1}

\begin{table}
    \centering
    \begin{tabular}{S S S S S S S S S S}
        \toprule
        \multicolumn{10}{c}{\tableSI{T_\text{gemessen}}{\second}} \\
        \midrule
        1.93 & 1.95 & 1.93 & 1.75 & 1.72 & 1.76 & 1.95 & 1.83 & 1.95 & 1.72 \\
        \bottomrule
    \end{tabular}
    \caption{gemessene 5-fache Periodendauer der schwingenden Puppe mit angelegten Armen und Beinen}
    \label{tab:puppe_1}
\end{table}

Aus den Werten in \autoref{tab:puppe_1} wird die gemittelte 5-fache Periodendauer
\begin{equation}
    \bar{T}_\text{gemessen} = \SI{1.85+-0.03}{\second}
\end{equation}
bestimmt. Hieraus lässt sich die Periodendauer
\begin{equation}
    T = \SI{0.370+-0.007}{\second}
\end{equation}
berechnen.

Mithilfe dieser Periodendauer und \autoref{eq:periodendauer} berechnet sich das Trägheitsmoment
\begin{equation}
    I_{1,\text{gemessen}} = \SI{-0.0026+-0.0005}{\kilo\gram\meter\squared}.
\end{equation}
Auch hier ist diese Größe offensichtlich nicht sinnvoll. Näheres dazu in der Diskussion.

Um die experimentell bestimmte Größe mit einem Theoriewert vergleichen zu können, wird mit den Werten aus \autoref{tab:puppe_maße} das theoretische Trägheitsmoment der Puppe in dieser Stellung bestimmt.

Da Trägheitsmomente als Addition der Trägheitsmomente der einzelnen Körper beschrieben werden können, wird nun nach \autoref{eq:steiner} das Trägheitsmoment über
\begin{equation}
    \begin{split}
        I_{1,\text{Theorie}} = {} & I_\text{Kopf} + I_\text{Torso} \\
        & + 2 \left( I_\text{1,Arm} + m_\text{Arm} \left(\frac{d_\text{Torso}}{2} + \frac{d_\text{Arm}}{2} \right)^2 \right) \\
        & + 2 \left( I_\text{1,Bein} + m_\text{Bein} \left(\frac{d_\text{Arm}}{2}\right)^2 \right) 
    \end{split}
\end{equation}
berechnet. Damit ergibt sich der Theoriewert
\begin{equation}
    I_{1,\text{Theorie}} = \SI{2.6e-5}{\kilo\gram\meter\squared}.
\end{equation}

\FloatBarrier
%%%%%%%%%%%%%%%%%%%%%%%%%%%%%%%%%%%%%%%%%%%%%%%%%%%%%%
\subsubsection{Puppe mit ausgestreckten Armen und Beinen}
\label{sec:puppe_2}

\begin{table}
    \centering
    \begin{tabular}{S S S S S S S S S S}
        \toprule
        \multicolumn{10}{c}{\tableSI{T_\text{gemessen}}{\second}} \\
        \midrule
        1.72 & 1.73 & 1.81 & 1.72 & 1.78 & 1.76 & 1.81 & 1.87 & 1.72 & 1.84 \\
        \bottomrule
    \end{tabular}
    \caption{gemessene 2-fache Periodendauer der schwingenden Puppe mit ausgestreckten Armen und Beinen}
    \label{tab:puppe_2}
\end{table}

Aus den Werten in \autoref{tab:puppe_1} wird die gemittelte 2-fache Periodendauer
\begin{equation}
    \bar{T}_\text{gemessen} = \SI{1.78+-0.02}{\second}
\end{equation}
bestimmt. Hieraus lässt sich die Periodendauer
\begin{equation}
    T = \SI{0.888+-0.009}{\second}
\end{equation}
berechnen.

Mithilfe dieser Periodendauer und \autoref{eq:periodendauer} berechnet sich das Trägheitsmoment
\begin{equation}
    I_{2,\text{gemessen}} = \SI{-0.0022+-0.0005}{\kilo\gram\meter\squared}.
\end{equation}
Auch hier ist diese Größe offensichtlich nicht sinnvoll. Näheres dazu in der Diskussion.

Wie oben begründet lässt sich das theoretische Trägheitsmoment der Puppe mit ausgestreckten Armen und Beinen mit
\begin{equation}
    \begin{split}
        I_{2,\text{Theorie}} = {} & I_\text{Kopf} + I_\text{Torso} \\
        & + 2 \left( I_\text{2,Arm} + m_\text{Arm} \left(\frac{d_\text{Torso}}{2} + \frac{l_\text{Arm}}{2} \right)^2 \right) \\
        & + 2 \left( I_\text{Bein} + m_\text{Bein} \left( \left(\frac{l_\text{Bein}}{2}\right)^2 + \left(\frac{d_\text{Bein}}{2}\right)^2 \right) \right)
    \end{split}
\end{equation}
berechnen. Wobei für den Abstand von Drehachse zum Schwerpunkt des Beins der Satz von Pythagoras verwendet wurde.

Damit ergibt sich das Trägheitsmoment
\begin{equation}
    I_{2,\text{Theorie}} = \SI{3.1e-4}{\kilo\gram\meter\squared}.
\end{equation}