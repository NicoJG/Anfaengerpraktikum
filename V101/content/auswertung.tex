\section{Auswertung}
\label{sec:Auswertung}

% Messwerte: Alle gemessenen physikalischen Größen sind übersichtlich darzustellen.

% Auswertung:
% Berechnung der geforderten Endergebnisse
% mit allen Zwischenrechnungen und Fehlerformeln, sodass die Rechnung nachvollziehbar ist.
% Eine kurze Erläuterung der Rechnungen (z.B. verwendete Programme)
% Graphische Darstellung der Ergebnisse

Zunächst muss die Winkelrichtgröße $D$ und das Eigenträgheitsmoment $I_D$ der Drillachse bestimmt werden.

Mit diesen Werten können dann über \autoref{eq:periodendauer} die Trägheitsmomente der verwendeten Körper bestimmt werden. Folgende Körper wurden bei diesem Versuch untersucht:
\begin{enumerate}
    \item Zylinder mit Drehachse entlang der Symmetrieachse
    \item Zylinder mit Drehachse entlang der Querachse
    \item Holzpuppe mit angelegten Armen und Beinen
    \item Holzpuppe mit ausgestreckten Armen und Beinen
\end{enumerate}

\subsection{Bestimmung der Winkelrichtgröße der Drillachse}
\label{sec:winkelrichtgroesse}

Zur Bestimmung der Winkelrichtgröße $D$ wurde die rücktreibende Kraft $F$ im Abstand $R=\SI{28.7}{\centi\meter}$ zur Drehachse in Abhängigkeit der Auslenkung $\varphi$ gemessen. Hiermit kann nun das rücktreibende Drehmoment $M=R \cdot F$ berechnet werden. Diese Werte sind in \autoref{tab:winkelrichtgroesse} aufgelistet und in \autoref{fig:plot_winkelrichtgroesse} dargestellt.

\begin{table}
    \centering
    \begin{tabular}{S[table-format=3.0] S[table-format=1.2] S[table-format=1.2] S[table-format=1.3]}
        \toprule
        \tableSI{\varphi}{\degree} & \tableSI{\varphi}{\radian} & \tableSI{F}{\newton} & \tableSI{M}{\newton\meter} \\
        \midrule
        30 & 0.52 & 0.05 & 0.014 \\
        60 & 1.05 & 0.09 & 0.026 \\
        90 & 1.57 & 0.13 & 0.037 \\
        120 & 2.09 & 0.17 & 0.049 \\
        150 & 2.62 & 0.22 & 0.063 \\
        180 & 3.14 & 0.26 & 0.075 \\
        210 & 3.67 & 0.31 & 0.089 \\
        240 & 4.19 & 0.34 & 0.098 \\
        270 & 4.71 & 0.39 & 0.112 \\
        300 & 5.24 & 0.43 & 0.123 \\
        \bottomrule
    \end{tabular}
    \caption{Messwerte zur Bestimmung der Winkelrichtgröße: Auslenkung $\varphi$, Kraft $F$ und Drehmoment $M=R \cdot F$}
    \label{tab:winkelrichtgroesse}
\end{table}

\begin{figure}
    \centering
    \includegraphics[width=\textwidth]{build/plot_winkelrichtgroesse.pdf}
    \caption{Plot des rücktreibenden Drehmoments $M$ in Abhängigkeit der Auslenkung $\varphi$ aus \autoref{tab:winkelrichtgroesse} mit dazugehöriger Ausgleichsgerade}
    \label{fig:plot_winkelrichtgroesse}
\end{figure}

Um nun die Winkelrichtgröße aus \autoref{eq:winkelrichtgroesse} bestimmen zu können, wurde eine Ausgleichsrechnung mit
\begin{equation}
    M = a \cdot \varphi
\end{equation}
mithilfe der Python Bibliothek SciPy ausgeführt.\cite{scipy} Dadurch ergibt sich der Parameter $a=\SI{0.0237+-0.0001}{\newton\meter}$. Nach \autoref{eq:winkelrichtgroesse} ergibt sich also die Winkelrichtgröße
\begin{equation}
    D = a = \SI{0.0237+-0.0001}{\newton\meter}.
\end{equation}

\subsection{Bestimmung des Eigenträgheitsmoments der Drillachse}
\label{sec:eigentraegheit}

Zur Bestimmung des Eigenträgheitsmoments $I_\text{D}$ wurden zwei zylinderförmige Massen $m=\SI{222.9}{\gram}$ in verschiedenen Abständen $a$ zur Drehachse befestigt und die Periodendauer $T$ wurde bei einer Initialauslenkung von $\varphi=\SI{10}{\degree}$ gemessen.

Die Massen werden hier als Punktmassen angenommen. Gemessen wurden hierbei allerdings die Abstände von der Drehachse bis zum Anfang des Zylinders. Also lässt sich der gesuchte Abstand mit $a=a_\text{gemessen} + L/2$ berechnen, wobei $L=\SI{3.0}{\centi\meter}$ die Länge des Zylinders ist.

Diese Werte sind in \autoref{tab:eigentraegheit} aufgelistet und in \autoref{fig:plot_eigentraegheit} dargestellt.

\begin{table}
    \centering
    \begin{tabular}{S[table-format=2.1] S[table-format=2.1] S[table-format=1.2]}
        \toprule
        \tableSI{a_\text{gemessen}}{\centi\meter} & \tableSI{a}{\centi\meter} & \tableSI{T}{\second} \\
        \midrule
        26.9 & 28.4 & 7.50 \\
        25.0 & 26.5 & 7.15 \\
        23.0 & 24.5 & 6.58 \\
        21.0 & 22.5 & 6.35 \\
        19.0 & 20.5 & 5.80 \\
        17.0 & 18.5 & 5.32 \\
        15.0 & 16.5 & 4.60 \\
        13.0 & 14.5 & 4.15 \\
        9.0 & 10.5 & 3.32 \\
        4.0 & 5.5 & 2.60 \\
        \bottomrule
    \end{tabular}
    \begin{tabular}{S[table-format=3.1] S[table-format=2.2]}
        \toprule
        \tableSI{a^2}{\centi\meter\squared} & \tableSI{T^2}{\second\squared} \\
        \midrule
        806.6 & 56.25 \\
        702.2 & 51.12 \\
        600.2 & 43.30 \\
        506.2 & 40.32 \\
        420.2 & 33.64 \\
        342.2 & 28.30 \\
        272.2 & 21.16 \\
        210.3 & 17.22 \\
        110.2 & 11.02 \\
        30.2 & 6.76 \\
        \bottomrule
    \end{tabular}
    \caption{Messwerte zur Bestimmung des Eigenträgheitsmoments: Abstände $a_\text{gemessen}$ und $a$, Periodendauer $T$ sowie die geplotteten Werte $a^2$ und $T^2$}
    \label{tab:eigentraegheit}
\end{table}

\begin{figure}
    \centering
    \includegraphics[width=\textwidth]{build/plot_eigentraegheit.pdf}
    \caption{Plot der Quadratperiodendauer $T^2$ in Abhängigkeit des Quadratabstands $a^2$ aus \autoref{tab:eigentraegheit} mit dazugehöriger Ausgleichsgerade}
    \label{fig:plot_eigentraegheit}
\end{figure}

Um nun aus den gemessenen Werten das Eigenträgheitsmoment der Drillachse auszurechnen kann \autoref{eq:periodendauer} verwendet werden. Hierbei ist $I=I_\text{D}+2I_m$ mit dem Trägheitsmoment $I_m=ma^2$ der zusätzlich angehangenen Massen. Somit ergibt sich die Gleichung
\begin{equation}
    T=2\pi \sqrt{\frac{I_\text{D}+2ma^2}{D}}.
    \label{eq:eigentraegheit_theorie}
\end{equation}
Also wird eine Ausgleichsrechnung mit
\begin{equation}
    T^2 = p_1 \cdot a^2 + p_2
    \label{eq:eigentraegheit_fit}
\end{equation}
mithilfe von SciPy ausgeführt.\cite{scipy} Dabei ergeben sich die Parameter $p_1=\SI{660+-18}{\second\squared\per\meter\squared}$ und $p_2=\SI{4.48+-0.86}{\second\squared}$. 

Nach vergleichen der \autoref{eq:eigentraegheit_fit} mit \autoref{eq:eigentraegheit_theorie} lässt sich das Eigenträgheitsmoment mit
\begin{equation}
    I_\text{D} = \frac{D}{(2\pi)^2}p_2
\end{equation}
berechnen. Somit ergibt sich
\begin{equation}
    I_\text{D} = \SI{0.0026+-0.0005}{\kilo\gram\per\meter\squared}
\end{equation}
mit ausgeführter Fehlerrechnung nach \autoref{eq:fehlerrechnung} mit der Python Bibliothek Uncertainties.\cite{uncertainties}