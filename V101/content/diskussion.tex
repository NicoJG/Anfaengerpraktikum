\section{Diskussion}
\label{sec:Diskussion}

% Kurze Zusammenfassung der Ergebnisse
% -Vergleich mit Literaturwerten
% -Vergleich mit verschiedenen Messverfahren
% -bei Abweichungen mögliche Ursachen finden

Die Messung der Eigenschaften der Drillachse ergeben die Winkelrichtgröße
\begin{equation}
    D = a = \SI{0.0237+-0.0001}{\newton\meter}
\end{equation}
und das Eigenträgheitsmoment
\begin{equation}
    I_\text{D} = \SI{0.0026+-0.0005}{\kilo\gram\meter\squared}.
\end{equation}

Damit haben sich die Werte in \autoref{tab:ergebnisse} ergeben.

\begin{table}
    \centering
    \begin{tabular}{c S[table-format=-1.4] S[table-format=1.6] S[table-format=1.4]}
        \toprule
        Körper & \tableSI{I_\text{gemessen}}{\kilo\gram\meter\squared} & \tableSI{I_\text{Theorie}}{\kilo\gram\meter\squared} & \tableSI{\Delta I}{\kilo\gram\meter\squared} \\
        \midrule
        Zylinder (Symmetrieachse) & -0.0022 & 0.000790 & 0.0030 \\
        Zylinder (Querachse) & 0.0001 & 0.003100 & 0.0030 \\
        Puppe (Stellung 1) & -0.0026 & 0.000026 & 0.0026 \\
        Puppe (Stellung 2) & -0.0022 & 0.000310 & 0.0025 \\
        \bottomrule
    \end{tabular}
    \caption{Ergebnisse der Trägheitsmomentbestimmung mit Theoriewert und Abweichung vom Theoriewert $\Delta I$, wobei die Ergebnisse der gemessenen Werte jeweils einen Fehler von $\SI{+-0.0005}{\kilo\gram\meter\squared}$ aufweisen}
    \label{tab:ergebnisse}
\end{table}

