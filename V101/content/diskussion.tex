\section{Diskussion}
\label{sec:Diskussion}

% Kurze Zusammenfassung der Ergebnisse
% -Vergleich mit Literaturwerten
% -Vergleich mit verschiedenen Messverfahren
% -bei Abweichungen mögliche Ursachen finden

Die Messung der Eigenschaften der Drillachse ergeben die Winkelrichtgröße
\begin{equation}
    D = \SI{0.0237+-0.0001}{\newton\meter}
\end{equation}
und das Eigenträgheitsmoment
\begin{equation}
    I_\text{D} = \SI{0.0026+-0.0005}{\kilo\gram\meter\squared}.
\end{equation}

Damit haben sich die Werte in \autoref{tab:ergebnisse} ergeben.

\begin{table}
    \centering
    \begin{tabular}{c S[table-format=-1.4] S[table-format=1.6] S[table-format=1.4]}
        \toprule
        Körper & \tableSI{I_\text{gemessen}}{\kilo\gram\meter\squared} & \tableSI{I_\text{Theorie}}{\kilo\gram\meter\squared} & \tableSI{\Delta I}{\kilo\gram\meter\squared} \\
        \midrule
        Zylinder (Symmetrieachse) & -0.0022 & 0.000790 & 0.0030 \\
        Zylinder (Querachse) & 0.0001 & 0.003100 & 0.0030 \\
        Puppe (Stellung 1) & -0.0026 & 0.000026 & 0.0026 \\
        Puppe (Stellung 2) & -0.0022 & 0.000310 & 0.0025 \\
        \bottomrule
    \end{tabular}
    \caption{Ergebnisse der Trägheitsmomentbestimmung mit Theoriewert und Abweichung vom Theoriewert $\Delta I$, wobei die Ergebnisse der gemessenen Werte jeweils einen Fehler von $\SI{+-0.0005}{\kilo\gram\meter\squared}$ aufweisen}
    \label{tab:ergebnisse}
\end{table}

Die Ergebnisse aus den gemessenen Periodendauern scheinen nicht sinnvoll zu sein, da diese sehr gering und teilweise sogar negativ sind.
Dies lässt sich nur teilweise erklären.

Die wahrscheinlichste Ursache dieser fehlerhaften Trägheitsmomente scheint eine falsche Bestimmung des Eigenträgheitsmoments der Drillachse zu sein.
Dabei wurden nämlich die angehangenen Massen als Punktmassen und die Stange als masselos angenommen.
Dies trägt anscheinend zu einem zu großen Eigenträgheitsmoment bei.

Wenn man in \autoref{tab:ergebnisse} die Abweichungen $\Delta I$ betrachtet fällt eine Ähnlichkeit dieser Werte auf, welche nach \autoref{eq:periodendauer} auf ein zu großes Eigenträgheitsmoment $I_\text{D}$ hinweist. 

Auch ohne diese fehlerhafte Bestimmung würden sich vermutlich relativ große Fehler aufzeigen, da eine genaue Bestimmung der Periodendauern mit der verwendeten Stoppuhr nicht möglich war.

Allerdings ist das berechnete Trägheitsmoment der Puppe in Stellung 2 größer als in Stellung 1.
Dies entspricht den Erwartungen, da sich ein größeres Trägheitsmoment ergibt, wenn die Massen weiter von der Drehachse entfernt sind.

