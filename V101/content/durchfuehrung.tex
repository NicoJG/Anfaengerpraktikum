\section{Durchführung}
\label{sec:Durchführung}

% Was wurde gemessen bzw. welche Größen wurden variiert?
Eine Drillachse wird wie in \autoref{fig:aufbau} aufgebaut.

\begin{figure}
    \centering
    \includegraphics[width=\textwidth/2]{images/aufbau.png}
    \caption{Darstellung einer Drillachse für das beschriebene Experiment \cite{V101}}
    \label{fig:aufbau}
\end{figure}

Zunächst wird die Winkelrichtgrößte $D$ bestimmt. 
Dafür wird eine Federwaage und eine masselose Stange verwendet.
Die Stange wird auf der Drillachse fixifiert und die Federwaage ein der Stange eingehakt. 
Der Abstand von der Rotationsachse bis zum Punkt, an dem die Federwaage eingehakt ist, wird gemessen.
Nun wird das System um einen Winkel $\varphi$ ausgelenkt. 
An der Federwaage wird eine Kraft $F$ abgelesen und in Abhängigkeit von $\varphi$ notiert. 
Diese Messung wird mindestens zehn mal durchgeführt. 
Aus dem Zusammenhang \autoref{eq:winkelrichtgröße} lässt sich $D$ bestimmen.

Für die Bestimmung von $I_\text{D}$ wird erneut eine als masselos anzunehmende Stange verwendet, sowie zwei Gewichte.
Diese Gewichte werden in einem Abstand $a$ von der Rotationsachse an der Stange befestigt und um einen kleinen Winkel $\varphi$ ins Schwingen gebracht. 
Daraus lässt sich mit einer Stoppuhr die Periodendauer $T$ bestimmen. 
Diese Messung wird für mindestens zehn verschiedene Abstände $a$ wiederholt. 

Im eigentlichen Versuch wird jetzt das Trägheitsmoment zwei verschiedene Körper experimentell bestimmt. 
Dafür werden alle verwendeten Körper ausgemessen und ihre Masse $m$ bestimmt. 
Dann wird je ein Körper auf der Drillachse befestigt. 
Das Vorgehen ist ähnlich wie zuvor. 
Der Körper wird für kleine Winkel $\varphi$ ausgelenkt und in eine Schwingung gebracht. 
Mit einer Stoppuhr wird die Periodendauer $T$ mindestens fünf mal bestimmt. 
Dieser Vorgang wird für insgesamt zwei verschiedene Körper durchgeführt.

Im letzten Aufgabenteil wird erneut ein Trägheitsmoment bestimmt. Dafür wird eine Puppe wie in \autoref{puppe} verwendet.

\begin{figure}
    \centering
    \includegraphics[width=\textwidth/2]{images/foto_7.jpg}
    \caption{Darstellung der verwendeten Puppe \cite{V101}}
    \label{fig:puppe}
\end{figure}

Diese wird in zwei verschiedene Positionen gebracht und durch das oben beschriebene Verfahren das entsprechende Trägheitsmoment $I$ dieser Positionen bestimmt. 
Die Puppe wird wieder in eine Schwingung gebracht, wobei die $T$ mit einer Stoppuhr gemessen wird.
Äquivalent zu den Körpern zuvor, wird die Puppe gewogen und die Masse notiert. Außerdem wird die Puppe später nur angenähert, also wird sie an allen wichtigen Stellen vermessen.