\section{Auswertung}
\label{sec:Auswertung}

% Messwerte: Alle gemessenen physikalischen Größen sind übersichtlich darzustellen.

% Auswertung:
% Berechnung der geforderten Endergebnisse
% mit allen Zwischenrechnungen und Fehlerformeln, sodass die Rechnung nachvollziehbar ist.
% Eine kurze Erläuterung der Rechnungen (z.B. verwendete Programme)
% Graphische Darstellung der Ergebnisse

\subsection{Biegung eines einseitig eingespannten zylindrischen Stabes}
\label{sec:Biegung eines einseitig eingespannten zylindrischen Stabes}

In \autoref{tab:einseitig_rund} wurden an je 15 verschiedenen Stellen, die jeweils den Abstand $x$ zu dem Einspannungspunkt besitzen, zwei Messungen durchgeführt. 
$D_0$ beschreibt die Auslenkung ohne eine Masse, also die Ruhelage, $D_m$ die Auslenkung des Stabes mit einer angehängten Masse $m$. Diese wurde in einem Abstand $L$ vom Einspannungspunkt an die Stange gehängt, dabei ist $m$ = $\SI{0.7474}{\kilo\gram}$ und $L$ = $\SI{450}{\milli\meter}$.
$\Delta x$ ist als die Differenz von $D_0$ und $D_m$ definiert, gibt also die eigentliche Auslenkung des Stabes an. 

\begin{table}
    \centering
    \caption{Messergebnisse zu dem einseitig eingespannten zylindrischen Stab}
    \label{tab:einseitig_rund}
    \begin{tabular}{S S S S}
        \toprule
        \tableSI{x}{\milli\meter} & \tableSI{D_0}{\milli\meter} & \tableSI{D_m}{\milli\meter} & \tableSI{\Delta x}{\milli\meter} \\
        \midrule
        25 & 8.01 & 7.95 & 0.06 \\
        50 & 8.06 & 7.89 & 0.17 \\
        75 & 8.13 & 7.81 & 0.32 \\
        100 & 8.19 & 7.66 & 0.53 \\
        150 & 8.32 & 7.25 & 1.07 \\
        200 & 8.47 & 6.71 & 1.76 \\
        250 & 8.61 & 6.05 & 2.56 \\
        300 & 8.72 & 5.27 & 3.35 \\
        350 & 8.74 & 4.38 & 4.36 \\
        375 & 8.73 & 3.88 & 4.85 \\
        400 & 8.74 & 3.37 & 5.37 \\
        410 & 8.68 & 3.14 & 5.54 \\
        420 & 8.68 & 2.93 & 5.75 \\
        430 & 8.68 & 2.71 & 5.97 \\
        435 & 8.69 & 2.62 & 6.07 \\
            \bottomrule
    \end{tabular}
\end{table}

Zunächst muss die Dichte $\rho$ des benutzten Stabs mithilfe von $\rho = m/\pi R^2 L$ berechnet werden. 
Die Stange besitzt insgesamt die Länge $L_\text{gesamt} = \SI{0.55}{\meter}$ und die Masse $m_1$ = $\SI{0.1213}{\kilo\gram}$. 
Der Radius beträgt $R$ = $\SI{5}{\milli\meter}$, somit ergibt sich die Dichte $\rho = \SI{2808.07}{\kilogram \per \cubic\meter}$.

Für die Bestimmung des Elastizitätsmoduls wird $\Delta x$ gegen $x$ aufgetragen.

\begin{figure}
    \centering
    \includegraphics[width=\textwidth]{build/plot_einseitig_rund.pdf}
    \caption{Graph der Werte aus \autoref{tab:einseitig_rund}}
    \label{fig:einseitig_rund_plot}
\end{figure}

Außerdem wurde ein Curve Fit mit SciPy \cite{scipy} und der Funktion aus \autoref{eq:durchbiegung_einseitig} durchgeführt. 
Wobei die folgenden Werte verwendet wurden. 
$F=mg$ ist die Kraft mit der das Gewicht die Stange nach unten verbiegt wird, wobei $g=\SI{9.81}{\meter\per\second\squared}$ die Erdbeschleunigung ist.\cite{physics_constants} 
$L$ ist die Länge der Stange vom Einspannungspunkt aus gemssen, also nicht die gesamte Länge. $R$ ist der Radius der Querschnittsfläche und $I$ ist das Flächenträgheitsmoment, das sich aus \autoref{eq:flächentragheitsmoment} ergibt. 
Im Falle eines Kreises ist das Flächenträgheitsmoment
\begin{equation}
    I_\text{Kreis} = \frac{\pi}{4} \cdot R^4. \: \text{\cite{flaechentraegheitsmomente}}
    \label{eq:flächentragheitsmoment_kreis}
\end{equation}
Da nicht anzunehmen ist, dass der Stab an jeder Stelle den gleichen Durchmesser $2R$ besitzt, wurde an zehn Stellen der Durchmesser gemessen, die Werte weichen nur sehr geringfügig ab. In \autoref{tab:werte_rund_einseitig} wurde der gemittelte Wert für $R$ eingetragen.

\begin{table}
  \centering
  \caption{Gegebene Werte für die Berechnung von E}
  \label{tab:werte_rund_einseitig}
  \begin{tabular}{c c c c}
    \toprule 
    \tableSI{F}{\newton} & \tableSI{L}{\meter} & \tableSI{R}{\meter}& \tableSI{I}{\meter\tothe{4}} \\ 
    \midrule 
     7.332 & 0.450 & 0.005 & $4.910 \cdot 10^{-10}$\\
    \bottomrule
  \end{tabular}
\end{table}  

Aus der Ausgleichskurve wird der Parameter $E$ bestimmt. Für diesen ergibt sich dann
\begin{equation}
    E = 70.40 \pm \SI{0.36}{\giga\pascal}.
    \label{eq:E_einseitig_rund}
\end{equation}

\subsection{Biegung eines einseitig eingespannten rechteckigen Stabes}
\label{sec:Biegung eines einseitig eingespannten rechteckigen Stabes}

Genau wie in \autoref{sec:Biegung eines einseitig eingespannten zylindrischen Stabes} sind in \autoref{tab:einseitig_eckig} die Verschiebungen bei einer bei $L$ angehängten Masse $m = \SI{1.2077}{\kilogram}$ aufgelistet.

\begin{table}
    \centering
    \caption{Messergebnisse zu dem einseitig eingespannten rechteckigen Stab}
    \label{tab:einseitig_eckig}
    \begin{tabular}{S S S S}
        \toprule
        \tableSI{x}{\milli\meter} & \tableSI{D_0}{\milli\meter} & \tableSI{D_m}{\milli\meter} & \tableSI{\Delta x}{\milli\meter} \\
        \midrule
        25 & 7.89 & 7.85 & 0.04\\
        50 & 7.87 & 7.74 & 0.13 \\
        75 & 7.86 & 7.62 & 0.24\\
        100 & 7.81 & 7.41 & 0.40 \\
        150 & 7.63 & 6.83 & 0.80\\
        200 & 7.43 & 6.11 & 1.32\\
        250 & 7.23 & 5.29 & 1.94\\
        300 & 6.98 & 4.38 & 2.60\\
        350 & 6.72 & 3.39 & 3.33\\
        375 & 6.56 & 2.84 & 3.72\\
        400 & 6.37 & 2.27 & 4.10 \\
        410 & 6.28 & 2.03 & 4.25\\
        420 & 6.20 & 1.79 & 4.41\\
        430 & 6.12 & 1.56 & 4.56 \\
        435 & 6.08 & 1.46 & 4.62\\
            \bottomrule
    \end{tabular}
\end{table}

Die Berechnung von $E$ wird hier genau so durchgeführt wie in \autoref{sec:Biegung eines einseitig eingespannten zylindrischen Stabes}. 
Im Folgenden sind alle notwendigen Werte aufgelistet, die für die Berechnung nötig sind. Die Gesamtlänge $L_\text{gesamt}$ des rechteckigen Stabes beträgt $L_\text{gesamt} = \SI{0.6000}{\meter}$, mit einer Masse $m = \SI{0.5023}{\kilogram}$. 
Die Querschnittsfläche ist ein Quadrat, sodass die Seitenlänge $R = \SI{0.0100}{\meter}$ an allen Seiten gleich ist. Aus diesem Informationen kann die Dichte der Stange mit $\rho = m/R^2L_\text{gesamt}$ bestimmt werden. 
Dafür wird exakt wie in \autoref{sec:Biegung eines einseitig eingespannten zylindrischen Stabes} vorgegangen. 
Es ergibt sich die Dichte $\rho = \SI{8371.67}{\kilogram \per \cubic\meter}$.
Wie zuvor wird nun $E$ bestimmt. 
Zunächst werden die Ergebnisse geplottet und mit SciPy und Curve Fit ergänzt. \cite{scipy}

\begin{figure}
    \centering
    \includegraphics[width=\textwidth]{build/plot_einseitig_eckig.pdf}
    \caption{Graph der Werte aus \autoref{tab:einseitig_eckig}}
    \label{fig:einseitig_eckig_plot}
\end{figure}

Da die Querschnittsfläche dieser Stange ein Quadrat ist, muss das Flächenträgheitsmoment $I$ nun mit 

\begin{equation}
    I_\text{Quadtrat} = \frac{R^4}{12}
    \label{eq:flächentragheitsmoment_quadrat}
\end{equation} 
berechnet werden. \cite{flaechentraegheitsmomente}
Damit ergeben sich erneut alle Werte, die zur Berechnung von $E$ und zum Erstellen des Plots gebraucht werden. 
Diese sind in \autoref{tab:werte_eckig_einseitig} aufgelistet.

\begin{table}
  \centering
  \caption{Gegebene Werte für die Berechnung von E}
  \label{tab:werte_eckig_einseitig}
  \begin{tabular}{c c c c}
    \toprule 
    \tableSI{F}{\newton} & \tableSI{L}{\meter} & \tableSI{R}{\meter}& \tableSI{I}{\meter\tothe{4}} \\ 
    \midrule 
     11.85 & 0.446 & 0.01 & $8.33 \cdot 10^{-10}$\\
    \bottomrule
  \end{tabular}
\end{table} 

Für $E$ ergibt sich dann durch die Ausgleichskurve der Elastizitätsmodul
\begin{equation}
    E = 86.64 \pm \SI{0.36}{\giga\pascal}.
    \label{eq:E_einseitig_eckig}
\end{equation}

\FloatBarrier
\subsection{Biegung eines beidseitig eingespannten rechteckigen Stabes}
\label{sec:Biegung eines beidseitig eingespannten rechteckigen Stabes}

Nun wurde die jeweilige Masse $m$ in der Mitte der Stange befestigt, welche beidseitig eingespannt ist.
Auf den beiden Hälften der Stange wurde mit je einer anderen Messuhr gemessen.
In \autoref{tab:beidseitig_eckig} und \autoref{tab:beidseitig_rund} muss beachtet werden, dass die jeweiligen Messuhren nicht gleich kalibriert wurden. 
Dies führt zu abweichenden Auslenkungen, allerdings kann dies vernachlässigt werden, da nur die Differenz der Werte wichtig sind. 
Nachfolgend sind wie zuvor die Auslenkungen $\Delta x$ bei verschiedenen Abständen vom Einspannungspunkt gemessen worden. $D0$ ist die Auslenkung ohne eine angehangene Masse, somit ist $DM$ die Auslenkung mit einer bei $x=\frac{L}{2}$ angehangenen Masse $m$. 
Für die rechteckige Stange beträgt $L = \SI{0.555}{\meter}$, wobei $L$ der Abstand zwischen den beiden Einspannungspunkten ist, und $L_\text{gesamt} = \SI{0.60}{\meter}$ die eigentliche Länge des Stabes ist.

\begin{table}
    \centering
    \caption{Messergebnisse zu dem beidseitig eingespannten rechteckigen Stab}
    \label{tab:beidseitig_eckig}
    \resizebox{\textwidth/2 - 0.1cm}{!}{%
    \begin{tabular}[t]{S S S S}
        \toprule
        \tableSI{x}{\milli\meter} & \tableSI{D_0}{\milli\meter} & \tableSI{D_m}{\milli\meter} & \tableSI{\Delta x}{\milli\meter} \\
        \midrule
        50 & 7.94 & 7.89 & 0.05 \\
        100 & 8.01 & 7.65 & 0.36\\
        120 & 7.97 & 7.55 & 0.42\\
        140 & 7.98 & 7.44 & 0.54\\
        160 & 7.96 & 7.34 & 0.62\\
        180 & 7.91 & 7.18 & 0.73\\
        200 & 7.90 & 7.09 & 0.81\\
        220 & 7.88 & 6.99 & 0.89\\
        240 & 7.88 & 6.92 & 0.96\\
        260 & 7.89 & 6.87 & 1.02\\
            \bottomrule
    \end{tabular}
    }
    \resizebox{\textwidth/2 - 0.1cm}{!}{%
    \begin{tabular}[t]{S S S S}
        \toprule
        \tableSI{x}{\milli\meter} & \tableSI{D_0}{\milli\meter} & \tableSI{D_m}{\milli\meter} & \tableSI{\Delta x}{\milli\meter} \\
        \midrule
        290 & 7.23 & 6.24 & 0.99\\
        310 & 7.22 & 6.23 & 0.99\\
        330 & 7.19 & 6.14 & 1.05\\
        350 & 7.18 & 6.24 & 0.94\\
        370 & 7.18 & 6.29 & 0.89\\
        390 & 7.16 & 6.32 & 0.84\\
        410 & 7.13 & 6.20 & 0.93\\
        430 & 7.13 & 6.45 & 0.68\\
        450 & 7.14 & 6.56 & 0.58\\
        500 & 7.15 & 6.84 & 0.31\\
            \bottomrule
    \end{tabular}
    }
\end{table}

Für diesen Teil des Experimentes wurde die Stange aus \autoref{sec:Biegung eines einseitig eingespannten rechteckigen Stabes} erneut genutzt. 
Die Dichte wurde bereits bestimmt und beträgt $\rho = \SI{8371.67}{\kilogram \per \cubic\meter}$. 
Mithilfe von \autoref{eq:durchbiegung_beidseitig} lässt sich eine Curve Fit Funktion mit Scipy aufstellen, sodass $E$ bestimmt werden kann. 
Die Werte aus \autoref{tab:werte_eckig_beidseitig} werden für das Aufstellen der Funktion benötigt. 
Das Flächenträgheitsmoment $I$ kann mit \autoref{eq:flächentragheitsmoment_quadrat} berechnet werden. 
$R$ ist dabei die Seitenlänge des Querschnitts und $F$ ist die Gewichtskraft, die bei $\frac{L}{2}$ wirkt und durch $F=mg$ gegeben ist. 
Die Erdbeschleunigung ist durch die Konstante $g=\SI{9.81}{\meter\per\second\squared}$ bekannt und die Masse $m$ ist in diesem Fall $m = \SI{4.7188}{\kilogram}$.\cite{physics_constants}
  
\begin{table}
  \centering
  \caption{Gegebene Werte für die Berechnung von E}
  \label{tab:werte_eckig_beidseitig}
  \begin{tabular}{c c c c}
    \toprule 
    \tableSI{F}{\newton} & \tableSI{L}{\meter} & \tableSI{R}{\meter}& \tableSI{I}{\meter\tothe{4}} \\ 
    \midrule 
     46.290 & 0.550 & 0.010 & $8.333 \cdot 10^{-10}$\\
    \bottomrule
  \end{tabular}
\end{table} 

Die Messwerte aus \autoref{tab:beidseitig_eckig} werden wie oben beschrieben mit einer Ausgleichskurve geplottet. 

\begin{figure}
    \centering
    \includegraphics[width=\textwidth]{build/plot_zweiseitig_eckig.pdf}
    \caption{Graph der Werte aus \autoref{tab:beidseitig_eckig}}
    \label{fig:zweiseitg_eckig_plot}
\end{figure}

SciPy berechntet einen Wert $E$.\cite{scipy}

\begin{equation}
    E = 202.04 \pm \SI{6.54}{\giga\pascal}.
    \label{eq:E_beidseitig_eckig}
\end{equation} 

\subsection{Biegung eines beidseitig eingespannten zylindrischen Stabes}
\label{Biegung eines beidseitig eingespannten zylindrischen Stabes}

Für die weitere Messung wurde ein neuer zylindrischer Stab verwerwendet. 
Er besitzt einen Radius von $R = \SI{0.005}{\meter}$ und eine Gesamtlänge von $L_\text{gesamt} = \SI{0.60}{\meter} $. 
Seine Masse $m$ beträgt $m = \SI{0.3943}{\kilogram}$ und hat somit eine Dichte von $\rho = \SI{8367.31}{\kilogram \per \cubic\meter}$. 
Der Abstand zwischen den beiden Einspannungspunkten beträgt $L = \SI{0.555}{\meter}$. 
Die Masse $m = \SI{4.7188}{\newton}$ wurde im Abstand $\frac{L}{2}$ vom Einspannungspunkt angebracht. 

\begin{table}
    \centering
    \caption{Messergebnisse zu dem beidseitig eingespannten zylindrischen Stab}
    \label{tab:beidseitig_rund}
    \resizebox{\textwidth/2 - 0.1cm}{!}{%
    \begin{tabular}[t]{S S S S}
        \toprule
        \tableSI{x}{\milli\meter} & \tableSI{D_0}{\milli\meter} & \tableSI{D_m}{\milli\meter} & \tableSI{\Delta x}{\milli\meter} \\
        \midrule
        50 & 7.91 & 7.73 & 0.18\\
        100 & 7.90 & 7.38 & 0.52\\
        120 & 7.85 & 7.16 & 0.69\\
        140 & 7.80 & 6.93 & 0.87\\
        160 & 7.75 & 6.71 & 1.04\\
        180 & 7.67 & 6.45 & 1.22\\
        200 & 7.66 & 6.28 & 1.38\\
        220 & 7.63 & 6.12 & 1.51\\
        240 & 7.59 & 5.95 & 1.64\\
        260 & 7.56 & 5.84 & 1.72\\
            \bottomrule
    \end{tabular}
    }
    \resizebox{\textwidth/2 - 0.1cm}{!}{%
    \begin{tabular}[t]{S S S S}
        \toprule
        \tableSI{x}{\milli\meter} & \tableSI{D_0}{\milli\meter} & \tableSI{D_m}{\milli\meter} & \tableSI{\Delta x}{\milli\meter} \\
        \midrule
        290 & 6.98 & 5.18 & 1.80\\
        310 & 6.97 & 5.16 & 1.81\\
        330 & 6.96 & 5.18 & 1.79\\
        350 & 6.95 & 5.22 & 1.73\\
        370 & 6.95 & 5.31 & 1.64\\
        390 & 6.95 & 5.42 & 1.53 \\
        410 & 6.95 & 5.57 & 1.38\\
        430 & 6.96 & 5.73 & 1.23\\
        450 & 6.99 & 5.92 & 1.07\\
        500 & 7.05 & 6.46 & 0.59\\
            \bottomrule
    \end{tabular}
    }
\end{table}

Nun wird wie in \autoref{sec:Biegung eines beidseitig eingespannten rechteckigen Stabes} ein Plot mit den Funktionen aus \autoref{eq:durchbiegung_beidseitig} mit den Messwerten angefertigt, aus dem der Elastizitätsmodul $E$ mit SciPy bestimmt werden kann. 
Die Werte aus \autoref{tab:werte_rund_beidseitig} werden für die Berechnung benötigt, dabei ist $R$ der Radius der Querschnittsfläche und $I$, der über \autoref{eq:flächentragheitsmoment_kreis} berechnete Flächenträgheitsmoment.
Für $m$ wird die gleiche Masse, wie in \autoref{sec:Biegung eines beidseitig eingespannten rechteckigen Stabes} verwendet, somit ist $m = \SI{4.7188}{\kilogram}$.

\begin{table}
  \centering
  \caption{Gegebene Werte für die Berechnung von E}
  \label{tab:werte_rund_beidseitig}
  \begin{tabular}{c c c c}
    \toprule 
    \tableSI{F}{\newton} & \tableSI{L}{\meter} & \tableSI{R}{\meter}& \tableSI{I}{\meter\tothe{4}} \\ 
    \midrule 
     46.290 & 0.550 & 0.005 & $4.910 \cdot 10^{-10}$\\
    \bottomrule
  \end{tabular}
\end{table}

Aus \autoref{tab:werte_rund_beidseitig} ergibt sich dann \autoref{fig:zweiseitg_rund_plot} mit Curve Fit aus SciPy und den Messwerten aus \autoref{tab:beidseitig_rund}.\cite{scipy}

\begin{figure}
    \centering
    \includegraphics[width=\textwidth]{build/plot_zweiseitig_rund.pdf}
    \caption{Graph der Werte aus \autoref{tab:beidseitig_rund}}
    \label{fig:zweiseitg_rund_plot}
\end{figure}

Aus \autoref{fig:zweiseitg_rund_plot} ergibt sich
\begin{equation}
    E = 197.86 \pm \SI{6.84}{\giga\pascal}.
    \label{eq:E_beidseitig_rund}
\end{equation}.