\section{Auswertung}
\label{sec:Auswertung}

% Messwerte: Alle gemessenen physikalischen Größen sind übersichtlich darzustellen.

% Auswertung:
% Berechnung der geforderten Endergebnisse
% mit allen Zwischenrechnungen und Fehlerformeln, sodass die Rechnung nachvollziehbar ist.
% Eine kurze Erläuterung der Rechnungen (z.B. verwendete Programme)
% Graphische Darstellung der Ergebnisse

\subsection{Biegung eines einseitig eingespannten zylindrischen Stabes}
\label{sec:Biegung eines einseitig eingespannten zylindrischen Stabes}

In \autoref{tab:einseitig_rund} wurden an je zehn verschiedenen Stellen, die jeweils den Abstand $x$ zu dem Aufhängepunkt besitzen, zwei Messungen durchgeführt. $D_0$ beschreibt die Auslenkung ohne eine Masse, also die Ruhelage, $D_m$ die Auslenkung des Stabes mit einer angehängten Masse $m$. Diese wurde in einem Abstand $L$ vom Aufhängepunkt an die Stange gehängt, dabei ist $m$ = $\SI{0.7474}{\kilo\gram}$ und $L$ = $\SI{450}{\milli\meter}$.
$\Delta x$ ist als die Differenz von $D_0$ und $D_m$ definiert, gibt also die eigentliche Auslenkung des Stabes an. 

\begin{table}
    \centering
    \caption{Messergebnisse zu dem einseitig eingespannten zylindrischen Stab}
    \label{tab:einseitig_rund}
    \begin{tabular}{c c c c}
        \toprule
        \tableSI{x}{\milli\meter} & \tableSI{D_0}{\milli\meter} & \tableSI{D_m}{\milli\meter} & \tableSI{\Delta x}{\milli\meter} \\
        \midrule
        25 & 8.01 & 7.95 & 0.06 \\
        50 & 8.06 & 7.89 & 0.17 \\
        75 & 8.13 & 7.81 & 0.32 \\
        100 & 8.19 & 7.66 & 0.53 \\
        150 & 8.32 & 7.25 & 1.07 \\
        200 & 8.47 & 6.71 & 1.76 \\
        250 & 8.61 & 6.05 & 2.56 \\
        300 & 8.72 & 5.27 & 3.35 \\
        350 & 8.74 & 4.38 & 4.36 \\
        375 & 8.73 & 3.88 & 4.85 \\
        400 & 8.74 & 3.37 & 5.37 \\
        410 & 8.68 & 3.14 & 5.54 \\
        420 & 8.68 & 2.93 & 5.75 \\
        430 & 8.68 & 2.71 & 5.97 \\
        435 & 8.69 & 2.62 & 6.07 \\
            \bottomrule
    \end{tabular}
\end{table}

Um den Elastizitätsmodul $E$ zu bestimmen wird zunächst die runde Stange aus \autoref{tab:einseitig_rund} näher untersucht. Sie besitzt insgesamt die Länge $L_\text{gesamt} = \SI{0.55}{\meter}$ und die Masse $m_1$ = $\SI{0.1213}{\kilo\gram}$. Der Radius beträgt $R$ = $\SI{5}{\milli\meter}$, somit ist das Volumen $V = \SI{0.000035342}{\cubic\meter}$, dadurch kann auch die Dichte mit FORMEL ausgerechnet werden. Es ergibt sich $\rho = \SI{2808.07}{\kilogram \per \cubic\meter}$, was in etwa der Dichte von Aluminium entspricht. 
Für die Auswertung wird $\Delta x$ gegen $x$ aufgetragen.

\begin{figure}
    \centering
    \includegraphics[width=\textwidth]{build/plot_einseitig_rund.pdf}
    \caption{Graph der Werte aus \autoref{tab:einseitig_rund}}
    \label{fig:einseitig_rund_plot}
\end{figure}

Außerdem wurde ein Curve Fit mit SyPy durchgeführt, mit der Funktion \autoref{eq:durchbiegung_einseitig}. Wobei folgende gegebene Werte verwendet wurden. $F$ ist die Kraft mit der das Gewicht die Stange nach unten verbiegt, $L$ ist die Länge der Stange vom Aufhängepunkt aus gemssen, also nicht die gesamte Länge. $R$ ist der Radius der Querschnittsfläche und $I$ ist der Flächenträgheitsmoment, der sich aus \autoref{eq:flächentragheitsmoment} ergibt.
Im Falle eines Kreises:

\begin{equation}
    I_\text{Kreis} = \frac{\pi}{4} \cdot R^4
    \label{eq:flächentragheitsmoment_kreis}
\end{equation}

\begin{table}
  \centering
  \caption{Gegebene Werte für die Berechnung von E}
  \label{tab:werte_rund_einseitig}
  \begin{tabular}{c c c c}
    \toprule 
    \tableSI{F}{\newton} & \tableSI{L}{\meter} & \tableSI{R}{\meter}& \tableSI{I}{\meter\tothe{4}} \\ 
    \midrule 
     7.332 & 0.450 & 0.005 & $4.910 \cdot 10^{-10}$\\
    \bottomrule
  \end{tabular}
\end{table}  

Aus Curve Fit wird der Parameter $E$, der bestimmt werden soll, zurückgegeben. Für diesen ergibt sich dann
\begin{equation}
    E = 70.40 \pm \SI{0.36}{\giga\pascal}
    \label{eq:E_einseitig_rund}
\end{equation}

Der Theoriewert für den Elastizitätsmodul von Aluminium beträgt $\SI{70}{\giga\pascal}$. Es wurde also bestägigt, dass es sich bei diesem Material um Aluminium handelt.

\subsection{Biegung eines einseitig eingespannten rechteckigen Stabes}
\label{sec:Biegung eines einseitig eingespannten rechteckigen Stabes}

Genau wie in \autoref{sec:Biegung eines einseitig eingespannten zylindrischen Stabes} sind hier die Verschiebungen 

\begin{table}
    \centering
    \caption{Messergebnisse zu dem einseitig eingespannten rechteckigen Stab}
    \label{tab:einseitig_eckig}
    \begin{tabular}{c c c c}
        \toprule
        \tableSI{x}{\milli\meter} & \tableSI{D_0}{\milli\meter} & \tableSI{D_m}{\milli\meter} & \tableSI{\Delta x}{\milli\meter} \\
        \midrule
        25 & 7.89 & 7.85 & 0.04\\
        50 & 7.87 & 7.74 & 0.13 \\
        75 & 7.86 & 7.62 & 0.24\\
        100 & 7.81 & 7.41 & 0.40 \\
        150 & 7.63 & 6.83 & 0.80\\
        200 & 7.43 & 6.11 & 1.32\\
        250 & 7.23 & 5.29 & 1.94\\
        300 & 6.98 & 4.38 & 2.60\\
        350 & 6.72 & 3.39 & 3.33\\
        375 & 6.56 & 2.84 & 3.72\\
        400 & 6.37 & 2.27 & 4.10 \\
        410 & 6.28 & 2.03 & 4.25\\
        420 & 6.20 & 1.79 & 4.41\\
        430 & 6.12 & 1.56 & 4.56 \\
        435 & 6.08 & 1.46 & 4.62\\
            \bottomrule
    \end{tabular}
\end{table}

Die Berechnung von E wird hier genau so durchgeführt wie in \autoref{sec:Biegung eines einseitig eingespannten zylindrischen Stabes}. Im Folgenden sind alle notwendigen Werte aufgelistet, die für die Berechnung nötig sind. Die Gesamtlänge $L$ des rechteckigen Stabes beträgt $L_\text{gesamt} = \SI{0.6000}{\meter}$, mit einer Masse $m = \SI{0.5023}{\kilogram}$. Die Querschnittsfläche ist ein Quadrat, sodass die Seitenlänge $R = \SI{0.0100}{\meter}$ an allen Seiten gleich ist. Aus diesem Informationen kann die Dichte der Stange, und somit auch das Material bestimmt werden. Dafür wird exakt wie in \autoref{sec:Biegung eines einseitig eingespannten zylindrischen Stabes} vorgegangen. Es ergibt sich für $\rho = \SI{8371.67}{\kilogram \per \cubic\meter}$. Diese Dichte entspricht etwa der Dichte von Messing. Wie zuvor wird nun E bestimmt und erneut mit dem Material verglichen. Zunächst werdne die Ergebnisse geplottet und mit SyPy und Curve Fit ergänzt. 

\begin{figure}
    \centering
    \includegraphics[width=\textwidth]{build/plot_einseitig_eckig.pdf}
    \caption{Graph der Werte aus \autoref{tab:einseitig_eckig}}
    \label{fig:einseitig_eckig_plot}
\end{figure}

Da die Querschnittsfläche dieser Stange ein Quadrat ist, muss der Flächenträgheitsmoment $I$ nun mit 

\begin{equation}
    I_\text{Quadtrat} = \frac{R^4}{12} berechnet werden. 
    \label{eq:flächentragheitsmoment_quadrat}
\end{equation}

Damit ergeben sich erneut alle Werte, die zur Berechnung von E und zum Erstellen des Plots gebraucht werden:

\begin{table}
  \centering
  \caption{Gegebene Werte für die Berechnung von E}
  \label{tab:werte_rund_einseitig}
  \begin{tabular}{c c c c}
    \toprule 
    \tableSI{F}{\newton} & \tableSI{L}{\meter} & \tableSI{R}{\meter}& \tableSI{I}{\meter\tothe{4}} \\ 
    \midrule 
     7.33211.85 & 0.446 & 0.01 & $8.33 \cdot 10^{-10}$\\
    \bottomrule
  \end{tabular}
\end{table} 

Für $E$ ergibt sich dann mit Curve Fit $E = 86.64 \pm \SI{0.36}{\giga\pascal}$. Damit liegt $E$ im Bereich für den Elastizitätsmodul von Messing, ca $78$ bis $\SI{123}{\giga\pascal}$.

\subsection{Biegung eines zweiseitig eingespannten rechteckigen Stabes}
\label{sec:Biegung eines zweiseitig eingespannten rechteckigen Stabes}

In den weiteren beiden Auswertung gibt es nun einen Unterschied zu den vorherigen beiden. Die jeweiligen Masse $m$ wurde in der Mitte der Stange befestigt, auf den beiden Hälften der Stange wurde mit je einer anderen Messuhr gemessen. Die beiden Hälften von \autoref{tab:zweiseitig_eckig} und \autoref{tab:zweiseitig_rund} sind je mit beiden Uhren gemessen worden, da es nur auf die Diffenrenz der Werte ankommt, sei die Abweichen der Uhren voneinander zu vernachlässigen. 

\begin{table}
    \centering
    \caption{Messergebnisse zu dem zweiseitig eingespannten rechteckigen Stab}
    \label{tab:zweiseitig_eckig}
    \resizebox{\textwidth/2 - 0.1cm}{!}{%
    \begin{tabular}[t]{c c c c}
        \toprule
        \tableSI{x}{\milli\meter} & \tableSI{D_0}{\milli\meter} & \tableSI{D_m}{\milli\meter} & \tableSI{\Delta x}{\milli\meter} \\
        \midrule
        50 & 7.94 & 7.89 & 0.05 \\
        100 & 8.01 & 7.65 & 0.36\\
        120 & 7.97 & 7.55 & 0.42\\
        140 & 7.98 & 7.44 & 0.54\\
        160 & 7.96 & 7.34 & 0.62\\
        180 & 7.91 & 7.18 & 0.73\\
        200 & 7.90 & 7.09 & 0.81\\
        220 & 7.88 & 6.99 & 0.89\\
        240 & 7.88 & 6.92 & 0.96\\
        260 & 7.89 & 6.87 & 1.02\\
            \bottomrule
    \end{tabular}
    }
    \resizebox{\textwidth/2 - 0.1cm}{!}{%
    \begin{tabular}[t]{c c c c}
        \toprule
        \tableSI{x}{\milli\meter} & \tableSI{D_0}{\milli\meter} & \tableSI{D_m}{\milli\meter} & \tableSI{\Delta x}{\milli\meter} \\
        \midrule
        290 & 7.23 & 6.24 & 0.99\\
        310 & 7.22 & 6.23 & 0.99\\
        330 & 7.19 & 6.14 & 1.05\\
        350 & 7.18 & 6.24 & 0.94\\
        370 & 7.18 & 6.29 & 0.89\\
        390 & 7.16 & 6.32 & 0.84\\
        410 & 7.13 & 6.20 & 0.93\\
        430 & 7.13 & 6.45 & 0.68\\
        450 & 7.14 & 6.56 & 0.58\\
        500 & 7.15 & 6.84 & 0.31\\
            \bottomrule
    \end{tabular}
    }
\end{table}

\subsection{Biegung eines zweiseitig eingespannten zylindrischen Stabes}

\begin{table}
    \centering
    \caption{Messergebnisse zu dem zweiseitig eingespannten zylindrischen Stab}
    \label{tab:zweiseitig_rund}
    \resizebox{\textwidth/2 - 0.1cm}{!}{%
    \begin{tabular}[t]{c c c c}
        \toprule
        \tableSI{x}{\milli\meter} & \tableSI{D_0}{\milli\meter} & \tableSI{D_m}{\milli\meter} & \tableSI{\Delta x}{\milli\meter} \\
        \midrule
        50 & 7.91 & 7.73 & 0.18\\
        100 & 7.90 & 7.38 & 0.52\\
        120 & 7.85 & 7.16 & 0.69\\
        140 & 7.80 & 6.93 & 0.87\\
        160 & 7.75 & 6.71 & 1.04\\
        180 & 7.67 & 6.45 & 1.22\\
        200 & 7.66 & 6.28 & 1.38\\
        220 & 7.63 & 6.12 & 1.51\\
        240 & 7.59 & 5.95 & 1.64\\
        260 & 7.56 & 5.84 & 1.72\\
            \bottomrule
    \end{tabular}
    }
    \resizebox{\textwidth/2 - 0.1cm}{!}{%
    \begin{tabular}[t]{c c c c}
        \toprule
        \tableSI{x}{\milli\meter} & \tableSI{D_0}{\milli\meter} & \tableSI{D_m}{\milli\meter} & \tableSI{\Delta x}{\milli\meter} \\
        \midrule
        290 & 6.98 & 5.18 & 1.80\\
        310 & 6.97 & 5.16 & 1.81\\
        330 & 6.96 & 5.18 & 1.79\\
        350 & 6.95 & 5.22 & 1.73\\
        370 & 6.95 & 5.31 & 1.64\\
        390 & 6.95 & 5.42 & 1.53 \\
        410 & 6.95 & 5.57 & 1.38\\
        430 & 6.96 & 5.73 & 1.23\\
        450 & 6.99 & 5.92 & 1.07\\
        500 & 7.05 & 6.46 & 0.59\\
            \bottomrule
    \end{tabular}
    }
\end{table}