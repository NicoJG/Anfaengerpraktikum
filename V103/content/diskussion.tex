\section{Diskussion}
\label{sec:Diskussion}

% Kurze Zusammenfassung der Ergebnisse
% -Vergleich mit Literaturwerten
% -Vergleich mit verschiedenen Messverfahren
% -bei Abweichungen mögliche Ursachen finden

Für die Auswertung der Elastizitätsmodule $E$ wurden als Vergleichswerte für die Bestimmung des Stangenmaterials ebenfalls die Dichten $\rho$ bestimmt. Dadurch ergeben sich die Werte aus \autoref{tab:rho_ergebnisse}.

\begin{table}
  \centering
  \caption{Ergebnisse für $\rho$}
  \label{tab:rho_ergebnisse}
  \begin{tabular}{c S S S}
    \toprule 
   \text{Stab} & \tableSI{\rho _\text{gemessen}}{\kilogram \per \cubic\meter} & \tableSI{\rho _\text{theorie}}{\kilogram \per \cubic\meter} & \text{Prozentuale Abweichungen} \\ 
    \midrule 
    einseitig zylindrisch & 2808.07 & {2560,00 \text{bis} 2640,00} & 6.36 \% \\
    einseitig rechteckig & 8371.67 & {8400,00 \text{bis} 8700,00} & 0.33 \% \\
    beidseitig rechteckig & 8371.67 & {8400,00 \text{bis} 8700,00} & 0.33 \% \\
    beidseitig zylindrisch & 8367.31 & {8400,00 \text{bis} 8700,00} & 0.39 \% \\
    \bottomrule
  \end{tabular}
\end{table}

$\rho _2$, $\rho _3$ und $\rho _4$ entsprechen am ehesten der Dichte von Messing. $\rho _1$ lässt sich auf die Dichte von Aluminium zurückführen.\cite{dichten_metalle}

Aus den Auswertungen ergeben sich zudem vier Elastizitätsmodule mit ihren Unsicherheiten. Im Folgenden werden sie mit den ihnen entsprechenden Theoriewerten:

\begin{table}
  \centering
  \caption{Ergebnisse für E}
  \label{tab:e_ergebnisse}
  \begin{tabular}{c S S S S}
    \toprule 
   \text{Stab} & \tableSI{E_\text{gemessen}}{\giga\pascal} & \tableSI{E_\text{gemessen}}{\giga\pascal} & \tableSI{E_\text{theorie}}{\giga\pascal} & \text{Prozentuale Abweichungen}\\ 
    \midrule 
    einseitig zylindrisch & 70.40 & 0.36 & 69.00 & 2.03 \pm 0.52 \% \\
    einseitig rechteckig & 86.64 & 0.36 & {102,00 \text{bis} 125,00} & 15.06 \pm 0.35 \% \\
    beidseitig rechteckig & 202.04 & 6.54 & {102,00 \text{bis} 125,00} & 61.63 \pm 5.23 \% \\
    beidseitig zylindrisch & 197.86 & 6.84 & {102,00 \text{bis} 125,00} & 57.60 \pm 5.47 \% \\
    \bottomrule
  \end{tabular}
\end{table}