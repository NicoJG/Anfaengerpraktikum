\section{Diskussion}
\label{sec:Diskussion}

% Kurze Zusammenfassung der Ergebnisse
% -Vergleich mit Literaturwerten
% -Vergleich mit verschiedenen Messverfahren
% -bei Abweichungen mögliche Ursachen finden

Für die Auswertung der Elastizitätsmodule $E$ wurden als Vergleichswerte für die Bestimmung des Stangenmaterials ebenfalls die Dichten $\rho$ bestimmt. Dabei ergaben sich:

\begin{table}
  \centering
  \caption{Ergebnisse für \rho }
  \label{tab:rho_ergebnisse}
  \begin{tabular}{S S S}
    \toprule 
   Stab & \tableSI{\rho _\text{gemessen}}{\kilogram \per \cubic\meter} & \tableSI{\rho _\text{theorie}}{\kilogram \per \cubic\meter} \\ 
    \midrule 
    einseitig zylindrisch & 2808.07 & 2560.00 bis 2640.00 \\
    einseitig rechteckig & 8371.67 & 8400.00 bis 8700.00 \\
    beidseitig rechteckig & 8371.67 & 8400.00 bis 8700.00 \\
    beidseitig zylindrisch & 8367.31 & 8400.00 bis 8700.00 \\
    \bottomrule
  \end{tabular}
\end{table}

$\rho _2$, $\rho _3$ und $\rho _4$ entsprechen am ehesten der Dichte von Messing. $\rho _1$ lässt sich auf die Dichte von Aluminium zurückführen.

Aus den Auswertungen ergeben sich vier Elastizitätsmodule:
\begin{align}
    E_1 =& \: 70.40 \pm \SI{0.36}{\giga\pascal} \\
    E_2 =& \: 86.64 \pm \SI{0.36}{\giga\pascal} \\
    E_3 =& \: 202.04 \pm \SI{6.54}{\giga\pascal} \\
    E_4 =& \: 197.86 \pm \SI{6.84}{\giga\pascal}
\end{align}