\section{Diskussion}
\label{sec:Diskussion}

% Kurze Zusammenfassung der Ergebnisse
% -Vergleich mit Literaturwerten
% -Vergleich mit verschiedenen Messverfahren
% -bei Abweichungen mögliche Ursachen finden

Für die Auswertung der Elastizitätsmodule $E$ wurden als Vergleichswerte für die Bestimmung des Stangenmaterials ebenfalls die Dichten $\rho$ bestimmt. Dadurch ergeben sich die Werte aus \autoref{tab:rho_ergebnisse}.

\begin{table}
  \centering
  \caption{Ergebnisse für die Dichte $\rho$ mit entsprechenden Referenzwerten aus der Literatur für ein Metall mit ähnlicher Dichte\cite{dichten_metalle}}
  \label{tab:rho_ergebnisse}
  \begin{tabular}{c S c S}
    \toprule 
   \text{Stab} & \tableSI{\rho _\text{gemessen}}{\kilogram \per \cubic\meter} & \tableSI{\rho _\text{referenz}}{\kilogram \per \cubic\meter} & \text{Abweichungen} \\ 
    \midrule 
    einseitig zylindrisch & 2808.07 & {2560 \text{bis} 2640} & 6.36 \% \\
    einseitig rechteckig & 8371.67 & {8400 \text{bis} 8700} & 0.33 \% \\
    beidseitig rechteckig & 8371.67 & {8400 \text{bis} 8700} & 0.33 \% \\
    beidseitig zylindrisch & 8367.31 & {8400 \text{bis} 8700} & 0.39 \% \\
    \bottomrule
  \end{tabular}
\end{table}

Die Dichte des ersten Stabs entspricht am ehesten der Dichte von Aluminium. Somit ist der erste Referenzwert in \autoref{tab:rho_ergebnisse} für die Dichte von Aluminium.
Alle anderen Dichten entsprechen am ehesten der Dichte von Messing und somit wurde hierfür der Referenzwert gelistet.

Aus den Auswertungen ergeben sich zudem vier Elastizitätsmodule mit ihren Unsicherheiten. Diese sind in \autoref{tab:e_ergebnisse} mit Referenzwerten der zuvor bestimmten Metalle gelistet.

\begin{table}
  \centering
  \caption{Ergebnisse für $E$ mit entsprechenden Referenzwerten\cite{youngs_modulus}}
  \label{tab:e_ergebnisse}
  \resizebox{\textwidth - 0.1cm}{!}{%
  \begin{tabular}{c S S S S}
    \toprule 
   \text{Stab} & \tableSI{E}{\giga\pascal} & \tableSI{\Delta E}{\giga\pascal} & \tableSI{E_\text{referenz}}{\giga\pascal} & {Abweichungen}\\ 
    \midrule 
    einseitig zylindrisch & 70.40 & \pm 0.36 & 69 & 2.03 \pm 0.52 \% \\
    einseitig rechteckig & 86.64 & \pm 0.36 & {102 \text{bis} 125} & 15.06 \pm 0.35 \% \\
    beidseitig rechteckig & 176.00 & \pm 11.00 & {102 \text{bis} 125} & 40.80 \pm 9.12 \% \\
    beidseitig zylindrisch & 173.00 & \pm 5.00 & {102 \text{bis} 125} & 38.40 \pm 0.96 \% \\
    \bottomrule
  \end{tabular}
  }
\end{table}

An \autoref{tab:e_ergebnisse} ist zu sehen, dass die Messungen der beidseitig eingespannten Stäbe deutlich höhere Abweichungen haben, als die der einseitigen Einspannung.

Diese Abweichungen lassen sich durch die erzeugte Biegung erklären, welche deutlich geringer ist, als die empfohlenen \SIrange{3}{7}{\milli\meter}.
Bei so geringen Auslenkungen können in der Theorie gemachte Vereinfachungen und die Messgenauigkeiten der Messvorrichtungen zu erheblichen Abweichungen führen.

Somit scheint eine Messung bei einseitiger Einspannung deutlich sinnvoller und effektiver.