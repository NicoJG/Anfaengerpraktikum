\section{Auswertung}
\label{sec:Auswertung}

% Messwerte: Alle gemessenen physikalischen Größen sind übersichtlich darzustellen.

% Auswertung:
% Berechnung der geforderten Endergebnisse
% mit allen Zwischenrechnungen und Fehlerformeln, sodass die Rechnung nachvollziehbar ist.
% Eine kurze Erläuterung der Rechnungen (z.B. verwendete Programme)
% Graphische Darstellung der Ergebnisse

Aus den gemessenen Werten in \autoref{tab:niedrigdruck_messwerte} und \autoref{tab:hochdruck_messwerte} wird nun die Verdampfungswärme von Wasser abgeschätzt.

%%%%%%%%%%%%%%%%%%%%%%%%%%%%%%%%%%%%%%%%%%%%%%%%%%%%
\subsection{Messung im Druckbereich unterhalb 1 bar}
\label{sec:Auswertung_niedrigdruck}

Die Durchführung wie in \autoref{sec:Durchführung_niedrigdruck} beschrieben ergab die Messwerte in \autoref{tab:niedrigdruck_messwerte}.

\begin{table}
    \centering
    \begin{tabular}{S[table-format=3.0] S[table-format=3.0]}
        \toprule
        \tableSI{T}{\celsius} & \tableSI{p}{\milli\bar} \\
        \midrule
        29 & 68 \\
        30 & 72 \\
        31 & 75 \\
        32 & 77 \\
        33 & 79 \\
        34 & 80 \\
        35 & 83 \\
        36 & 86 \\
        37 & 88 \\
        38 & 91 \\
        39 & 94 \\
        40 & 97 \\
        41 & 101 \\
        42 & 111 \\
        43 & 113 \\
        44 & 115 \\
        45 & 119 \\
        46 & 124 \\
        47 & 129 \\
        48 & 133 \\
        49 & 140 \\
        50 & 147 \\
        51 & 152 \\
        52 & 159 \\
        53 & 166 \\
        54 & 172 \\
        55 & 180 \\
        56 & 188 \\
        \bottomrule
    \end{tabular}
    \begin{tabular}{S[table-format=3.0] S[table-format=3.0]}
        \toprule
        \tableSI{T}{\celsius} & \tableSI{p}{\milli\bar} \\
        \midrule
        57 & 195 \\
        58 & 204 \\
        59 & 212 \\
        60 & 221 \\
        61 & 229 \\
        62 & 238 \\
        63 & 249 \\
        64 & 258 \\
        65 & 269 \\
        66 & 276 \\
        67 & 289 \\
        68 & 299 \\
        69 & 304 \\
        70 & 315 \\
        71 & 330 \\
        72 & 342 \\
        73 & 355 \\
        74 & 366 \\
        75 & 379 \\
        76 & 394 \\
        77 & 405 \\
        78 & 419 \\
        79 & 430 \\
        80 & 447 \\
        82 & 477 \\
        84 & 502 \\
        85 & 516 \\
        86 & 527 \\
        \bottomrule
    \end{tabular}
    \begin{tabular}{S[table-format=3.0] S[table-format=3.0]}
        \toprule
        \tableSI{T}{\celsius} & \tableSI{p}{\milli\bar} \\
        \midrule
        87 & 539 \\
        88 & 552 \\
        89 & 565 \\
        90 & 580 \\
        91 & 595 \\
        92 & 611 \\
        93 & 623 \\
        94 & 643 \\
        95 & 651 \\
        96 & 665 \\
        97 & 681 \\
        98 & 690 \\
        99 & 702 \\
        100 & 720 \\
        101 & 733 \\
        102 & 749 \\
        103 & 770 \\
        104 & 786 \\
        105 & 809 \\
        106 & 831 \\
        107 & 857 \\
        108 & 874 \\
        109 & 910 \\
        110 & 940 \\
        111 & 970 \\
        112 & 996 \\
        & \\
        & \\
        \bottomrule
    \end{tabular}
    \caption{Messwerte im Druckbereich unterhalb \SI{1}{bar}}
    \label{tab:niedrigdruck_messwerte}
\end{table}

Mithilfe dieser Messwerte wird nun ein Plot angefertig, in dem, auf Grundlage von \autoref{eq:druck}, $\ln(p)$ gegen $1/T$ aufgetragen wird.
Dieser Plot ist in \autoref{fig:plot_niedrigdruck} zu sehen.

Dann wird mit der Gleichung
\begin{equation}
    \ln(p) = \frac{a}{T} + b
\end{equation}
und der Funktion curve\_fit aus der Python Bibliothek SciPy eine Ausgleichsgerade für die Parameter $a$ und $b$ erstellt.\cite{scipy} Hierdurch ergeben sich die Parameter
\begin{align*}
    a =& \SI{-881+-21}{\kelvin} \\
    b =& \SI{14.40+-0.06}{}.
    \label{eq:niedrigdruck_parameter}
\end{align*}

Nach \autoref{eq:Druck} wird nun die Verdampfungswärme $L$ über
\begin{equation}
    L = -R \cdot a
\end{equation}
berechnet und ergibt
\begin{equation*}
    L = \SI{7327+-177}{\joule\per\mol}
\end{equation*}
wobei $R=\SI{8.314}{\joule\per\mol\kelvin}$ die allgemeine Gaskonstante ist.\cite{physics_constants}

\begin{figure}
    \centering
    \includegraphics[width=\textwidth]{build/plot_niedrigdruck.pdf}
    \caption{Plot des Messergebnisses im Niedrigdruckbereich mit dem Druck $p$ und der Temperatur $T$}
    \label{fig:plot_niedrigdruck}
\end{figure}

Um nun die innere Verdampfungswärme $L_\text{i}$, also die benötigte Arbeit um bei der Verdampfung die Anziehungskräfte der Teilchen zu überwinden, bei einer Temperatur $T=373$ abschätzen zu können wird
\begin{equation}
    L_\text{i} = L - L_\text{a}
\end{equation}
verwendet. 
Hier ist $L_\text{a}$ die äußere Verdampfungswärme, also die Arbeit, die benötigt wird um das Volumen im flüssigen Zustand auf das Volumen um gasförmigen Zustand auszudehnen.\cite{V203}

Diese Arbeit entspricht $pV$ und ergibt über die Allgemeine Gasgleichung
\begin{equation*}
    L_\text{a} = R \cdot T = \SI{3101.122}{\joule\per\mol}.
\end{equation*}

Hieraus ergibt sich die innere Verdampfungswärme
\begin{equation*}
    L_\text{i} = \SI{4226+-176}{\joule\per\mol}.
\end{equation*}
Um diese Größe besser darstellen zu können wird pro Molekül und in Elektronenvolt angegeben.
Diese Umrechnung geschieht mit der Avogadrokonstante $N_\text{a} = \SI{6.022e23}{\per\mol}$ und dem Elektronenvolt $1 eV = \SI{1.602e-19}{\joule}$ und ergibt
\begin{equation*}
    L_\text{i} = \SI{0.044+-0.002}{\electronvolt}. \: \text{\cite{physics_constants}}
\end{equation*}


%%%%%%%%%%%%%%%%%%%%%%%%%%%%%%%%%%%%%%%%%%%%%%%%%%%%
\subsection{Messung im Druckbereich oberhalb 1 bar}
\label{sec:Auswertung_hochdruck}

Die Messung im Hochdruckbereich, wie in \autoref{sec:Durchführung_hochdruck} beschrieben, ergab die Messergebnisse in \autoref{tab:hochdruck_messwerte}.
Aus diesen Werten wird nun eine Abhängigkeit der Verdampfungswärme zur Temperatur bestimmt.

\begin{table}
    \centering
    \begin{tabular}{S[table-format=3.0] S[table-format=2.1]}
        \toprule
        \tableSI{T}{\celsius} & \tableSI{p}{\bar} \\
        \midrule
        103 & 0.5 \\
        118 & 1.0 \\
        125 & 1.5 \\
        130 & 2.0 \\
        136 & 2.5 \\
        141 & 3.0 \\
        145 & 3.5 \\
        149 & 4.0 \\
        153 & 4.5 \\
        156 & 5.0 \\
        159 & 5.5 \\
        162 & 6.0 \\
        164 & 6.5 \\
        167 & 7.0 \\
        170 & 7.5 \\
        \bottomrule
    \end{tabular}
    \begin{tabular}{S[table-format=3.0] S[table-format=2.1]}
        \toprule
        \tableSI{T}{\celsius} & \tableSI{p}{\bar} \\
        \midrule
        172 & 8.0 \\
        174 & 8.5 \\
        176 & 9.0 \\
        178 & 9.5 \\
        180 & 10.0 \\
        182 & 10.5 \\
        184 & 11.0 \\
        186 & 11.5 \\
        188 & 12.0 \\
        189 & 12.5 \\
        190 & 13.0 \\
        192 & 13.5 \\
        193 & 14.0 \\
        195 & 14.5 \\
        197 & 15.0 \\
        \bottomrule
    \end{tabular}
    \caption{Messergebnisse im Hochdruckbereich}
    \label{tab:hochdruck_messwerte}
\end{table}

Dafür wird zunächst aus den Messwerten ein Plot angefertigt und mit
\begin{equation}
    p = a \cdot T^3 + b \cdot T^2 + c \cdot T + d
    \label{eq:hochdruck_fit}
\end{equation}
und der Funktion curve\_fit aus der Python Bibliothek ScyPi eine Ausgleichskurve erstellt.\cite{scipy}
Der Plot ist in \autoref{fig:plot_hochdruck} zu sehen.
Hier ergeben sich die Parameter
\begin{align*}
    a =& \SI{0.8+-0.1}{\pascal\per\kelvin\cubed} \\
    b =& \SI{-862+-130}{\pascal\per\kelvin\squared} \\
    c =& \SI{3.1+-0.6e5}{\pascal\per\kelvin} \\
    d =& \SI{-3.7+-0.8e7}{\pascal}.
\end{align*}

\begin{figure}
    \centering
    \includegraphics[width=\textwidth]{build/plot_hochdruck.pdf}
    \caption{Plot des Messergebnisses im Hochdruckbereich mit dem Druck $p$ und der Temperatur $T$}
    \label{fig:plot_hochdruck}
\end{figure}

Die Abhängigkeit der Verdampfungswärme zur Temperatur ergibt sich über \autoref{eq:clapeyron} zu
\begin{equation}
    L = (V_\text{D}-V_\text{F}) \cdot T \cdot \frac{\dif{p}}{\dif{T}}.
    \label{eq:hochdruck_ergebnis}
\end{equation}
Hierbei ergibt sich $\dif{p}/\dif{T}$ aus \autoref{eq:hochdruck_fit} zu
\begin{equation}
    \frac{\dif{p}}{\dif{T}} = 3 \cdot a \cdot T^2 + 2 \cdot b \cdot T + c.
    \label{eq:ableitung}
\end{equation}

Wie auch in der Theorie beschrieben kann $V_\text{F}$ vernachlässigt werden. 
Nun muss allerdings $V_\text{D}$ über eine Erweiterung der idealen Gasgleichung
\begin{equation}
    \left( p + \frac{A}{V^2} \right) \cdot V = R \cdot T
\end{equation}
mit $A = \SI{0.9}{\joule\meter\cubed\per\mol\squared}$ bestimmt werden und ergibt die zwei Lösungen
\begin{equation}
    V_{\text{D},\pm} = \frac{R \cdot T}{2 \cdot p} \pm \sqrt{\frac{R^2 \cdot T^2}{4 \cdot p^2} - \frac{A}{p}}
\end{equation}
wobei für $p$ \autoref{eq:hochdruck_fit} verwendet wird.

\autoref{fig:plot_hochdruck_plus} und \autoref{fig:plot_hochdruck_minus} zeigen \autoref{eq:hochdruck_ergebnis} eingesetzt mit allen benannten Werten und Gleichungen.

\begin{figure}
    \centering
    \includegraphics{build/plot_hochdruck_plus.pdf}
    \caption{Plot der Temperaturabhängigkeit im Falle von $V_{\text{D},+}$}
    \label{fig:plot_hochdruck_plus}
\end{figure}

\begin{figure}
    \centering
    \includegraphics{build/plot_hochdruck_minus.pdf}
    \caption{Plot der Temperaturabhängigkeit im Falle von $V_{\text{D},-}$}
    \label{fig:plot_hochdruck_minus}
\end{figure}