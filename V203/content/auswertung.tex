\section{Auswertung}
\label{sec:Auswertung}

% Messwerte: Alle gemessenen physikalischen Größen sind übersichtlich darzustellen.

% Auswertung:
% Berechnung der geforderten Endergebnisse
% mit allen Zwischenrechnungen und Fehlerformeln, sodass die Rechnung nachvollziehbar ist.
% Eine kurze Erläuterung der Rechnungen (z.B. verwendete Programme)
% Graphische Darstellung der Ergebnisse

Aus den gemessenen Werten wird nun die Verdampfungswärme von Wasser abgeschätzt.

Zur Fehlerfortpflanzung wird 
\begin{equation}
    \Delta y = \sum_{i=1}^n \left| \frac{\delta f(x_1,...,x_n)}{\delta x_i} \right| \Delta x_i
    \label{eq:fehlerrechnung}
\end{equation}
bzw. die Python Bibliothek Uncertainties verwendet.\cite{uncertainties}

%%%%%%%%%%%%%%%%%%%%%%%%%%%%%%%%%%%%%%%%%%%%%%%%%%%%
\subsection{Messung im Druckbereich unterhalb 1 bar}
\label{sec:Auswertung_niedrigdruck}

Die Durchführung wie in \autoref{sec:Durchführung_niedrigdruck} beschrieben ergab die in \autoref{fig:plot_niedrigdruck} dargestellten Messwerte.

Mithilfe der Messwerte wird ein Plot angefertig, in dem, auf Grundlage von (\ref{eq:druck}), $\ln(p)$ gegen $1/T$ aufgetragen wird.

Dann wird mit der Gleichung
\begin{equation}
    \ln(p) = \frac{a}{T} + b
\end{equation}
und der Funktion curve\_fit aus der Python Bibliothek SciPy eine Ausgleichsgerade für die Parameter $a$ und $b$ erstellt.\cite{scipy} Hierdurch ergeben sich die Parameter
\begin{align*}
    a =& \SI{-3859+-23}{\kelvin} \\
    b =& \SI{21.57+-0.07}{}.
    \label{eq:niedrigdruck_parameter}
\end{align*}

Nach (\ref{eq:druck}) wird nun die Verdampfungswärme $L$ über
\begin{equation}
    L = -R \cdot a
\end{equation}
berechnet und ergibt
\begin{equation*}
    L = \SI{32083+-192}{\joule\per\mol} \, \text{,}
\end{equation*}
wobei $R=\SI{8.314}{\joule\per{\mol\kelvin}}$ die allgemeine Gaskonstante ist.\cite{physics_constants}

\begin{figure}
    \centering
    \includegraphics[width=\textwidth]{build/plot_niedrigdruck.pdf}
    \caption{Plot des Messergebnisses im Niedrigdruckbereich mit dem Druck $p$ und der Temperatur $T$}
    \label{fig:plot_niedrigdruck}
\end{figure}

Um nun die innere Verdampfungswärme $L_\text{i}$, also die benötigte Arbeit um bei der Verdampfung die Anziehungskräfte der Teilchen zu überwinden, bei einer Temperatur $T=\SI{373}{\kelvin}$ abschätzen zu können wird
\begin{equation}
    L_\text{i} = L - L_\text{a}
\end{equation}
verwendet. 
Hier ist $L_\text{a}$ die äußere Verdampfungswärme, also die Arbeit, die benötigt wird um das Volumen im flüssigen Zustand auf das Volumen um gasförmigen Zustand auszudehnen.\cite{V203}

Diese Arbeit entspricht $pV$ und ergibt über die Allgemeine Gasgleichung
\begin{equation*}
    L_\text{a} = R \cdot T = \SI{3101.122}{\joule\per\mol}.
\end{equation*}

Hieraus ergibt sich die innere Verdampfungswärme
\begin{equation*}
    L_\text{i} = \SI{28982+-192}{\joule\per\mol}.
\end{equation*}
Um diese Größe besser darstellen zu können wird sie pro Molekül und in Elektronenvolt angegeben.
Diese Umrechnung geschieht mit der Avogadrokonstante $N_\text{a} = \SI{6.022e23}{\per\mol}$ und dem Elektronenvolt $\SI{1}{\electronvolt} = \SI{1.602e-19}{\joule}$ und ergibt
\begin{equation*}
    L_\text{i} = \SI{0.300+-0.002}{\electronvolt}. \: \text{\cite{physics_constants}}
\end{equation*}


%%%%%%%%%%%%%%%%%%%%%%%%%%%%%%%%%%%%%%%%%%%%%%%%%%%%
\subsection{Messung im Druckbereich oberhalb 1 bar}
\label{sec:Auswertung_hochdruck}

Die Messung im Hochdruckbereich, wie in \autoref{sec:Durchführung_hochdruck} beschrieben, ergab die in \autoref{fig:plot_hochdruck} dargestellten Messergebnisse.
Aus diesen Werten wird nun eine Abhängigkeit der Verdampfungswärme zur Temperatur bestimmt.

Dafür wird zunächst aus den Messwerten ein Plot angefertigt und mit
\begin{equation}
    p = a \cdot T^3 + b \cdot T^2 + c \cdot T + d
    \label{eq:hochdruck_fit}
\end{equation}
und der Funktion curve\_fit aus der Python Bibliothek ScyPi eine Ausgleichskurve erstellt.\cite{scipy}
Der Plot ist in \autoref{fig:plot_hochdruck} zu sehen.
Hier ergeben sich die Parameter
\begin{align*}
    a =& \SI{0.8+-0.1}{\pascal\per\kelvin\cubed} \\
    b =& \SI{-863+-130}{\pascal\per\kelvin\squared} \\
    c =& \SI{3.1+-0.6e5}{\pascal\per\kelvin} \\
    d =& \SI{-3.7+-0.8e7}{\pascal}.
\end{align*}

\begin{figure}
    \centering
    \includegraphics[width=\textwidth]{build/plot_hochdruck.pdf}
    \caption{Plot des Messergebnisses im Hochdruckbereich mit dem Druck $p$ und der Temperatur $T$}
    \label{fig:plot_hochdruck}
\end{figure}

Die Abhängigkeit der Verdampfungswärme zur Temperatur ergibt sich über (\ref{eq:clapeyron}) zu
\begin{equation}
    L = (V_\text{D}-V_\text{F}) \cdot T \cdot \frac{\dif{p}}{\dif{T}}.
    \label{eq:hochdruck_ergebnis}
\end{equation}
Hierbei ergibt sich $\dif{p}/\dif{T}$ aus (\ref{eq:hochdruck_fit}) zu
\begin{equation}
    \frac{\dif{p}}{\dif{T}} = 3 \cdot a \cdot T^2 + 2 \cdot b \cdot T + c.
    \label{eq:ableitung}
\end{equation}

Wie auch in der Theorie beschrieben kann $V_\text{F}$ vernachlässigt werden. 
Nun muss allerdings $V_\text{D}$ über eine Erweiterung der idealen Gasgleichung
\begin{equation}
    \left( p + \frac{A}{V^2} \right) \cdot V = R \cdot T
\end{equation}
mit $A = \SI{0.9}{\joule\meter\cubed\per\mol\squared}$ bestimmt werden und ergibt die zwei Lösungen
\begin{equation}
    V_{\text{D},\pm} = \frac{R \cdot T}{2 \cdot p} \pm \sqrt{\frac{R^2 \cdot T^2}{4 \cdot p^2} - \frac{A}{p}}
\end{equation}
wobei für $p$ (\ref{eq:hochdruck_fit}) verwendet wird.\cite{V203}

\autoref{fig:plot_hochdruck_plus} und \autoref{fig:plot_hochdruck_minus} stellen (\ref{eq:hochdruck_ergebnis}) eingesetzt mit allen benannten Werten und Gleichungen dar.

\begin{figure}
    \centering
    \includegraphics{build/plot_hochdruck_plus.pdf}
    \caption{Plot der Temperaturabhängigkeit im Falle von $V_{\text{D},+}$}
    \label{fig:plot_hochdruck_plus}
\end{figure}

\begin{figure}
    \centering
    \includegraphics{build/plot_hochdruck_minus.pdf}
    \caption{Plot der Temperaturabhängigkeit im Falle von $V_{\text{D},-}$}
    \label{fig:plot_hochdruck_minus}
\end{figure}