\section{Auswertung}
\label{sec:Auswertung}

% Messwerte: Alle gemessenen physikalischen Größen sind übersichtlich darzustellen.

% Auswertung:
% Berechnung der geforderten Endergebnisse
% mit allen Zwischenrechnungen und Fehlerformeln, sodass die Rechnung nachvollziehbar ist.
% Eine kurze Erläuterung der Rechnungen (z.B. verwendete Programme)
% Graphische Darstellung der Ergebnisse

\begin{table}
    \centering
    \begin{tabular}{S[table-format=2.0] S[table-format=1.1] S[table-format=2.1] S[table-format=2.1] S[table-format=2.1] S[table-format=3.0]}
        \toprule
    \tableSI{t}{\minute} & \tableSI{p_\text{a}}{\bar} & \tableSI{p_\text{a}}{\bar} & \tableSI{T_1}{\celsius} & \tableSI{T_1}{\celsius} & \tableSI{N_\text{el}}{\watt} \\
        \midrule
        0 & 4.2 & 3.9 & 22.0 & 21.9 & 0 \\
        1 & 3.5 & 5.5 & 22.7 & 21.7 & 115 \\
        2 & 3.7 & 6.0 & 24.1 & 20.3 & 125 \\
        3 & 3.8 & 6.5 & 25.7 & 18.8 & 123 \\
        4 & 3.7 & 7.0 & 27.2 & 17.8 & 125\\
        5 & 3.6 & 7.0 & 28.5 & 17.0 & 125\\
        6 & 3.5 & 7.5 & 30.0 & 16.1 & 125\\
        7 & 3.3 & 7.5 & 31.5 & 15.2 & 125\\
        8 & 3.2 & 8.0 & 32.8 & 14.3 & 125\\
        9 & 3.0 & 8.0 & 34.1 & 13.3 & 125\\
        10 & 2.9 & 8.5 & 35.3 & 12.5 & 125\\
        11 & 2.8 & 9.0 & 36.5 & 11.6 & 125\\
        12 & 2.7 & 9.0 & 37.7 & 10.7 & 127\\
        13 & 2.6 & 9.5 & 38.8 & 9.9 & 128\\
        14 & 2.6 & 10.0 & 39.8 & 9.0 & 129\\
        15 & 2.5 & 10.0 & 40.9 & 8.2 & 130\\
        16 & 2.4 & 10.5 & 41.7 & 7.5 & 130\\
        17 & 2.4 & 11.0 & 42.6 & 6.8 & 120\\
        18 & 2.3 & 11.0 & 43.5 & 6.2 & 119\\
        19 & 2.2 & 11.0 & 44.2 & 5.5 & 117\\
        20 & 2.2 & 11.5 & 45.0 & 4.8 & 116\\
        21 & 2.1 & 12.0 & 45.8 & 4.2 & 117\\
        22 & 2.1 & 12.0 & 46.5 & 3.6 & 117\\
        23 & 2.0 & 12.0 & 47.3 & 3.1 & 117\\
        24 & 2.0 & 12.5 & 48.0 & 2.6 & 117\\
        25 & 2.0 & 12.5 & 48.7 & 2.7 & 116\\
        26 & 2.0 & 12.5 & 49.2 & 1.7 & 116\\
        27 & 1.9 & 13.0 & 49.8 & 1.2 & 116\\
            \bottomrule
    \end{tabular}
    \caption{Darstellung aller aufgenommenen Werte während des Versuches}
    \label{tab:werte}
\end{table}

Die elektrische Arbeit des Kompressormotors $N_\text{el}$ wird über $\delta t$ mit der Formel der Mittelwertsberechnung

\begin{equation}
    \bar{x} = \frac{1}{n} \sum_{i=1}^n x_i
\end{equation}

gemittelt.
Über 
\begin{equation}
    \Delta\bar{x} = \sqrt{\frac{1}{n(n-1)}\sum_{i=1}^n (x_i - \bar{x})^2}
\end{equation}

wird die entsprechende Abweichung dieses Mittelwerts berechnet. 
Damit ist das gemittelte $N_\text{el}$ mitsamt Abweichung

\begin{equation}
   N_\text{el} = 117.68 \pm \SI{4.46}{\watt}.
\end{equation}

Damit die Güteziffer $\nu$ berechnet werden kann, wird die Formel

\begin{equation}
    \nu = (V_\text{W} \cdot \rho _\text{W} \cdot c_\text{W} + m_\text{k} c_\text{k}) \cdot \frac{\dif{T_1}}{\dif{t}} \cdot \frac{1}{N_\text{el}}
    \label{eq:gueteziffer}
\end{equation}

verwendet. Dabei ist $V_\text{W}$ das Volumen des Wassers, mit der Temperatur $T_1$ und $\rho _\text{W}$ die Dichte des Wassers bei etwa Raumtemperatur. 
$c_\text{W}$ ist die spezifische Wärmekapazität von Wasser bei etwa Raumtemperatur. 
Der Faktor $m_\text{k} \cdot c_\text{k}$ ist dabei eine vorgegeben Konstante und die Wärmekapazität der verwendeten Kupferschlange. Damit die berechnete Güterziffer am Ende mit einer Ungenauigkeit angegeben werden kann, wird zunächst die Ungenauigkeit von $T_1$ und dem Differenzialquotienten $\frac{\dif{T_1}}{\dif{t}}$ berechnet. 
Dazu wird die Gaußsche Fehlerformel

\begin{equation}
    \Delta f = \sqrt{\sum_{i=1}^N \left(\frac{\partial{f}}{\partial{y_\text{i}}} \cdot \Delta y_\text{i} \right)^2}
    \label{eq:gauß}
\end{equation}

verwendet. Mithilfe von SciPy und Curve Fit ist es möglich die Ergebnisse für $T_1$ und $T_2$ mit einer Ausgleichskurve annähern. Der Curve Fit wurde mit einer Funktion der Form

\begin{equation}
    T = a_1 \cdot t^2 + b_1 \cdot t + c_1
    \label{eq:curvefit}
\end{equation}

durchgeführt. Wodurch es drei Parameter zu bestimmen gilt, sowie ihre Unsicherheiten. Durch Curve Fit ergeben sich folgende Werte und der Graph in \autoref{fig:waermepumpe_plot}.

\begin{table}
    \centering
    \begin{tabular}{c c c}
        \toprule
        Parameter & Wert & Unsicherheit \\
        \midrule
        a & -5.5755 \cdot 10^{-6} & 0.1808 \cdot 10^{-6} \\
        b & 0.0266 & 0.0003 \\
        c & 294.4484 & 0.1061 \\
            \bottomrule
    \end{tabular}
    \caption{Parameter zu dem Curve Fit von $T_1$}
    \label{tab:fit1}
\end{table}

\begin{table}
    \centering
    \begin{tabular}{S[table-format=3.0] S[table-format=1.3] S[table-format=1.3]}
        \toprule
        \tableSI{t}{\second} & \tableSI{\frac{\dif{T_1}}{\dif{t}}}{\kelvin \per \second} & \tableSI{\frac{\dif{T_2}}{\dif{t}}}{\kelvin \per \second} \\
        \midrule
        300 & 0.023 & -0.017 \\
        600 & 0.020 & -0.015 \\
        900 & 0.017 & -0.012 \\
        1200 & 0.013 &  -0.010 \\
            \bottomrule
    \end{tabular}
    \caption{Differenzialquotienten zu vier verschiedenen Zeiten}
    \label{tab:diff_T1}
\end{table}

\begin{figure}
    \centering
    \includegraphics[width=\textwidth]{build/plot_waermepumpe.pdf}
    \caption{Graph der gemessenen Temperaturen $T_1$ und $T_2$ aus \autoref{fig:werte}.}
    \label{fig:waermepumpe_plot}
\end{figure}