\section{Auswertung}
\label{sec:Auswertung}

% Messwerte: Alle gemessenen physikalischen Größen sind übersichtlich darzustellen.

% Auswertung:
% Berechnung der geforderten Endergebnisse
% mit allen Zwischenrechnungen und Fehlerformeln, sodass die Rechnung nachvollziehbar ist.
% Eine kurze Erläuterung der Rechnungen (z.B. verwendete Programme)
% Graphische Darstellung der Ergebnisse

\subsection{Auswertungsteil a) Darstellung der Messwerte in einem Diagramm}
\label{ssec:a}
Zunächst werden in \autoref{tab:werte} alle gemessenen Werte aufgetragen. 
Damit mit folgenden Werten gerechnet werden kann, werden sie in allen folgenden Schritten in die SI-Basiseinheiten umgerechnet. 
Auf die beiden Drücke $p_\text{a}$ und $p_\text{b}$ werden noch jeweils $\SI{1}{\bar}$ addiert, für den Druck, der während des Versuches im Labor vorlag.

\begin{table}
    \centering
    \begin{tabular}{S[table-format=2.0] S[table-format=1.1] S[table-format=2.1] S[table-format=2.1] S[table-format=2.1] S[table-format=3.0]}
        \toprule
    \tableSI{t}{\minute} & \tableSI{p_\text{a}}{\bar} & \tableSI{p_\text{b}}{\bar} & \tableSI{T_1}{\celsius} & \tableSI{T_2}{\celsius} & \tableSI{N_\text{el}}{\watt} \\
        \midrule
        0 & 4.2 & 3.9 & 22.0 & 21.9 & 0 \\
        1 & 3.5 & 5.5 & 22.7 & 21.7 & 115 \\
        2 & 3.7 & 6.0 & 24.1 & 20.3 & 125 \\
        3 & 3.8 & 6.5 & 25.7 & 18.8 & 123 \\
        4 & 3.7 & 7.0 & 27.2 & 17.8 & 125\\
        5 & 3.6 & 7.0 & 28.5 & 17.0 & 125\\
        6 & 3.5 & 7.5 & 30.0 & 16.1 & 125\\
        7 & 3.3 & 7.5 & 31.5 & 15.2 & 125\\
        8 & 3.2 & 8.0 & 32.8 & 14.3 & 125\\
        9 & 3.0 & 8.0 & 34.1 & 13.3 & 125\\
        10 & 2.9 & 8.5 & 35.3 & 12.5 & 125\\
        11 & 2.8 & 9.0 & 36.5 & 11.6 & 125\\
        12 & 2.7 & 9.0 & 37.7 & 10.7 & 127\\
        13 & 2.6 & 9.5 & 38.8 & 9.9 & 128\\
        14 & 2.6 & 10.0 & 39.8 & 9.0 & 129\\
        15 & 2.5 & 10.0 & 40.9 & 8.2 & 130\\
        16 & 2.4 & 10.5 & 41.7 & 7.5 & 130\\
        17 & 2.4 & 11.0 & 42.6 & 6.8 & 120\\
        18 & 2.3 & 11.0 & 43.5 & 6.2 & 119\\
        19 & 2.2 & 11.0 & 44.2 & 5.5 & 117\\
        20 & 2.2 & 11.5 & 45.0 & 4.8 & 116\\
        21 & 2.1 & 12.0 & 45.8 & 4.2 & 117\\
        22 & 2.1 & 12.0 & 46.5 & 3.6 & 117\\
        23 & 2.0 & 12.0 & 47.3 & 3.1 & 117\\
        24 & 2.0 & 12.5 & 48.0 & 2.6 & 117\\
        25 & 2.0 & 12.5 & 48.7 & 2.7 & 116\\
        26 & 2.0 & 12.5 & 49.2 & 1.7 & 116\\
        27 & 1.9 & 13.0 & 49.8 & 1.2 & 116\\
            \bottomrule
    \end{tabular}
    \caption{Darstellung aller aufgenommenen Werte während des Versuches}
    \label{tab:werte}
\end{table}

Die Temperaturverläufe $T_1$ und $T_2$ aus \autoref{tab:werte} werden nun in einem Temperaturdiagramm gegen die Zeit $t$ aufgetragen.

\begin{figure}
    \centering
    \includegraphics[width=\textwidth]{build/plot_waermepumpe.pdf}
    \caption{Graph der gemessenen Temperaturen $T_1$ und $T_2$ aus \autoref{tab:werte}.\cite{numpy}}
    \label{fig:waermepumpe_plot}
\end{figure}

\subsection{Auswertungsteil b) Bestimmung der Ausgleichskurven}
\label{ssec:b} 
Mithilfe von SciPy und Curve Fit ist es möglich die Ergebnisse für $T_1$ und $T_2$ mit einer Ausgleichskurve anzunähern. 
Durch die Existenz dieser Funktionen werden im weiteren Verlauf der Berechnungen die Differenzenquotienten $\frac{\Delta T}{\Delta t}$ als Differenzialquotienten $\frac{\dif{T}}{\dif{t}}$ ausgedrückt.
Der Curve Fit wurde mit einer Funktion der Form

\begin{equation}
    T = a \cdot t^2 + b \cdot t + c
    \label{eq:curvefit}
\end{equation}

durchgeführt. Wodurch es drei Parameter zu bestimmen gilt, sowie ihre Unsicherheiten, diese lassen sich mit der Fehlerformel von Gauß

\begin{equation}
    \Delta f = \sqrt{\sum_{i=1}^N \left(\frac{\partial{f}}{\partial{y_\text{i}}} \cdot \Delta y_\text{i} \right)^2}
    \label{eq:gauß}
\end{equation}

berechnen. In \autoref{tab:fit1} sind die, durch Curve Fit berechneten Parameter von $T_1$ und ihre Unsicherheiten angegeben. 

\begin{table}
    \centering
    \begin{tabular}{c S[table-format=1.4e-1] S[table-format=1.4e-1]}
        \toprule
        Parameter & {Wert} & {Unsicherheit} \\
        \midrule
        a & -5.5755e-6 & 0.1808e-6 \\
        b & 0.0266 & 0.0003 \\
        c & 294.4484 & 0.1061 \\
        \bottomrule
    \end{tabular}
    \caption{Parameter zu dem Curve Fit von $T_1$}
    \label{tab:fit1}
\end{table}

Äquivalent dazu, wird die Rechnung auch mit $T_2$ durchgeführt.
Die drei Paramenter werden mit Curve Fit bestimmt und ihre Unsicherheit über \autoref{eq:gauß} berechnet.

\begin{table}
    \centering
    \begin{tabular}{c S S}
        \toprule
        \text{Parameter} & \text{Wert} & \text{Unsicherheit} \\
        \midrule
        a & 3.5689e-6 & 0.1842e-6 \\
        b & -0.0188 & 0.0003 \\
        c & 295.4911 & 0.1081 \\
            \bottomrule
    \end{tabular}
    \caption{Parameter zu dem Curve Fit von $T_2$}
    \label{tab:fit2}
\end{table}

\subsection{Auswertungsteil c) Differenzialquotienten der Temperaturen}
\label{ssec:c}
Im Folgenden Abschnitt werden für vier verschiedene Zeiten $t$ $\frac{\dif{T_1}}{\dif{t}}$ sowie $\frac{\dif{T_2}}{\dif{t}}$ bestimmt. 
Dafür wird erneut die Funktion aus \autoref{eq:curvefit} verwendet und die zeitliche Ableitung 

\begin{equation}
    \frac{\dif{T}}{\dif{t}} = 2 \cdot a \cdot t + b
    \label{eq:diffT}
\end{equation}

gebildet. 
Die berechneten Parameter aus \autoref{tab:fit1} werden dann verwendet, um die Steigung mit Hilfe von \autoref{eq:diffT} an vier verschiedenen Stellen zu berechnen.
Mit Hilfe von \autoref{eq:gauß} wird auch vom Differenzialquotienten eine Fehlerreichnung durchgeführt werden, damit die Unsicherheit angegeben werden kann. Dann ergeben sich die Werte aus \autoref{tab:T1_err}.

\begin{table}
    \centering
    \begin{tabular}{S[table-format=3.0] S[table-format=2.4] S[table-format=1.4] S[table-format=1.4] S[table-format=1.4]}
        \toprule
        \tableSI{t}{\second} & \tableSI{T_1}{\celsius} & \tableSI{T_1}{\kelvin} & \tableSI{\frac{\dif{T_1}}{\dif{t}}}{\kelvin \per \second} & \tableSI{\Delta \frac{\dif{T_1}}{\dif{t}}}{\kelvin \per \second} \\
        \midrule
        300 & 28.5000 &  301.6500 & 0.0230 & 0.0003\\
        600 & 35.3000 & 308.4500 & 0.0200 & 0.0004\\
        900 & 40.9000 & 314.0500 & 0.0170 & 0.0004\\
        1200 & 45.0000 & 318.1500 & 0.0130 & 0.0005\\
            \bottomrule
    \end{tabular}
    \caption{Differenzialquotienten von $T_1$ zu vier verschiedenen Zeiten}
    \label{tab:T1_err}
\end{table}

Äquivalent dazu werden nun zu diesen vier Zeiten die Differenzialquotienten von $T_2$ berechnet und in einer Tabelle eingetragen.

\begin{table}
    \centering
    \begin{tabular}{S[table-format=3.0] S[table-format=2.4] S[table-format=1.4] S[table-format=1.4] S[table-format=1.4]}
        \toprule
        \tableSI{t}{\second} & \tableSI{T_2}{\celsius} & \tableSI{T_2}{\kelvin} & \tableSI{\frac{\dif{T_2}}{\dif{t}}}{\kelvin \per \second} & \tableSI{\Delta \frac{\dif{T_2}}{\dif{t}}}{\kelvin \per \second} \\
        \midrule
        300 & 17.0000 &  290.1500 & -0.0167 & 0.0003\\
        600 & 12.5000 & 285.6500 & -0.0145 & 0.0004\\
        900 & 8.2000 & 281.3500 & -0.0124 & 0.0004\\
        1200 & 4.8000 & 277.9500 & -0.0102 & 0.0005\\
            \bottomrule
    \end{tabular}
    \caption{Differenzialquotienten von $T_2$ zu vier verschiedenen Zeiten}
    \label{tab:T2_err}
\end{table}

\subsection{Auswertungsteil d) Bestimmung der Güteziffer}
\label{ssec:d}

Die Güteziffer $\nu$ ist über \autoref{eq:gueteziffer_2} definiert und ist ein Maß für die Effizienz der Wärmepumpe. Für ihre Berechnung werden zunächst alle nötigen Größen bestimmt. 
Die elektrische Arbeit des Kompressormotors $N_\text{el}$ wird über $\delta t$ mit der Formel der Mittelwertsberechnung

\begin{equation}
    \bar{x} = \frac{1}{n} \sum_{i=1}^n x_i
\end{equation}

gemittelt.
Über 

\begin{equation}
    \Delta\bar{x} = \sqrt{\frac{1}{n(n-1)}\sum_{i=1}^n (x_i - \bar{x})^2}
\end{equation}

wird die entsprechende Abweichung dieses Mittelwerts berechnet. 
Damit ist das gemittelte $N_\text{el}$ mitsamt Abweichung

\begin{equation}
   N_\text{el} = 117.68 \pm \SI{4.46}{\watt}.
\end{equation}

Für die Berechung werden Konstanten benötigt.
Der Ausdruck $m_\text{k} c_\text{k}$ ist eine gegebene Gerätekonstante und gibt die Wärmekapazität der verwendeten Kupferschlange an.
In diesem Fall beträgt sie
\begin{equation}
   m_\text{k} c_\text{k} = \SI{750}{\joule \per \kelvin}.
\end{equation}
Die Wärmekapazität des Wassers wird ebenso bestimmt, über die Dichte des Wassers $\rho _\text{w}$ und das Volumen des Wassers in den Gefäßen $V_\text{w}$ ergibt sich die Masse $m_\text{w}$.
Die spezifische Wärmekapazität von Wasser ist damit
\begin{equation}
   m_\text{w} c_\text{w} = \SI{12449}{\joule \per \kelvin}.
\end{equation}
Aus \autoref{ssec:c} wird der Differenzialquotient von $T_1$ entnommen. Zunächst wird die Fehlerformel \autoref{eq:gauß} aufgestellt. Es ergibt sich dann

\begin{equation}
    \Delta \nu = \sqrt{\left((V_\text{w} \rho _\text{w} c_\text{w} + m_\text{k} c_\text{k}) \frac{1}{N_\text{el}} \Delta \frac{\dif{T_1}}{\dif{t}}\right)^2 + \left(-(V_\text{w} \rho _\text{w} c_\text{w} + m_\text{k} c_\text{k}) \frac{\dif{T_1}}{\dif{t}} \frac{1}{{N_\text{el}}^2} \Delta N_\text{el}\right)^2}.
    \label{eq:guetefehler}
\end{equation}

Damit kann nun die Formel aus \autoref{eq:gueteziffer} für die Güteziffer verwendet werden. 
Dadurch ergeben sich die folgenden vier dimensionslosen Güterziffern. 
Zum Vergleich wird zusätzlich die ideale Güterziffer zu den jeweiligen Zeitpunkten mit \autoref{eq:gueteziffer} berechnet.

\begin{table}
    \centering
    \begin{tabular}{c c c c c}
        \toprule
        \tableSI{t}{\second} & $\nu$ & $\Delta\nu$ & $\nu _\text{ideal}$ & \text{Abweichung} \\
        \midrule
        300 & 2.58 & 0.10 & 26.23 & 90.16 \% \\   
        600 & 2.24 & 0.10 & 13.53 & 83.44 \% \\
        900 & 1.91 & 0.09 & 9.57 & 80.04 \% \\
        1200 & 1.46 & 0.08 & 7.91 & 81.54 \% \\
        \bottomrule
    \end{tabular}
    \caption{Güteziffer zu vier verschiedenen Zeiten.}
    \label{tab:guete}
\end{table}
%Das klappt iwie nicht

\subsection{Auswertungsteil e)}
\label{ssec:e}

Zur Berechnung des Massendurchsatzes, also wie viel Masse pro Zeitintervall transpotiert wird, zu berechnen werden die Differenzialquotienten von $T_2$ aus \autoref{tab:T2_err} verwendet.
Aus einer Dampfdruck-Kurve wird zunächst die sogenannte Verdampfungswärme $L$ mit einem Curve Fit bestimmt. 
Der Curve Fit wurde mithilfe der Funktion

\begin{equation}
    p = p_0 \cdot \exp{\left(\frac{-L}{R}\frac{1}{T}\right)}
    \label{eq:L}
\end{equation}

durchgeführt. Sie wird nach 

\begin{equation}
    \ln{\left(\frac{p_b}{p_0}\right)} = \frac{-L}{R} \frac{1}{T}
    \label{eq:L2}
\end{equation}

umgeformt und geplottet.

\begin{figure}
    \centering
    \includegraphics[width=\textwidth]{build/plot_dampfdruck.pdf}
    \caption{Dampfkurve für die Berechnung von $L$ \autoref{tab:werte}.\cite{numpy}}
    \label{fig:dampfdruck_plot}
\end{figure}

Für den Parameter $L$ ergibt sich dann
\begin{equation*}
    L = 23308.25 \pm \SI{990.74}{\joule \per \mol}
    \label{eq:L3}
\end{equation*}

Mit \autoref{eq:massen} wird dann der Massendurchsatz berechnet, die Konstanten sind aus \autoref{ssec:d} bekannt. Dabei ist $m_1 = m_2$. 
Zuletzt wird noch die Gaußsche Fehlerabweichung mit \autoref{eq:gauß} berechnet.

\begin{equation}
    \Delta\frac{\Delta m}{\Delta t} = \sqrt{\left((m_2 c_\text{w} + m_\text{k} c_\text{k}) \frac{1}{L} \Delta \frac{\Delta T_2}{\Delta t}\right)^2 + \left(-(m_2 c_\text{w} + m_\text{k} c_\text{k}) \frac{1}{L^2} \frac{\Delta T_2}{\Delta t} \Delta L\right)^2}
    \label{eq:m_err}
\end{equation}

Dadurch ergibt sich das Ergebnis für den Massendurchsatz, umgerechnet in SI-Basiseinheiten, mit der molaren Masse von Dichlordifluormethan $\SI{0.121}{\kilogram \per \mol}$ ergeben sich die folgenden Werte in \autoref{tab:guete}. \cite{dichte_CL2F2C}

\begin{table}
    \centering
    \begin{tabular}{c c c c}
        \toprule
        \tableSI{t}{\second} & \tableSI{\frac{\dif{T_2}}{\dif{t}}}{\kelvin \per \second} & \tableSI{\frac{\Delta m}{\Delta t}}{\kilogram \per \second} & \tableSI{\Delta\frac{\Delta m}{\Delta t}}{\kilogram \per \second}\\
        \midrule
        300 & -0.0167 & -0.00115 & 0.00005 \\
        600 & -0.0145 & -0.00099 & 0.00005 \\
        900 & -0.0124 & -0.00085 & 0.00005\\
        1200 & -0.0102 & -0.00070 & 0.00005 \\
        \bottomrule
    \end{tabular}
    \caption{Massendurchsatz bei vier verschiedenen Temperaturen.}
    \label{tab:masse}
\end{table}

\subsection{Auswertungsteil f)}
\label{f}

Die Berechnung der mechanischen Leistung des Kompressors wird über \autoref{eq:arbeit} durchgeführt, die dafür benötigten Größen sind entweder, aus vorherigen Aufgaben bekannt, oder angegeben. 
Es werden wieder für die vier bekannten Zeiten bzw Temperaturen die Werte entnommen, $p_\text{a}$ und $p_\text{b}$ werden also entsprechen aus \autoref{tab:werte} genommen.
$\rho _0$ sei dabei
\begin{equation}
    \rho _0 = \SI{5.51}{\kilogram \per \cubic\meter}.
    \label{eq:dichte}
\end{equation}
Aus $\rho _0$ kann das benötigte $\rho$ berechnet werden, sie können über die ideale Gasgleichung in Zusammenhang gebracht werden.
Dann ergibt sich

\begin{equation}
    \rho  = \frac{\rho _0 T_0 p_\text{a}}{T_2 p_0}.
    \label{eq:dichte2}
\end{equation}

Die Leistung wird bei $T_0 = \SI{0}{\celsius}$ berechnet, also beträgt der Außendruck $p_0 = \SI{1}{\bar}$.
$\kappa$, das Verhältnis der beiden Molwärmen wird als $\kappa = 1.14$ festgelegt. 
In diesem Fall, sind alle Parameter der Gleichung bekannt, sodass kein Curve Fit durchgeführt werden muss.
Alle notwendigen Größen und Unsicherheiten werden in die Gaußsche Fehlerformel \autoref{eq:gauß} eingesetzt, dadurch ergibt sich

\begin{equation}
    \Delta N _\text{mech} = \sqrt{\left(\frac{1}{\kappa - 1} \left( p_b \sqrt[\kappa]{\frac{p_\text{a}}{p_\text{b}}} - p_\text{a} \right) \frac{T_2 p_0}{\rho _0 T_0 p_\text{a}} \Delta \frac{\Delta m}{\Delta t}\right)^2 }.
    \label{eq:n_err}
\end{equation}

\begin{table}
    \centering
    \begin{tabular}{c c c c}
        \toprule
        \tableSI{t}{\second} & \tableSI{\frac{\Delta m}{\Delta t}}{\kilogram \per \second} & \tableSI{N _\text{mech}}{\watt} & \tableSI{\Delta N _\text{mech}}{\watt}\\
        \midrule
        300 & -0.0167 & 11.1 & 0.5\\
        600 & -0.0145 &  15.5 & 0.8 \\
        900 & -0.0124 & 17.1 & 1.0\\
        1200 & -0.0102 & 16.8 & 1.2 \\
        \bottomrule
    \end{tabular}
    \caption{Mechanische Leistung des Kompressors bei vier verschiedenen Temperaturen.}
    \label{tab:leistung}
\end{table}