\section{Diskussion}
\label{sec:Diskussion}

% Kurze Zusammenfassung der Ergebnisse
% -Vergleich mit Literaturwerten
% -Vergleich mit verschiedenen Messverfahren
% -bei Abweichungen mögliche Ursachen finden

Im Folgenden wird diskutiert ob die Ergebnisse in \autoref{sec:Auswertung} mit der Theorie vereinbar sind und ob der Lock-In Verstärker das gewünschte Verhalten zeigt.

Durch den Vergleich von \autoref{fig:1} und \autoref{fig:2} lässt sich hier nicht interpretieren.
Die Unterschiede in der Amplitude lassen sich allerdings auf eine verstellte Signalspannung zurückführen.
Außerdem zeigt \autoref{fig:2}, dass ein verrauschtes Signal mithilfe des Lock-In Verstärkers bereinigt werden und vermessen werden kann.

\autoref{fig:plot_phase} und \autoref{fig:plot_phase_noise} zeigen, dass abgesehen von einer leichten Phasenverschiebung des Kosinus und eines umgedrehten Vorzeichens in \autoref{fig:plot_phase_noise} die gemessenen Werte der Theorie wie in \autoref{eq:u_out} folgen.

Die Vermessung der Lichtintensität der LED zeigt erneut die Funktion des Lock-In Verstärkers.
Die Messergebnisse in \autoref{fig:plot_led} zeigen, dass ein Abfall der Lichtintensität gemessen werden konnte, obwohl das Eingangssignal aufgrund von Umgebungslicht verrauscht war.

Die Unstimmigkeiten der Messwerte für kleine Abstände lassen sich auf eine nicht gut ausgerichtete Photodiode zurückführen.
Da die Photodiode vermutlich nicht parallel zur LED ausgerichtet war, hat das Licht erst ab einem Abstand von $\SI{14}{\centi\meter}$ seine Maximale Intensität auf der Photodiode abgebildet. 
Bei größeren Abständen war anscheinend die Drehung der Diode nicht wichtig, da der Lichtstrahl nicht stark gebündelt war. 

Die Ausgleichsrechnung in \autoref{fig:plot_led} zeigt die erwartete $1/r^2$ Abhängigkeit der Intensität und somit scheint die Messung der Lichtintensität erfolgreich gewesen zu sein.