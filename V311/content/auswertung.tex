\section{Auswertung}
\label{sec:Auswertung}

% Messwerte: Alle gemessenen physikalischen Größen sind übersichtlich darzustellen.

% Auswertung:
% Berechnung der geforderten Endergebnisse
% mit allen Zwischenrechnungen und Fehlerformeln, sodass die Rechnung nachvollziehbar ist.
% Eine kurze Erläuterung der Rechnungen (z.B. verwendete Programme)
% Graphische Darstellung der Ergebnisse

\subsection{Bestimmung der Hysteresekurve}
\label{ssec:a}



Bevor die eigentliche Auswertung beginnt, wird das B-Feld im Verhältnis zu der $I_\text{B}$ Spannung, also die Spannung, die das B-Feld erzeugt, gemessen.
Da die Stärke des Feldes von dem Zustand vor Beginn der Messung abhängt, wird ein aufsteigendes, $B_1$ Feld und ein abfallendes, $B_2$ Feld betrachtet.

\begin{table}
    \centering
    \caption{Messergebnisse der Hysteresekurve}
    \label{tab:hysterese}
    \begin{tabular}{S[table-format=1.1] S[table-format=4.0] S[table-format=3.0] S[table-format=4.2] S[table-format=-2.2]}
        \toprule
        \tableSI{I_\text{B}}{\ampere} & \tableSI{B_1}{\milli\tesla} & \tableSI{B_2}{\milli\tesla} & \multicolumn{2}{c}{\tableSI{B}{\milli\tesla}} \\
        \midrule
        0.5 & 145 & 150 & 66.23 & +-30.45 \\
        1.0 & 287 & 290 & 312.20 & +-31.99\\
        1.5 & 426 & 431 & 435.19 & +-33.82\\
        2.0 & 567 & 580 & 558.17 & +-36.22\\
        2.5 & 706 & 719 & 681.16 & +-39.09\\
        3.0 & 836 & 855 & 804.14 & +-42.35\\
        3.5 & 966 & 985 & 927.13 & +-45.89\\
        4.0 & 1076 & 1092 & 1050.11 & +-49.67\\
        4.5 & 1154 & 1162 & 1173.10 & +-53.63\\
        5.0 & 1213 & 1213 & 1296.08 & +-57.73\\
        \bottomrule
    \end{tabular}
\end{table}

Da es nicht sinnvoll ist mit beiden Werten zu rechnen, wird die Hysteresekurve durch eine Ausgleichsgeradem der Form

\begin{equation}
    f = a \cdot x + b
    \label{eq:gerade}
\end{equation}

angenähert.
Dabei ergeben sich folgende Parameter für $a$ und $b$,

\begin{align}
    a =& \SI{245.97+-9.81}{} \\
    b =& \SI{66.23+-30.45}{}.
    \label{eq:params1}
\end{align}

Über diese Parameter werden im folgenden alle Werte von $I_\text{B}$ in $B$ übersetzt.
Durch die Unsicherheiten der Parameter ergibt sich auf eine Ungenauigkeit in $B$.
Diese lässt sich über

\begin{equation}
    \sigma _B = \sqrt{\left(I_B \sigma _a \right)^2 + \left(\sigma _b \right)^2}
    \label{eq:B_fehler}
\end{equation}

berechnen.
Der durch die Ausgleichsgeraden errechnete Wert, sowie die Unsicheheit, werden in \autoref{tab:hysterese} notiert.

\begin{figure}
    \centering
    \includegraphics[width=\textwidth]{build/plot_hysterese.pdf}
    \caption{Hysteresekurve von $B_1$ und $B_2$ mit Ausgleichsgeraden.\cite{numpy}}
    \label{fig:hysterese_plot}
\end{figure}

\subsection{Auswertung der Messergebnisse, bei Variation des B-Feldes}
\label{ssec:mess}

Zunächst werden die Messwerte von Kupfer, Silber und Zink in Tabellen festgehalten.
In \autoref{tab:werte_kupfer_B} werden die Messwerte von Kupfer eingetragen.
Dort wird neben $I_\text{B}$ auch das daraus resultierende B-Feld angegeben, sowie die erzeugte Hall-Spannung $U_\text{H}$.

\begin{table}
    \centering
    \caption{Messergebnisse der Variation des Magnetfeldes bei Kupfer}
    \label{tab:werte_kupfer_B}
    \begin{tabular}{S[table-format=1.1] S[table-format=4.2] S[table-format=-2.2] S[table-format=-1.3]}
        \toprule
        \tableSI{I_\text{B}}{\ampere} & \multicolumn{2}{c}{\tableSI{B}{\milli\tesla}} & \tableSI{U_\text{H}}{\milli\volt} \\
        \midrule
        0.0 & 66.23 & +-30.45 & -0.009\\
        0.5 & 189.22 & +-30.84 & -0.007\\
        1.0 & 312.20 & +-31.99 & -0.005\\
        1.5 & 435.19 & +-33.82 & -0.003\\
        2.0 & 558.17 & +-36.22 & -0.001\\
        2.5 & 681.16 & +-39.09 & 0.001\\
        3.0 & 804.14 & +-42.35 & 0.003\\
        3.5 & 927.13 & +-45.89 & 0.005\\
        4.0 & 1050.11 & +-49.67 & 0.006\\
        4.5 & 1173.10 & +-53.63 & 0.007\\
        5.0 & 1296.08 & +-57.73 & 0.008\\
        \bottomrule
    \end{tabular}
\end{table}

Die Werte von $B$ und $U_\text{H}$ werden in \autoref{fig:kupfer_Ib_plot} geplottet und es wird eine Ausgleichsgerade der Form \autoref{eq:gerade} genutzt um einen Curve Fit durchzuführen.
Die Parameter dieses Fits sind dann

\begin{align}
    a =& \SI{1.43(6)e-5}{\meter\squared\per\second} \\
    b =& \SI{-9.31(43)e-6}{\volt}.
    \label{eq:params_Ib}
\end{align}

\begin{figure}
    \centering
    \includegraphics[width=\textwidth]{build/plot_kupfer_Ib.pdf}
    \caption{Gemessene Werte von $U_\text{H}$ bei Variation von $I_\text{B}$ mit Ausgleichsgeraden.\cite{numpy}}
    \label{fig:kupfer_Ib_plot}
\end{figure}

Für die Messreihe in der Silber verwendet wurde werden ebenfalls die Messwerte notiert.

\begin{table}
    \centering
    \caption{Messergebnisse der Variation des Magnetfeldes bei Silber}
    \label{tab:werte_silber_B}
    \begin{tabular}{S[table-format=1.1] S[table-format=4.2] S[table-format=-2.2] S[table-format=-1.3]}
        \toprule
        \tableSI{I_\text{B}}{\ampere} & \multicolumn{2}{c}{\tableSI{B}{\milli\tesla}} & \tableSI{U_\text{H}}{\milli\volt} \\
        \midrule
        0.0 & 66.23 & +-30.45 & -0.178\\
        0.5 & 189.22 & +-30.84 & -0.174\\
        1.0 & 312.20 & +-31.99 & -0.171\\
        1.5 & 435.19 & +-33.82 & -0.167\\
        2.0 & 558.17 & +-36.22 & -0.164\\
        2.5 & 681.16 & +-39.09 & -0.161\\
        3.0 & 804.14 & +-42.35 & -0.157\\
        3.5 & 927.13 & +-45.89 & -0.154\\
        4.0 & 1050.11 & +-49.67 & -0.151\\
        4.5 & 1173.10 & +-53.63 & -0.150\\
        5.0 & 1296.08 & +-57.73 & -0.148\\
        \bottomrule
    \end{tabular}
\end{table}

Durch diese Werte wird ebenfalls eine Funktion der Form \autoref{eq:gerade} gelegt.
Es werden wieder zwei Parameter 

\begin{align}
    a =& \SI{2.51(9)e-5}{\meter\squared\per\second} \\
    b =& \SI{17.84(7)e-5}{\volt}
    \label{eq:params_Ib2}
\end{align}

definiert.
Der Plot ist dann in \autoref{fig:silber_Ib_plot} zu sehen.

\begin{figure}
    \centering
    \includegraphics[width=\textwidth]{build/plot_silber_Ib.pdf}
    \caption{Gemessene Werte von $U_\text{H}$ bei Variation von $I_\text{B}$ mit Ausgleichsgeraden.\cite{numpy}}
    \label{fig:silber_Ib_plot}
\end{figure}

Für Zink werden ebenfalls alle Werte in einer Tabelle dokumentiert.

\begin{table}
    \centering
    \caption{Messergebnisse der Variation des Magnetfeldes bei Zink}
    \label{tab:werte_zink_B}
    \begin{tabular}{S[table-format=1.1] S[table-format=4.2] S[table-format=-2.2] S[table-format=-1.3]}
        \toprule
        \tableSI{I_\text{B}}{\ampere} & \multicolumn{2}{c}{\tableSI{B}{\milli\tesla}} & \tableSI{U_\text{H}}{\milli\volt} \\
        \midrule
        0.0 & 66.23 & +-30.45 & -0.338\\
        0.5 & 189.22 & +-30.84 & -0.340\\
        1.0 & 312.20 & +-31.99 & -0.342\\
        1.5 & 435.19 & +-33.82 & -0.344\\
        2.0 & 558.17 & +-36.22 & -0.347\\
        2.5 & 681.16 & +-39.09 & -0.349\\
        3.0 & 804.14 & +-42.35 & -0.352\\
        3.5 & 927.13 & +-45.89 & -0.355\\
        4.0 & 1050.11 & +-49.67 & -0.357\\
        4.5 & 1173.10 & +-53.63 & -0.357\\
        5.0 & 1296.08 & +-57.73 & -0.359\\
        \bottomrule
    \end{tabular}
\end{table}

Hier ist ebenfalls ein linearer Zusammenhang erkennbar, daher wird eine Funktion \autoref{eq:gerade} durch die Messwerte gelegt.
Folgende Parameter ergeben sich aus dem Curve Fit,

\begin{align}
    a =& \SI{-1.81(7)e-5}{\meter\squared\per\second} \\
    b =& \SI{33.68(5)e-5}{\volt}.
    \label{eq:params_Ib3}
\end{align}

Der Plot wird dann in \autoref{fig:zink_Ib_plot} angezeigt.

\begin{figure}
    \centering
    \includegraphics[width=\textwidth]{build/plot_zink_Ib.pdf}
    \caption{Gemessene Werte von $U_\text{H}$ bei Variation von $I_\text{B}$ mit Ausgleichsgeraden.\cite{numpy}}
    \label{fig:zink_Ib_plot}
\end{figure}

\subsection{Leitfähigkeit von Metallen durch Variation des B-Feldes}
\label{ssec:b}

Aus \autoref{ssec:mess} werden die Parameter $a$, also die Steigungen der Ausgleichsgeraden, definiert als

\begin{equation}
    a = \frac{-I_\text{q}}{n e_0 d}.
    \label{eq:a1}
\end{equation}

$I_\text{q}$ ist jeweils der Querstrom bei dem gemessen wurde.
Im Fall von Kupfer und Silber beträgt $I_\text{q} = \SI{10}{\ampere}$ und bei Zink $I_\text{q} = \SI{8}{\ampere}$.
$n$ ist dabei die Anzahl an Ladungsträger pro Volumen und $e_0$ die Elektronenladung $e_0 = \SI{1.602e-19}{\coulomb}$. 
$d$ ist die Dicke der verwendeten Platte, im Falle der Kupferplatte begrägt $d_\text{K} = \SI{18}{\micro\meter}$, bei Silber $d_\text{S} = \SI{0.026}{\milli\meter}$ und bei Zink $d_\text{Z} = \SI{0.037}{\milli\meter}$.
Über den Zusammenhang aus \autoref{eq:a1} wird die Anzahl der Ladungsträger pro Volumen berechnet, dafür wird die Gleichung nach 

\begin{equation}
    n = \frac{-I_\text{q}}{a e_0 d}.
    \label{eq:n}
\end{equation}

umgestellt und eine Fehlerformel 

\begin{equation}
    \sigma _\text{n} = \sqrt{\frac {I_\text{q}^{2} d^{2} e^{2} \sigma_{a}^{2} }{a^{4}}}.
    \label{eq:n_fehler}
\end{equation}

aufgestellt.
Daraus ergibt sich dann drei verschiedene Werte für $n$,

\begin{align}
    n_\text{K} =& \SI{2.42(9)e29}{\per\cubic\meter} \\
    n_\text{S} =& \SI{9.58(34)e28}{\per\cubic\meter} \\
    n_\text{Z} =& \SI{-7.45(29)e28}{\per\cubic\meter}.
    \label{eq:n1}
\end{align}

Das negative Vorzeichen bei $n_\text{Z}$ wird im folgenden weggelassen, da es physikalisch nicht sinnvoll ist.
Zur Berechnung der mittleren Flugzeit $\bar{\tau}$ wird \autoref{eq:widerstand} nach

\begin{equation}
    \bar{\tau} = \frac{2m_0 L}{{e_0}^2 n R Q}
    \label{eq:tau}
\end{equation}

umgeformt.
$R = \SI{2.76}{\ohm}$ ist der Widerstand von Kupfer, $L = \SI{1.37}{\meter}$ die Länge des Kupferdrahtes und $Q$ der Querschnitt des verwendeten Drahtes, der zur Bestimmung von $R$ verwendet wurde.
$Q$ wird aus der gegebenen Dicke des Drahtes berechnet, $Q$ ist dann

\begin{equation}
    Q = \pi \left(\frac{d_2}{2}\right)^2.
    \label{eq:Q}
\end{equation}

Dabei ist $d_2 = \SI{0.1}{\milli\meter}$ der Durchmesser des Drahtes.
Der Widerstand von Silber ist $R = \SI{0.58}{\ohm}$ und $L = \SI{1.73}{\meter}$.
$Q$ ist bei Silber der selbe, wie bei Kupfer.
Für Zink kann diese Formel nicht genutzt werden, da keine Widerstandsmessung durchgeführt werden konnte.
Daher wird der spezifische Widerstand $\rho$ von Zink aus einer externen Quelle entnommen.
Dann kann über 

\begin{equation}
    \bar{\tau} = \frac{2m_0}{{e_0}^2 n \rho}
    \label{eq:tau2}
\end{equation}

$\bar{\tau}$ ebenso bestimmt werden.
Der spezifische Widerstand von Zink ist $\rho = \SI{0.06}{\micro\ohm\meter}$.\cite{zink}
Die Ruhemasse eines Elektrons $m_0$ und die Elektronenmasse $e_0$ sind Naturkonstanten.\cite{physics_constants}
Es wird ebenfalls zwei Fehlerformeln für $\bar{\tau}$ aufgestellt, die für Kupfer und Silber, 

\begin{equation}
    \sigma _{\bar{\tau}} = \sqrt{\frac{4 L^{2} \sigma_{n}^{2} m_{0}^{2}}{Q^{2} R^{2} e_{0}^{4} n^{4}}}
    \label{eq:tau_fehler}
\end{equation}

und die Fehlerfortpflanzungsformel für Zink ist durch

\begin{equation}
    \sigma _{\bar{\tau}} = \sqrt{\frac{4 \sigma_{n}^{2} m_{0}^{2}}{{\rho}^{2} e_{0}^{4} n^{4}}}
    \label{eq:tau_fehler2}
\end{equation}

gegeben. 
Dadurch ergibt sich folgende drei Werte für die mittlere freie Flugzeit

\begin{align}
    \bar{\tau}_\text{K} =& \SI{2.32(9)e-14}{\second} \\
    \bar{\tau}_\text{S} =& \SI{3.52(12)e-13}{\second} \\
    \bar{\tau}_\text{Z} =& \SI{1.59(6)e-14}{\second}.
    \label{eq:tau1}
\end{align}

Zur Berechnung der mittleren Driftgeschwindigkeit wird \autoref{eq:driftgeschwindigkeit} verwendet.
$n$ und $e_0$ sind bekannte Größen, die Stromdichte $j$ ist vorgeben und beträgt $j = \SI{1e6}{\ampere\per\meter\squared}$.
Die Fehlerfortpflanzungsformel wird verwendet um die Fehlerformel

\begin{equation}
    \sigma _{\bar{v_\text{d}}} = \sqrt{\frac{j \sigma _n}{n^2 e_0}}.
    \label{eq:vd_fehler}
\end{equation}

für $\bar{v_\text{d}}$ aufzustellen.
Dann ergibt sich ein Ergebnis für die mittlere Driftgeschwindigkeit 

\begin{align}
    \bar{v_\text{d}}_\text{K} =& \SI{2.58(10)e-5}{\meter\per\second} \\
    \bar{v_\text{d}}_\text{S} =& \SI{6.52(23)e-5}{\meter\per\second} \\
    \bar{v_\text{d}}_\text{Z} =& \SI{8.38(33)e-5}{\meter\per\second}.
    \label{eq:vd1}
\end{align}

Die Beweglichkeit $\mu$ wird über \autoref{eq:beweglichkeit} bestimmt.
Für die Beweglichkeit wird ebenfalls eine Fehlerformel aufgestellt, diese wird dann

\begin{equation}
    \sigma _{\mu} = \sqrt{\frac{e_{0}^{4} \sigma_{n}^{2} \tau^{2} {\bar{v_\text{d}}}^{2}}{4 j^{2} m_{0}^{2}} + \frac{e_{0}^{4} \sigma_{\tau}^{2} n^{2} {\bar{v_\text{d}}}^{2}}{4 j^{2} m_{0}^{2}} + \frac{e_{0}^{4} \sigma_{\bar{v_\text{d}}}^{2} n^{2} \tau^{2}}{4 j^{2} m_{0}^{2}}}.
    \label{eq:bewegl_fehler}
\end{equation}

Zuletzt werden alle benötigten Werte in beide Formeln eingesetzt.
Die Beweglichkeit $\mu$ beträgt somit

\begin{align}
    \mu _\text{K} =& \SI{0.00204(14)}{\meter\squared\per\volt\per\second}\\
    \mu _\text{S} =& \SI{0.0310(19)}{\meter\squared\per\volt\per\second}\\
    \mu _\text{Z} =& \SI{0.00140(9)}{\meter\squared\per\volt\per\second}
    \label{eq:mue1}
\end{align}

Mithilfe von \autoref{eq:totalgeschwindigkeit} und \autoref{eq:fermienergie} wird die Totalgeschwindigkeit berechnet.
Beide Formeln werden verbunden und zu einer endgültigen Formel

\begin{equation}
    \bar{v}_\text{total} = \sqrt{\frac{2 \frac{\hbar^2}{2 m_0}\sqrt[3]{\left(\frac{3n}{8\symup{\pi}}\right)^2}}{m_0}}
    \label{eq:totalgeschw.}
\end{equation}

zusammengefügt.
Die einzige neu auftauchende Größe ist die Naturkonstante $h = \SI{6.63e-34}{\joule\per\hertz}$, das Plancksche Wirkungsquantum.\cite{physics_constants}
Die Fehlerformel für $\bar{v}_\text{total}$ wird durch 

\begin{equation}
    \sigma _{\bar{v}_\text{total}} = \frac{1}{6}  \sqrt{\frac{\sigma_{n}^{2} \left(\frac{h \left(\frac{3 n}{8 \Pi}\right)^{\frac{1}{6}}}{m_{0}^{2}}\right)^{1.0}}{n^{2}}}
    \label{eq:totalgeschw._fehler}
\end{equation}

berechnet.
Damit wird die Totalgeschwindigkeit

\begin{align}
    \bar{v}_\text{total,K} =& \SI{2.233(28)e6}{\meter\per\second}\\
    \bar{v}_\text{total,S} =& \SI{1.640(19)e6}{\meter\per\second}\\
    \bar{v}_\text{total,Z} =& \SI{1.508(20)e6}{\meter\per\second}
    \label{eq:vtotal1}
\end{align}

bestimmt.
Wird $\bar{v}_\text{total}$ nun mit $\bar{\tau}$ multipliziert, ergibt sich die mittlere freie Weglänge $\bar{l}$.
Die Fehlerformel wird dann zu

\begin{equation}
    \sigma _{\bar{l}} = \sqrt{\sigma_{\bar{\tau}}^{2} {\bar{v}_\text{total}}^{2} + \sigma_{\bar{v}_\text{total}}^{2} {\bar{\tau}}^{2}}
    \label{eq:weglaenge_fehler}
\end{equation}

Die mittlere freie Weglänge ist dann

\begin{align}
    \bar{l}_\text{K} = \SI{5.18(21)e-8}{\meter}\\
    \bar{l}_\text{S} = \SI{5.77(21)e-7}{\meter}\\
    \bar{l}_\text{Z} = \SI{2.40(10)e-8}{\meter}
    \label{eq:weglaenge1}
\end{align}

\subsection{Auswertung der Messergebnisse, bei Variation des Querstroms}
\label{ssec:mess2}

Wie zuvor werden zunächst alle Messergebnisse in Tabellen und Plots aufgetragen.
Angefangen mit Kupfer, werden die eingestellten Werte von $I_\text{q}$ zusammen mit den entsprechenden Werten von $U_\text{H}$ eingetragen.

\begin{table}
    \centering
    \caption{Messergebnisse der Variation des Querstroms bei Kupfer}
    \label{tab:werte_kupfer_Iq}
    \begin{tabular}{S[table-format=1.1] S[table-format=-1.3]}
        \toprule
        \tableSI{I_\text{q}}{\ampere} & \tableSI{U_\text{H}}{\milli\volt} \\
        \midrule
        0.0 & -0.003 \\
        1.0 & -0.002\\
        2.0 & -0.001\\
        3.0 & 0.000\\
        4.0 & 0.002\\
        5.0 & 0.003\\
        6.0 & 0.005\\
        7.0 & 0.006\\
        8.0 & 0.008\\
        9.0 & 0.009\\
        10.0 & 0.010\\
        \bottomrule
    \end{tabular}
\end{table}

Durch diese Messwerte wird wieder eine Ausgleichsgerade der Form \autoref{eq:gerade} gelegt, dadurch gibt es zwei Parameter $a$ und $b$, die sich von Curve Fit berechnen lassen.
Bei Kupfer sind diese Parameter

\begin{align}
    a =& \SI{-1.45(4)e-6}{\volt\second\per\coulomb} \\
    b =& \SI{-3.50(21)e-6}{\volt}.
    \label{eq:params_Iq1}
\end{align}

Der Plot mit diesen Parametern ist in \autoref{fig:kupfer_Iq_plot} zu sehen.

\begin{figure}
    \centering
    \includegraphics[width=\textwidth]{build/plot_kupfer_Iq.pdf}
    \caption{Gemessene Werte von $U_\text{H}$ bei Variation von $I_\text{q}$ mit Ausgleichsgeraden.\cite{numpy}}
    \label{fig:kupfer_Iq_plot}
\end{figure}

Die Messergebnisse für Silber werden ebenso wie die von Kupfer in einer Tabelle eingetragen.

\begin{table}
    \centering
    \caption{Messergebnisse der Variation des Querstroms bei Silber}
    \label{tab:werte_silber_Iq}
    \begin{tabular}{S[table-format=1.1] S[table-format=-1.3]}
        \toprule
        \tableSI{I_\text{q}}{\ampere} & \tableSI{U_\text{H}}{\milli\volt} \\
        \midrule
        0.0 & -0.001 \\
        1.0 & -0.016\\
        2.0 & -0.030\\
        3.0 & -0.044\\
        4.0 & -0.059\\
        5.0 & -0.073\\
        6.0 & -0.088\\
        7.0 & -0.102\\
        8.0 & -0.117\\
        9.0 & -0.131\\
        10.0 & -0.147\\
        \bottomrule
    \end{tabular}
\end{table}

Eine Ausgleichsgerade der Form \autoref{eq:gerade} hätte bei diesen Messwerten die Parameter 

\begin{align}
    a =& \SI{-1.451(5)e-5}{\volt\second\per\coulomb} \\
    b =& \SI{-9.09(280)e-5}{\volt}.
    \label{eq:params_Iq2}
\end{align}

Der Plot mit der Ausgleichgeraden wird dann in \autoref{fig:silber_Iq_plot} dargestellt.

\begin{figure}
    \centering
    \includegraphics[width=\textwidth]{build/plot_silber_Iq.pdf}
    \caption{Gemessene Werte von $U_\text{H}$ bei Variation von $I_\text{q}$ mit Ausgleichsgeraden.\cite{numpy}}
    \label{fig:silber_Iq_plot}
\end{figure}

Zuletzt wird noch die Messreihe von Zink in einer Tabelle festgehalten.

\begin{table}
    \centering
    \caption{Messergebnisse der Variation des Querstroms bei Zink}
    \label{tab:werte_zink_Iq}
    \begin{tabular}{S[table-format=1.1] S[table-format=-1.3]}
        \toprule
        \tableSI{I_\text{q}}{\ampere} & \tableSI{U_\text{H}}{\milli\volt} \\
        \midrule
        0.0 & 0.000\\
        0.5 & -0.024\\
        1.0 & -0.046\\
        1.5 & -0.067\\
        2.0 & -0.089\\
        2.5 & -0.109\\
        3.0 & -0.131\\
        3.5 & -0.153\\
        4.0 & -0.175\\
        4.5 & -0.196\\
        5.0 & -0.219\\
        \bottomrule
    \end{tabular}
\end{table}

Es wird eine Ausgleichsgerade der Form \autoref{eq:gerade} durch die Werte gelegt, Curve Fit liefert folgende Werte für die Parameter $a$ und $b$

\begin{align}
    a =& \SI{-4.33(2)e-5}{\volt\second\per\coulomb} \\
    b =& \SI{-1.55(50)e-6}{\volt}.
    \label{eq:params_Iq3}
\end{align}

Diese Parameter liefern die Ausgleichsgerade in \autoref{fig:zink_Iq_plot}.

\begin{figure}
    \centering
    \includegraphics[width=\textwidth]{build/plot_zink_Iq.pdf}
    \caption{Gemessene Werte von $U_\text{H}$ bei Variation von $I_\text{q}$ mit Ausgleichsgeraden.\cite{numpy}}
    \label{fig:zink_Iq_plot}
\end{figure}

\subsection{Leitfähigkeit von Metallen durch Variation des Querstroms}
\label{ssec:d}

Der Parameter $a$, also die jeweils die Steigung der Ausgleichgeraden, wird im weiteren für alle Berechnungen benötigt, $a$ ist als 

\begin{equation}
    a = \frac{-B}{n e_0 d}
    \label{eq:a2}
\end{equation}

definiert.
Alle Größen, die nicht mehr expilizit genannt werden, sind identisch mit den Größen aus \autoref{ssec:b}, $e_0$ und die Dicke $d$ bleiben also gleich.
Die Dicke der Kupferplatte begrägt also $d_\text{K} = \SI{18}{\micro\meter}$, bei Silber $d_\text{S} = \SI{0.026}{\milli\meter}$ und bei Zink $d_\text{Z} = \SI{0.037}{\milli\meter}$.
Das Experiment wurde bei einem konstanten B-Feld der Stärke $B = \SI{1296.08}{\milli\tesla}$ durchgeführt.
\autoref{eq:a2} wird nach der Anzahl der Ladungsträger $n$ umgeformt.
Damit ergibt sich dann

\begin{equation}
    n = \frac{B}{a e_0 d}.
    \label{eq:n2}
\end{equation}

Es wird außerdem eine Fehlerformel 

\begin{equation}
    \sigma _\text{n} = \sqrt{\frac {B^{2} d^{2} e^{2} \sigma_{a}^{2} }{a^{4}}}
    \label{eq:n2_fehler}
\end{equation}

aufgestellt.
Damit wird der Wert

\begin{align}
    n_\text{K} =& \SI{3.27(9)e29}{\per\cubic\meter}\\
    n_\text{S} =& \SI{-2.145(7)e28}{\per\cubic\meter}\\
    n_\text{Z} =& \SI{-5.045(20)e27}{\per\cubic\meter}
    \label{eq:n2_wert}
\end{align}

berechnet.

Für die Berechnung der mittleren freien Flugzeit $\bar{\tau}$ wird erneut \autoref{eq:tau} verwendet, allerdings nur für Kupfer und Silber, als Fehlerformel wird \autoref{eq:tau_fehler} genutzt.
Für Zink wird wieder \autoref{eq:tau2} genutzt, wobei der spezifische Widerstand $\rho$ als Materialkonstante gleich bleibt, für Zink wird \autoref{eq:tau_fehler2} als Fehlerfortpflanzungsformel verwendet.
Über alle Formeln wird dann 

\begin{align}
    \bar{\tau}_\text{K} =& \SI{1.72(5)e-14}{\second} \\
    \bar{\tau}_\text{S} =& \SI{1.571(5)e-12}{\second} \\
    \bar{\tau}_\text{Z} =& \SI{2.345(9)e-13}{\second}
    \label{eq:tau2_werte}
\end{align}

berechnet.

An der Berechnung der Driftgeschwindigkeit $\bar{v_\text{d}}$ hat sich nichts geändert, es wird \autoref{eq:driftgeschwindigkeit} zur Berechnug des eigentlichen Wertes genommen und \autoref{eq:vd_fehler} für die Berechnung des Fehlers.

Somit ergibt sich die Driftgeschwindigkeit

\begin{align}
    \bar{v_\text{d}}_\text{K} =& \SI{1.91(5)e-5}{\meter\per\second} \\
    \bar{v_\text{d}}_\text{S} =& \SI{29.10(9)e-5}{\meter\per\second} \\
    \bar{v_\text{d}}_\text{Z} =& \SI{12.37(5)e-4}{\meter\per\second}.
    \label{eq:vd2}
\end{align}

Die Beweglichkeit wird ebenfalls erneut über \autoref{eq:beweglichkeit} berechnet.
Die Fehlerfortpflanzungsformel ergibt dann die Fehlerformel aus \autoref{eq:bewegl_fehler}.
Es werden alle Werte in die Formel eingesetzt und es ergibt sich 

\begin{align}
    \mu _\text{K} =& \SI{0.00151(7)}{\meter\squared\per\volt\per\second}\\
    \mu _\text{S} =& \SI{0.1381(8)}{\meter\squared\per\volt\per\second}\\
    \mu _\text{Z} =& \SI{0.02061(14)}{\meter\squared\per\volt\per\second}
    \label{eq:mue2}
\end{align}

Über \autoref{eq:totalgeschw.} kann die Totalgeschwindigkeit $\bar{v}_\text{total}$ berechnet werden.
Die Fehlerformel \autoref{eq:totalgeschw._fehler} ist ebenso bereits bekannt, somit kann 

\begin{align}
    \bar{v}_\text{total,K} =& \SI{2.469(23)e6}{\meter\per\second}\\
    \bar{v}_\text{total,S} =& \SI{9.956(11)e5}{\meter\per\second}\\
    \bar{v}_\text{total,Z} =& \SI{6.146(8)e5}{\meter\per\second}
    \label{eq:vtotal2}
\end{align}

sofort bestimmt werden.

Die mittlere freie Weglänge $\bar{l}$ wird über \autoref{eq:weglaenge} bestimmt, die Ungenauigkeit wird dann über \autoref{eq:weglaenge_fehler} berechnet.
So ergibt sich

\begin{align}
    \bar{l}_\text{K} = \SI{4.25(13)e-8}{\meter}\\
    \bar{l}_\text{S} = \SI{1.564(5)e-6}{\meter}\\
    \bar{l}_\text{Z} = \SI{1.441(6)e-7}{\meter}
    \label{eq:weglaenge2}
\end{align}

als Endergebnis.