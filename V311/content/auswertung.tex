\section{Auswertung}
\label{sec:Auswertung}

% Messwerte: Alle gemessenen physikalischen Größen sind übersichtlich darzustellen.

% Auswertung:
% Berechnung der geforderten Endergebnisse
% mit allen Zwischenrechnungen und Fehlerformeln, sodass die Rechnung nachvollziehbar ist.
% Eine kurze Erläuterung der Rechnungen (z.B. verwendete Programme)
% Graphische Darstellung der Ergebnisse

\subsection{Bestimmung der Hysteresekurve}
\label{ssec:a}

Bevor die eigentliche Auswertung beginnt, wird das B-Feld im Verhältnis zu der $I_\text{B}$ Spannung, also die Spannung, die das B-Feld erzeugt, gemessen.
Da die Stärke des Feldes von dem Zustand vor Beginn der Messung abhängt, wird ein aufsteigendes, $B_1$ Feld und ein abfallendes, $B_2$ Feld betrachtet.

\begin{table}
    \centering
    \caption{Messergebnisse der Hysteresekurve}
    \label{tab:hysterese}
    \begin{tabular}{S[table-format=1.1] S[table-format=4.0] S[table-format=3.0] S[table-format=4.2+-2.2]}
        \toprule
        \tableSI{I_\text{B}}{\ampere} & \tableSI{B_1}{\milli\tesla} & \tableSI{B_2}{\milli\tesla} & \tableSI{B}{\milli\tesla} \\
        \midrule
        0.5 & 145 & 150 & 66.23+-30.45 \\
        1.0 & 287 & 290 & 312.20+-31.99\\
        1.5 & 426 & 431 & 435.19+-33.82\\
        2.0 & 567 & 580 & 558.17+-36.22\\
        2.5 & 706 & 719 & 681.16+-39.09\\
        3.0 & 836 & 855 & 804.14+-42.35\\
        3.5 & 966 & 985 & 927.13+-45.89\\
        4.0 & 1076 & 1092 & 1050.11+-49.67\\
        4.5 & 1154 & 1162 & 1173.10+-53.63\\
        5.0 & 1213 & 1213 & 1296.08+-57.73\\
        \bottomrule
    \end{tabular}
\end{table}

Da es nicht sinnvoll ist mit beiden Werten zu rechnen, wird die Hysteresekurve durch eine Ausgleichsgeradem der Form

\begin{equation}
    f = a \cdot x + b
    \label{eq:gerade}
\end{equation}

angenähert.
Dabei ergeben sich folgende Parameter für $a$ und $b$,

\begin{align*}
    a =& \SI{245.97+-9.81}{} \\
    b =& \SI{66.23+-30.45}{}.
    \label{eq:params1}
\end{align*}

Über diese Parameter werden im folgenden alle Werte von $I_\text{B}$ in $B$ übersetzt.
Durch die Unsicherheiten der Parameter ergibt sich auf eine Ungenauigkeit in $B$.
Diese lässt sich über

\begin{equation}
    \sigma _B = \sqrt{\left(I_B \sigma _a \right)^2 + \left(\sigma _b \right)^2}
    \label{eq:B_fehler}
\end{equation}

berechnen.
Der durch die Ausgleichsgeraden errechnete Wert, sowie die Unsicheheit, werden in \autoref{tab:hysterese} notiert.

\begin{figure}
    \centering
    \includegraphics[width=\textwidth]{build/plot_hysterese.pdf}
    \caption{Hysteresekurve von $B_1$ und $B_2$ mit Ausgleichsgeraden.\cite{numpy}}
    \label{fig:hysterese_plot}
\end{figure}

\subsection{Leitfähigkeit von Kupfer}
\label{ssec:b}

In \autoref{tab:werte_kupfer_B} wird neben $I_\text{B}$ auch das daraus resultierende B-Feld angegeben, sowie die erzeugte Hall-Spannung $U_\text{H}$.

\begin{table}
    \centering
    \caption{Messergebnisse der Variation des Magnetfeldes bei Kupfer}
    \label{tab:werte_kupfer_B}
    \begin{tabular}{S[table-format=1.1] S[table-format=4.2+-2.2] S[table-format=-1.3]}
        \toprule
        \tableSI{I_\text{B}}{\ampere} & \tableSI{B}{\milli\tesla} & \tableSI{U_\text{H}}{\milli\volt} \\
        \midrule
        0.0 & 66.23+-30.45 & -0.009\\
        0.5 & 189.22+-30.84 & -0.007\\
        1.0 & 312.20+-31.99 & -0.005\\
        1.5 & 435.19+-33.82 & -0.003\\
        2.0 & 558.17+-36.22 & -0.001\\
        2.5 & 681.16+-39.09 & 0.001\\
        3.0 & 804.14+-42.35 & 0.003\\
        3.5 & 927.13+-45.89 & 0.005\\
        4.0 & 1050.11+-49.67 & 0.006\\
        4.5 & 1173.10+-53.63 & 0.007\\
        5.0 & 1296.08+-57.73 & 0.008\\
        \bottomrule
    \end{tabular}
\end{table}

Die Werte von $B$ und $U_\text{H}$ werden in \autoref{fig:kupfer_Ib_plot} geplottet und es wird eine Ausgleichsgerade der Form \autoref{eq:gerade} genutzt um einen Curve Fit durchzuführen.
Die Parameter dieses Fits sind dann

\begin{align*}
    a =& \SI{0.0143+-0.0006}{\per\second\meter} \\
    b =& \SI{-0.0093+-0.0004}{\volt}.
    \label{eq:params_Ib}
\end{align*}

\begin{figure}
    \centering
    \includegraphics[width=\textwidth]{build/plot_kupfer_Ib.pdf}
    \caption{Gemessene Werte von $U_\text{H}$ bei Variation von $I_\text{B}$ mit Ausgleichsgeraden.\cite{numpy}}
    \label{fig:kupfer_Ib_plot}
\end{figure}

Dabei ist das $b$ nicht für die weiteren Rechnungen relevant, aber das $a$, also die Steigung der Ausgleichgeraden ist definiert als

\begin{equation}
    a = \frac{-I_\text{q}}{n e_0 d}.
    \label{eq:gleichung1}
\end{equation}

$n$ ist dabei die Anzahl an Elektronen pro Volumen und $e_0$ die Elektronenladung $e_0 = \SI{1.602e-19}{\coulomb}$. $d$ ist die Dicke der verwendeten Platte, im Falle der Kupferplatte begrägt $d = \SI{18}{\micro\meter}$
Das Experiment wird bei der maximal möglichen Querspannung $I_\text{q} = \SI{10}{\ampere}$ durchgeführt.
Über den Zusammenhang aus \autoref{eq:gleichung1} wird die Anzahl der Ladungsträger pro Volumen berechnet, dafür wird die Gleichung nach 

\begin{equation}
    n = \frac{-I_\text{q}}{a e_0 d}.
    \label{eq:gleichung2}
\end{equation}

umgestellt und eine Fehlerformel aufgestellt

\begin{equation}
    \sigma _\text{n} = \sqrt{\frac {I_\text{q}^{2} d^{2} e^{2} \sigma_{a}^{2} }{a^{4}}}.
    \label{eq:gleichung3}
\end{equation}

Daraus ergibt sich dann

\begin{equation}
    n = \SI{2.43(10)e26}{\cubic\meter}
    \label{eq:n1}
\end{equation}

Zur Berechnung der mittleren Flugzeit $\bar{\tau}$ wird THEORIEFORMEL FÜR R VERLINKEN nach

\begin{equation}
    \bar{\tau} = \frac{2m_0 L}{{e_0}^2 n R Q}
    \label{eq:tau}
\end{equation}

umgeformt.
$R = \SI{2.76}{\ohm}$ ist der Widerstand von Kupfer, $L = \SI{1.37}{\meter}$ die Länge des Drahtes und $Q$ der Querschnitt des verwendeten Drahtes, der zur Bestimmung von $R$ verwendet wurde.
$Q$ wird aus der gegebenen Dicke des Drahtes berechnet, $Q$ ist dann

\begin{equation}
    Q = \pi \left(\frac{d_2}{2}\right)^2.
    \label{eq:Q}
\end{equation}

Dabei ist $d_2 = \SI{0.1}{\milli\meter}$ der Durchmesser des Drahtes.
Die Ruhemasse eines Elektrons $m_0$ und die Elektronenmasse $e_0$ sind Naturkonstanten.\cite{physics_constants}
Es wird ebenfalls eine Fehlerformel für $\bar{\tau}$ aufgestellt, diese ist durch 

\begin{equation}
    \sigma _{\bar{\tau}} = \sqrt{\frac{4 L^{2} \sigma_{n}^{2} m_{0}^{2}}{Q^{2} R^{2} e_{0}^{4} n^{4}}}
    \label{eq:tau_fehler}
\end{equation}

gegeben. 
Nun werden alle nötigen Größen in \autoref{eq:tau} und \autoref{eq:tau_fehler} eingesetzt, dadurch ergibt sich

\begin{equation}
    \bar{\tau} = \SI{1.85(8)e-11}{\second}.
    \label{eq:Tau1}
\end{equation}