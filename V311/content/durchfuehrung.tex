\section{Durchführung}
\label{sec:Durchführung}

% Was wurde gemessen bzw. welche Größen wurden variiert?

Alle hier beschriebenen Messvorgänge werden für insgesamt drei Plättchen unterschiedlichen Materials durchgeführt. 
Hier werden Kupfer, Zink und Silber vermessen.

Um in der Auswertung geometrieunabhängige Größen bestimmen zu können, wird zuerst die Dicke der zu vermessenden Metallplättchen vermessen.

Weiterhin wird der Widerstand eines Drahtes gleichen Materials mit gegebener Länge und Durchmesser gemessen.
Dies geschieht mit einem Multimeter.

Das Magnetfeld wird mit einem Eisenjoch erzeugt, um welches an beiden Seiten eine Spule gewickelt ist.
Die beiden Spulen werden in Reihe an einen Gleichstromgenerator geschaltet.
An der Öffnung des Eisenjochs wird das Magnetfeld auf eine Stelle fokussiert, welche in etwa die Größe der zu vermessenden Metallplättchen hat.

Wichtig zu beachten ist, dass das Magnetfeld nie direkt abgeschaltet werden sollte, sondern immer nur langsam runtergedreht werden sollte, bevor der Generator abgeschaltet wird. 
Dies verhindert Fehler in der Aperatur, da das Magnetfeld Werte über $\SI{1}{\tesla}$ annimmt.

Um eine Hysteresekurve des Magnetfelds erstellen zu können, wird mithilfe einer Hallsonde das Magnetfeld für verschiedene Stromstärken gemessen.
Hierbei ist zu beachten, dass zuerst bei aufsteigender Stromstärke und dann bei gleichen Stromstärken absteigend gemessen wird.

Nun wird bei abgeschaltetem Magnetfeld eines der Metallplättchen an der Stelle fixiert wo das Magnetfeld homogen ist.
Die Kabel werden so angeschlossen, dass ein Querstrom erzeugt werden kann und die Hallspannung vermessen werden kann. 
Hierbei ist zu beachten, dass die Konstellation der Kabel bei allen Metallplättchen gleich ist, da das Vorzeichen der Hallspannung eine wichtige Rolle spielt.

Dann wird je eine Stromstärke auf dem Maximum gehalten und die Hallspannung wird für aufsteigende Werte der anderen Stromstärke gemessen.
Dies wird auch andersherum durchgeführt.