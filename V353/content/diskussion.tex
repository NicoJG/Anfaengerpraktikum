\section{Diskussion}
\label{sec:Diskussion}

% Kurze Zusammenfassung der Ergebnisse
% -Vergleich mit Literaturwerten
% -Vergleich mit verschiedenen Messverfahren
% -bei Abweichungen mögliche Ursachen finden

Aus den drei verschiedenen Bestimmungsmethoden für die Zeitkonstante ergeben sich folgende Werte:
\begin{align}
    RC =& \SI{0.6988}{\milli\second} \\
    RC =& \SI{0.8078}{\milli\second} \\
    RC =& \SI{0.8258}{\milli\second}
\end{align}
Die angegebenen Werte für den Widerstand $R=\SI{15.058}{\kilo\ohm}$ und den Kondensator $C=\SI{93.2}{\nano\farad}$ ergeben allerdings eine Zeitkonstante
\begin{equation}
    RC = \SI{1.4034}{\milli\second}.
\end{equation}
Die Abweichungen der einzelnen Ergebnisse müssen im Folgenden erklärt werden.

An den Ergebnissen ist zu sehen, dass die verwendeten Messmethoden auf ähnliche Ergebnisse kommen.
Allerdings sind die vorhandenen Abweichungen auf Ungenauigkeiten der Messmethoden zurückzuführen. So lässt sich beispielsweise am Oszilloskop die Spannung zu einem bestimmten Zeitpunkt nicht sehr genau ablesen. 

Nun ist allerdings auch zu sehen, dass die bestimmten Zeitkonstanten um etwa 50\% von dem erwarteten Wert abweichen. Dies lässt sich vermutlich auf einen systematischen Fehler zurückführen, da die berechneten Ergebnisse nur eine kleine Abweichung aufzeigen. Eine genaue Ursache lässt sich jedoch nicht finden.

Auch wurde gezeigt, dass eine Schaltung bestehend aus einem Kondensator und einem Widerstand als Integrator dienen kann. Bei angeschlossener Sinusspannung wurde auch gezeigt, dass die Spannung am Kondensator in Abhängigkeit des Phasenunterschieds zwischen Generator- und Kondensatorspannung den erwarteten Zusammenhang aus \autoref{eq:spannung_phase} abbildet.