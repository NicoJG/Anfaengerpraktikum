\section{Auswertung}
\label{sec:Auswertung}

%Messwerte: Alle gemessenen physikalischen Größen sind übersichtlich darzustellen.


%Auswertung:
%Berechnung der geforderten Endergebnisse
%mit allen Zwischenrechnungen und Fehlerformeln, sodass die Rechnung nachvollziehbar ist.
%Eine kurze Erläuterung der Rechnungen (z.B. verwendete Programme)
%Graphische Darstellung der Ergebnisse.

\subsection{Messergebnisse}

%Im Folgendem werden die während des Praktikums gemessenen Werte aufgelistet.

Die Messung der Resonanzfrequenz aus \autoref{sec:resonanz} ergibt $\SI{35.6}{\kilo\hertz}$.

Für die Messungen aus \autoref{sec:schwebung} werden hier jeweils die eingestellten Werte für die Kopplungskapazität $C_\text{k}$ aufgetragen, sowie die Frequenz bei der die Maxima gezählt wurden. Die gezählten Maxima und Extrema werden dahinter notiert. 

\begin{table}
  \centering
  \caption{Messwerte: Kopplungskapazität $C_\text{k}$ und Frequenz $f$ mit der entsprechenden Anzahl Maxima, sowie Extrema}
  \label{tab:schwebung}
  \begin{tabular}{c c c c}
    \toprule 
    $C_\text{k} \:/\: \si{\nano\farad}$ & $f \:/\: \si{\kilo\hertz}$ & Anzahl Maxima &  Anzahl Extrema   \\ 
    \midrule 
    0.997 & 0.626 & 2 & 3 \\
    2.290 & 0.626 & 3 & 5 \\
    2.860 & 0.626 & 5 & 9 \\
    4.740 & 0.626 & 6 & 12 \\
    6.860 & 0.626 & 9 & 17 \\
    8.180 & 0.626 & 11 & 23 \\
    9.990 & 0.626 & 13 & 25 \\
    12.000 & 0.626 & 18 & 35 \\
    \bottomrule
  \end{tabular}
\end{table}

In \autoref{sec:frequenzen} wurden die Frequenzen für die jeweiligen Fundamentalschwingungen untersucht und gemessen, diese sind nachfolgend mit ihrer entsprechenden Kopplungskapazität $C_\text{k}$ aufgeschrieben.

\begin{table}
  \centering
  \caption{Messwerte: Kopplungskapazität $C_\text{k}$ und die Frequenzen $\nu _+$ und $\nu _-$}
  \label{tab:frequenzen}
  \begin{tabular}{c c c}
    \toprule 
    $C_\text{k} \:/\: \si{\nano\farad}$ & $\nu _+ \:/\: \si{\kilo\hertz}$ & $\nu _- \:/\: \si{\kilo\hertz}$   \\ 
    \midrule 
    0.997 & 35.7 & 56.1 \\
    2.190 & 35.7 & 46.5 \\
    2.860 & 35.7 & 44.3 \\
    4.740 & 35.7 & 41.2 \\
    6.860 & 35.7 & 39.6 \\
    8.180 & 35.7 & 39.1 \\
    9.990 & 35.7 & 38.5 \\
    12.000 & 35.7 & 38.1 \\
    \bottomrule
  \end{tabular}
\end{table}

Für die in \autoref{sec:amplituden} gemessenen Werte sind hier in Abhängigkeit von der Kopplungskapazität $C_\text{k}$ die jeweilien maximalen Spannungsamplituden gelistet. Für jeden $C_\text{k}$ Wert wurde mit den beiden gemessenen Frequenzen aus \autoref{tab:frequenzen} die maximale Spannungsamplitude gemessen. Dabei entsprechen $U_{2+}$ und $U_{2-}$ der maximalen Spannungsamplitude im äußeren Schwingkreis bei den beiden Frequenzen $\nu _+$ und $\nu _-$. $U_k$ ist definiert als die maximale Spannungsamplitude innerhalb des gekoppelten Schwingkreises. Dort kann nur mit $\nu _-$ gemessen werden, da bei $\nu _+$, $C_\text{k}$ keinen Einfluss hat.

\begin{table}
  \centering
  \caption{Messwerte: Kopplungskapazität $C_\text{k}$ und Ampliduten von $U_{2+}$, $U_{2-}$ und $U_k$}
  \label{tab:amplituden}
  \begin{tabular}{c c c c}
    \toprule 
    $C_\text{k} \:/\: \si{\nano\farad}$ & $U_{2+} \:/\: \si{\volt}$ & $U_{2-} \:/\: \si{\volt}$ &  $U_k \:/\: \si{\volt}$       \\ 
    \midrule 
    2.190 & 2.50 & 2.00 & 2.50 \\
    2.860 & 2.50 & 2.25 & 2.75 \\
    4.740 & 2.50 & 2.25 & 2.75 \\
    6.860 & 2.50 & 2.25 & 2.75 \\
    8.180 & 2.50 & 2.38 & 2.75 \\
    9.990 & 2.50 & 2.38 & 2.88 \\
    12.000 & 2.50 & 2.50 & 2.88 \\
    \bottomrule
  \end{tabular}
\end{table}

Zur Referenz der Generatorspannung wurde die Amplidute der Spannung bei verschiedenen Frequenzen gemessen. Wobei jeweils bei den Fundamentalfrequenzen und einer von den Fundamentalfrequenzen gänzlich verschiedenen Frequenz gemessen wurde.

\begin{table}
  \centering
  \caption{Referenzwerte der Generatorspannung bei verschiedenen Frequenzen}
  \label{tab:resonanz}
  \begin{tabular}{c c c c}
    \toprule 
    $C_\text{k} \:/\: \si{\nano\farad}$ & $U \:$ bei$\:  \nu _+ \:/\: \si{\volt}$ & $U \:$ bei $\: \nu _- \:/\: \si{\volt}$ & $U \:$ bei $\: \SI{20.1}{\kilo\hertz} \:/\: \si{\volt} $ \\ 
    \midrule 
     8.180 & 8.00 & 7.50 & 10.00 \\
    \bottomrule
  \end{tabular}
\end{table}

\subsection{Vergleich der Messwerte zu den erwarteten Werten}
\label{sec:vergleich}

Die Resonanzfrequenz kann nun mit der \autoref{eq:frequenz+} für $\nu _+$ berechnet werden.

\begin{table}
  \centering
  \caption{Gegebene Werte der Bauelementschaltung}
  \label{tab:bauelement}
  \begin{tabular}{c c c c c}
    \toprule 
    \tableSI{R}{\ohm} & \tableSI{L}{\milli\henry} & \tableSI{C}{\nano\farad} & \tableSI{C_\text{sp}}{\nano\farad} & \tableSI{C_\text{gesamt}}{\nano\farad} \\ 
    \midrule 
    48 & 23.9540 & 0.7932 & 0.0280 & 0.8212 \\
    \bottomrule
  \end{tabular}
\end{table}

In \autoref{tab:bauelement} muss beachtet werden, dass die Induktivität $L$ ebenfalls eine Kapazität $C_\text{sp}$ besitzt.
Bevor die Resonanzfrequenz $\nu _+$ berechnet werden kann, muss also die Gesamtkapazität $C_\text{gesamt} = C + C_\text{sp}$ berechnet werden und wird von nun an als $C$ verwendet.

Damit ergibt sich dann der Theoriewert, der in der folgenden Tabelle mit dem gemessenen Wert verglichen wird.

\begin{table}
  \centering
  \caption{Vergleich der gemessenen Resonanzfrequenz und dem berechneten Theoriewert}
  \label{tab:resonanz}
  \begin{tabular}{c c}
    \toprule 
    $\nu _{+,\text{gemessen}} \:/\: \si{\kilo\hertz}$ & $\nu _{+,\text{berechnet}} \:/\: \si{\kilo\hertz}$    \\ 
    \midrule 
    35.600 & 35.884 \\
    \bottomrule
  \end{tabular}
\end{table}

Die entsprechenden Theoriewerte zu den gemessenen Maxima aus \autoref{tab:schwebung} können mit der \autoref{eq:extrema} berechnet werden. Die Werte für $\nu _+$ und $\nu _-$ sind die theoretisch berechneten Werte aus \autoref{tab:frequenzentheorie}, für die jeweiligen Kopplungskapazitäten $C_\text{k}$, die man in der Durchführung eingestellt hat. 

\begin{table}
  \centering
  \caption{Messwerte: Kopplungskapazität $C_\text{k}$ mit der gemessenen Anzahl Maxima und der theoretischen Anzahl Maxima }
  \label{tab:schwebungstheorie}
  \begin{tabular}{c c c}
    \toprule 
    \multirow{2}{*}{$C_\text{k} \:/\: \si{\nano\farad}$} & gemessene & berechnete \\
    & Anzahl Maxima & Anzahl Maxima \\ 
    \midrule 
    0.997 & 2 & 2.00 \\
    2.290 & 3 & 3.60 \\
    2.860 & 5 & 4.43 \\
    4.740 & 6 & 6.73 \\
    6.860 & 9 & 9.33 \\
    8.180 & 11 & 10.94 \\
    9.990 & 13 & 13.14 \\
    12.000 & 18 & 15.60 \\
    \bottomrule
  \end{tabular}
\end{table}

Die gemessenen Frequenzen zu den Fundamentalschwingungen aus \autoref{tab:frequenzen} werden ebenfalls mit den Theoriewerten verglichen. Mithilfe der \autoref{eq:frequenz-} für $\nu _-$ und den bekannten Werten für $L$ und $C$ aus \autoref{tab:bauelement} kann besagte Frequenz bestimmt werden. Für $\nu_-$ ist nun jedoch die Kopplungskapazität $C_\text{k}$ relevant, sodass sich für jedes $C_\text{k}$ ein anderer Theoriewert ergibt.

\begin{table}
  \centering
  \caption{Vergleich der gemessenen Fundamentalschwingungen mit den berechneten Theoriewerten}
  \label{tab:frequenzentheorie}
  \begin{tabular}{c c c}
    \toprule 
    $C_\text{k} \:/\: \si{\nano\farad}$ & $\nu _{-,\text{gemessen}} \:/\: \si{\kilo\hertz}$ & $\nu _{-,\text{berechnet}} \:/\: \si{\kilo\hertz}$    \\ 
    \midrule 
    0.997 & 56.100 & 58.386 \\
    2.190 & 46.500 & 47.470 \\
    2.860 & 44.300 & 45.024 \\
    4.740 & 41.200 & 41.640 \\
    6.860 & 39.600 & 39.950 \\
    8.180 & 39.100 & 39.322 \\
    9.990 & 38.500 & 38.722 \\
    12.000 & 38.100 & 38.261 \\
    \bottomrule
  \end{tabular}
\end{table}

\begin{figure}
    \centering
    \includegraphics[width=\textwidth]{build/plot_frequenzentheorie.pdf}
    \caption{Graph der Werte aus \autoref{tab:frequenzentheorie}}
    \label{fig:frequenzenthorie_plot}
\end{figure}

In diesem Versuch wurden lediglich die Werte für $U_{2+}$, $U_{2-}$ und $U_\text{k}$ gemessen, die Werte für $I_{2+}$, $I_{2-}$ und $I_k$ sind noch zu bestimmen, diese werden über die Formel $I = \frac{U}{R}$ berechnet.

$R$, $L$ und $C$ sind bekannte Werte aus \autoref{tab:bauelement} und $U$ wird jeweils aus \autoref{tab:amplituden} entnommen. In der Tabelle stehen jeweils die aus den gemessenen $U_{2+}$ und $U_{2-}$ berechneten Werte für $I_{2+}$ und $I_{2-}$. Die Theoriewerte für $I_{2+}$ und $I_{2-}$ werden mit \autoref{eq:stromamplitude} berechnet. Der kleinste Wert für $C_\text{k}$ wurde beim messen nicht beachtet. Daraus ergeben sich dann folgende Werte.

\begin{table}
  \centering
  \caption{Vergleich der gemessenen Maximalstromamplituden mit den berechneten Theoriewerten}
  \label{tab:amplitudentheorie}
  \begin{tabular}{c c c c c}
    \toprule 
    $C_\text{k} \:/\: \si{\nano\farad}$ & $I_{2+,\text{gemessen}} \:/\: \si{\ampere}$ & $I_{2+,\text{berechnet}} \:/\: \si{\ampere}$ & $I_{2-,\text{gemessen}} \:/\: \si{\ampere}$ & $I_{2-,\text{berechnet}} \:/\: \si{\ampere}$    \\ 
    \midrule 
    2.190 & 0.0521 & 0.0833 & 0.0417 & 0.0781 \\
    2.860 & 0.0521 & 0.0833 & 0.0469 & 0.0781 \\
    4.740 & 0.0521 & 0.0833 & 0.0469 & 0.0781 \\
    6.860 & 0.0521 & 0.0833 & 0.0469 & 0.0781 \\
    8.180 & 0.0521 & 0.0833 & 0.0496 & 0.0780 \\
    9.990 & 0.0521 & 0.0832 & 0.0496 & 0.0780 \\
    12.000 & 0.0521 & 0.0832 & 0.0521 & 0.0779 \\
    \bottomrule
  \end{tabular}  
\end{table}

\begin{figure}
    \centering
    \includegraphics[width=\textwidth]{build/plot_amplitudentheorie.pdf}
    \caption{Graph der Werte aus \autoref{tab:amplitudentheorie}}
    \label{fig:amplitudentheorie_plot}
\end{figure}


