\section{Diskussion}
\label{sec:Diskussion}

%Kurze Zusammenfassung der Ergebnisse
%-Vergleich mit Literaturwerten
%-Vergleich mit verschiedenen Messverfahren
%-bei Abweichungen mögliche Ursachen finden

Wie in \autoref{sec:vergleich} zu sehen ist sind einige der gemessenen Ergebnisse abweichend zu den berechneten Werten. Im Folgenden gilt es die möglichen Ursachen dieser Abweichungen zu beschreiben.

Vorab sei zu sagen, dass der verwendete Wechselspannungsgenerator nicht genau einstellbar war. Geschätzt war die Generatorfrequenz nur auf $\pm \SI{0.5}{\kilo\hertz}$ einstellbar. 
Da diese Frequenz bei jeder Schaltung wichtig war, kommt es generell zu Abweichungen.

Wenn dies beachtet wird scheint die Abweichung der Resonanzfrequenz in \autoref{tab:resonanz} plausibel.

Die Abweichungen der Anzahl der Maxima in einer Schwebungsperiode in \autoref{tab:schwebungstheorie} lässt sich einerseits durch die Abweichungen der Frequenzen in \autoref{tab:frequenzentheorie} erklären und andererseits war beim Abzählen der Maxima teilweise unklar welches Maximum noch mitgezählt werden muss, damit es innerhalb einer Schwebungsperiode liegt.

In \autoref{tab:frequenzentheorie} lassen sich die Abweichungen genau wie oben genannt erklären.

Die aus den gemessenen Spannungsamplituden berechneten Stromaplituden in \autoref{tab:amplitudentheorie} zeigen starke Abweichungen zu den berechneten Werten.
Eine mögliche Ursache dafür ist wie oben beschrieben die Ungenauigkeit der Generatorfrequenz.
Eine weitere Mögliche Ursache stellt die Abweichung der hierfür genutzten Frequenzen aus \autoref{tab:frequenzentheorie} dar, welche im Kilohertz Bereich liegen.
Außerdem wurde die Generatorspannungsamplitude nur für $C_k = \SI{8.180}{\nano\farad}$ gemessen und es wurde angenommen, dass diese für andere $C_k$ Werte gleich bleibt. 
Diese Annahme scheint nicht richtig zu sein und andere Generatorspannungen müssten gemessen und zur Berechnung genutzt werden. Diese Messung war allerdings nicht gefragt.