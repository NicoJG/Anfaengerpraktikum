\section{Zielsetzung}
\label{sec:Zielsetzung}

Ein elektrischer Schwingkreis ist eine Schaltung bestehend aus einem Widerstand $R$, einer Spule $L$ und einem Kondensator $C$.
In diesem Versuch wird die Interaktion zweier solcher Schwingkreise untersucht, welche über einen zusätzlichen Kondensator $C_k$ gekoppelt sind und die gleiche Resonanzfrequenz besitzen. 
Genauer gesagt möchte man:
\begin{enumerate}[label=\alph*)]
    \item Den Zeitlichen Verlauf des Energieaustausches im Schwebungsfall untersuchen.
    \item Die Frequenzen für beide Fundamentalschwingungen bestimmen.
    \item Die Amplitude des Stroms für beide Fundamentalschwingungen bestimmen.
\end{enumerate}