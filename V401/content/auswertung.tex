\section{Auswertung}
\label{sec:Auswertung}

% Messwerte: Alle gemessenen physikalischen Größen sind übersichtlich darzustellen.

% Auswertung:
% Berechnung der geforderten Endergebnisse
% mit allen Zwischenrechnungen und Fehlerformeln, sodass die Rechnung nachvollziehbar ist.
% Eine kurze Erläuterung der Rechnungen (z.B. verwendete Programme)
% Graphische Darstellung der Ergebnisse

Die gemessenen Intensitäten $z$ sind in \autoref{tab:welle} notiert. 
Sie wurden jeweils bei $\Delta d = \SI{5}{\milli\meter}$ bestimmt. 
\autoref{eq:welle} wird nach $\lambda$ umgestellt, so ergibt sich 
\begin{equation}
    \lambda = \frac{2 \cdot \Delta d}{z \cdot u}.
    \label{eq:lambda}
\end{equation}
Es wird ebenfalls der Faktor $1/u$ multipliziert.
Hier ist $u = 5.017$ die Hebelübersetzung. 
Dann werden über \autoref{eq:lambda} die Wellenlängen $\lambda$ berechnet und ebenfalls in \autoref{tab:welle} notiert.

\begin{table}
    \centering
    \caption{Intensitäten und Wellenlänge zu $\Delta d = \SI{5}{\milli\meter}$}
    \begin{tabular}{S[table-format=4.0] S[table-format=3.2]}
        \toprule
        $z$ & \tableSI{\lambda}{\nano\metre} \\
        \midrule
        3104 & 642.15\\
        3070& 649.26\\
        3057& 652.02\\
        2560& 778.60\\
        3001& 664.19\\
        2595& 768.10\\
        2909 & 685.19\\
        2524& 789.71\\
        3068& 649.68\\
        3054& 652.66\\
        \bottomrule
    \end{tabular}
    \label{tab:welle}
\end{table}

Der Mittelwert der berechneten Wellenlängen beträgt dann
\begin{equation*}
    \lambda = \SI{693.16+-19.12}{\nano\meter}.
\end{equation*}

Für die Berechnung des Brechungsindex $n$ von Luft wird \autoref{eq:index} verwendet. 
Dazu werden einige Konstanten benötigt, welche nachfolgend aufgelistet sind.
\begin{align*}
    T_0 &= \SI{273.15}{\kelvin}\\
    p_0 &= \SI{1.0132}{\bar}\\
    b &= \SI{50}{\milli\meter}\\
    T &= \SI{293.15}{\kelvin}
\end{align*}
Der Normaldruck $p_0$ entspricht dabei ebenfalls dem normalen Druck innerhalb der Messzelle $p$.

\begin{table}
    \centering
    \caption{Berechnete Brechungsindizes für Luft.}
    \begin{tabular}{S[table-format=1.2] S[table-format=2.0] S[table-format=1.6]}
        \toprule
        \tableSI{p}{\bar} & $z$ & $n$ \\
        \midrule
        0.22 & 31 & 1.000295\\
        0.20 & 32 & 1.000297\\
        0.30 & 28 & 1.000296\\
        0.24 & 31 & 1.000302\\
        0.24 & 30 & 1.000292\\
        \bottomrule
    \end{tabular}
    \label{tab:luft}
\end{table}

Der Mittelwert ergibt sich dann zu 
\begin{equation*}
    n = \num{1.000296 +- 0.000002}.
\end{equation*}
