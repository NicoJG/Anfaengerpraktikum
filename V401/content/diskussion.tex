\section{Diskussion}
\label{sec:Diskussion}

% Kurze Zusammenfassung der Ergebnisse
% -Vergleich mit Literaturwerten
% -Vergleich mit verschiedenen Messverfahren
% -bei Abweichungen mögliche Ursachen finden

Die Berechnung der Wellenlänge $\lambda$ kann nicht mit einem Theoriewert verglichen werden, die Wellenlänge des Lasers nicht angegeben wurde. Allerdings liegt $\lambda = \SI{693.16+-19.12}{\nano\meter}$ im Bereich von rotem Licht und der verwendete Laser hat rotes Licht ausgesendet. 
Daher scheint die Messung recht genau zu gewesen zu sein, obwohl einige Messwerte von einander abweichen.
Es hat sich schon während der Messung herausgestellt, dass die Messreihen die von $0$ bis $\SI{5}{\milli\meter}$ liefen allgemein mehr Impulse besaßen.

Der berechnete Brechungsindex für Luft beträgt $n = \num{1.000296 \pm 0.000002}$.
Verglichen mit einem Theoriewert von $n_\text{lit} = \num{1.000272}$ ergibt sich eine Abweichung von $\SI{0.002}{\percent}$. \cite{brechung}
Der Wert ist also ziemlich genau, was bei einer so kleinen Größenordnung zu erwarten war. 
