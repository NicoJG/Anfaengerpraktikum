\section{Diskussion}
\label{sec:Diskussion}

% Kurze Zusammenfassung der Ergebnisse
% -Vergleich mit Literaturwerten
% -Vergleich mit verschiedenen Messverfahren
% -bei Abweichungen mögliche Ursachen finden

Die Berechnung der Wellenlänge $\lambda$ kann nicht mit einem Theoriewert verglichen werden, da keiner vorliegt. Allerdings liegt $\lambda = 693.16 \pm \SI{19.12}{\nano\meter}$ im Bereich eines Lasers mit rotem Licht. 
Daher scheint die Messung recht genau zu gewesen zu sein, auch wenn einige Messwerte von einander abgewichen sind.
Es hat sich schon während der Messung herausgestellt, dass die Messungreihen die von $0$ bis $\SI{5}{\milli\meter}$ liefen allgemein mehr Impulse besaßen.

Der berechnete Brechungsindex für Luft beträgt $n = 1.000296 \pm 0.000002$.
Verglichen mit einem Theoriewert $n_\text{lit} = 1.000272$ ergibt sich eine Abweichung von $\SI{0.002}{\percent}$. \cite{brechung}
Der Wert ist also ziemlich genau, was bei einer so kleinen Größenordnung zu erwarten war. 
