\section{Auswertung}
\label{sec:Auswertung}

\subsection{Bestimmung der Brennweite mithilfe der Abhängigkeit der Bildweite zur Gegenstandsweite}
\label{ssec:Auswertung_brennweite}

Die gemessenen Werte und die nach \autoref{eq:linsengesetz} berechneten Brennweiten sind in \autoref{tab:brennweite50} für Linse 1 ($f_\text{Referenz}=\SI{50}{\milli\metre}$) und in \autoref{tab:brennweite100} für Linse 2 ($f_\text{Referenz}=\SI{100}{\milli\metre}$) aufgelistet.

\begin{table}
    \centering
    \caption{Messergebnisse für Linse 1 ($f_\text{Referenz}=\SI{50}{\milli\metre}$)}
    \begin{tabular}{S[table-format=2.0] S[table-format=2.1] S[table-format=1.2]}
        \toprule
        \tableSI{g}{\centi\metre} & \tableSI{b}{\centi\metre} & \tableSI{f}{\centi\metre} \\
        \midrule
        6 & 75.9 & 5.56 \\
        7 & 24.1 & 5.42 \\
        8 & 16.4 & 5.38 \\
        9 & 12.9 & 5.30 \\
        10 & 11.9 & 5.43 \\
        11 & 10.5 & 5.37 \\
        12 & 9.5 & 5.30 \\
        13 & 8.9 & 5.28 \\
        14 & 8.4 & 5.25 \\
        16 & 8.2 & 5.42 \\
        \bottomrule
    \end{tabular}
    \label{tab:brennweite50}
\end{table}

\begin{table}
    \centering
    \caption{Messergebnisse für Linse 2 ($f_\text{Referenz}=\SI{100}{\milli\metre}$)}
    \begin{tabular}{S[table-format=2.0] S[table-format=2.1] S[table-format=1.2]}
        \toprule
        \tableSI{g}{\centi\metre} & \tableSI{b}{\centi\metre} & \tableSI{f}{\centi\metre} \\
        \midrule
        12 & 66.0 & 10.15 \\
        13 & 44.6 & 10.07 \\
        14 & 35.5 & 10.04 \\
        15 & 30.1 & 10.01 \\
        16 & 26.6 & 9.99 \\
        17 & 24.3 & 10.00 \\
        18 & 21.9 & 9.88 \\
        19 & 21.0 & 9.98 \\
        20 & 19.4 & 9.85 \\
        21 & 19.1 & 10.00 \\
        22 & 18.1 & 9.93 \\
        \bottomrule
    \end{tabular}
    \label{tab:brennweite100}
\end{table}

Um nun einen Wert für die Brennweite der Linsen zu bestimmen. 
Wird der Mittelwert der $f$-Werte aus den Tabellen genommen und auch der Fehler des Mittelwerts berechnet.
Damit ergeben sich die Brennweiten
\begin{align*}
    f_{1,\text{berechnet}} &= \SI{5.37+-0.03}{\centi\metre} \\
    f_{2,\text{berechnet}} &= \SI{9.99+-0.03}{\centi\metre} \, .
\end{align*}

Eine weitere Methode die Brennweiten zu bestimmen ist in einem Plot die Gegenstandsweiten auf der x-Achse und die Bildweiten auf der y-Achse aufzutragen, die Punktepaare miteinander zu verbinden und den Schnittpunkt der Linien zu betrachten.
Diese Plots sind in \autoref{fig:plot_brennweite50} bis \ref{fig:plot_brennweite100_zoom} zu sehen.
Idealerweise sollten sich alle Schnittpunkte an einem Punkt sammeln, der die Brennweite der Linse als x- und x-Koordinate besitzt.
Die statistische Natur des Versuchs erzeugt allerdings eine Verteilung der Schnittpunkte.
Als Brennweite werden nun die Mittelwerte der x- und y-Koordinaten der Schnittpunkte berechnet.
Somit ergeben sich die Brennweiten grafisch zu
\begin{align*}
    f_{1,\text{grafisch},x} &= \SI{5.41+-0.09}{\centi\metre}
    f_{1,\text{grafisch},y} &= \SI{5.24+-0.07}{\centi\metre}
    f_{2,\text{grafisch},x} &= \SI{10.08+-0.12}{\centi\metre}
    f_{2,\text{grafisch},y} &= \SI{9.76+-0.12}{\centi\metre} \, .
\end{align*}

\begin{figure}
    \centering
    \includegraphics[width=0.9\textwidth]{build/plot_brennweite50.pdf}
    \caption{Plot der Verbindungslinien der Gegendstands- und Bildweiten für Linse 1 ($f_\text{Referenz}=\SI{50}{\milli\metre}$)}
    \label{fig:plot_brennweite50}
\end{figure}

\begin{figure}
    \centering
    \includegraphics[width=0.9\textwidth]{build/plot_brennweite50_zoom.pdf}
    \caption{Vergrößerter Plot der Verbindungslinien der Gegendstands- und Bildweiten für Linse 1 ($f_\text{Referenz}=\SI{50}{\milli\metre}$) um die Schnittpunkte besser zu erkennen}
    \label{fig:plot_brennweite50_zoom}
\end{figure}

\begin{figure}
    \centering
    \includegraphics[width=0.9\textwidth]{build/plot_brennweite100.pdf}
    \caption{Plot der Verbindungslinien der Gegendstands- und Bildweiten für Linse 1 ($f_\text{Referenz}=\SI{100}{\milli\metre}$)}
    \label{fig:plot_brennweite100}
\end{figure}

\begin{figure}
    \centering
    \includegraphics[width=0.9\textwidth]{build/plot_brennweite100_zoom.pdf}
    \caption{Vergrößerter Plot der Verbindungslinien der Gegendstands- und Bildweiten für Linse 1 ($f_\text{Referenz}=\SI{100}{\milli\metre}$) um die Schnittpunkte besser zu erkennen}
    \label{fig:plot_brennweite100_zoom}
\end{figure}



\subsection{Bestimmung der Brennweite nach der Methode von Bessel}
\label{ssec:Auswertung_bessel}

Die Messwerte ($e$, $g_1$, $g_2$) der Methode von Bessel sind in \autoref{tab:bessel} aufgelistet.
Zur Bestimmung der Brennweite wird nun der Abstand der beiden Linsenpositionen
\begin{equation}
    d = g_2 - g_1
    \label{eq:d}
\end{equation}
berechnet.
Hiermit lässt sich nach Bessel die Brennweite der Linse über
\begin{equation}
    f = \frac{e^2-d^2}{4e}
    \label{eq:bessel}
\end{equation}
berechnen.
Die so berechneten Werte für $d$ und $f$ sind ebenfalls in \autoref{tab:bessel} aufgelistet.

\begin{table}
    \centering
    \caption{tab:bessel}
    \sisetup{
        table-format=2.1
    }
    \begin{tabular}{S[table-format=2.0] S S S S[table-format=2.2]}
        \toprule
        \tableSI{e}{\centi\metre} & \tableSI{g_1}{\centi\metre} & \tableSI{g_2}{\centi\metre} & \tableSI{d}{\centi\metre} & \tableSI{f}{\centi\metre} \\
        \midrule
        40 & 18.0 & 22.5 & 4.5 & 9.87 \\
        42 & 16.2 & 26.1 & 9.9 & 9.92 \\
        44 & 15.4 & 29.1 & 13.7 & 9.93 \\
        45 & 15.0 & 30.4 & 15.4 & 9.93 \\
        46 & 14.7 & 31.5 & 16.8 & 9.97 \\
        48 & 14.4 & 34.2 & 19.8 & 9.96 \\
        50 & 13.9 & 36.5 & 22.6 & 9.95 \\
        52 & 13.7 & 38.7 & 25.0 & 10.00 \\
        54 & 13.5 & 41.0 & 27.5 & 10.00 \\
        56 & 13.4 & 43.1 & 29.7 & 10.06 \\
        58 & 13.1 & 45.3 & 32.2 & 10.03 \\
        60 & 12.9 & 47.4 & 34.5 & 10.04 \\
        \bottomrule
    \end{tabular}
\end{table}

Aus den berechneten Werten für $f$ lässt sich der Mittelwert mit entsprechendem Fehler zu
\begin{equation*}
    f_\text{Bessel} = \SI{9.97+-0.02}{\centi\metre}
\end{equation*}
bestimmen. 
Als Referenzwert dient hier eine Brennweite von $f_\text{Referenz} = \SI{100}{\milli\metre}$.