\section{Diskussion}
\label{sec:Diskussion}

In \autoref{tab:ergebnisse} sind alle in diesem Versuch berechneten Brennweiten aufgelistet. Außerdem wurde die prozentuale Abweichung des gemessenen Wert zum Referenzwert berechnet.

\begin{table}
    \centering
    \caption{Ergebnisse der Brennweitenbestimmung mit entsprechendem Referenzwert und Abweichung vom Referenzwert.}
    \begin{tabular}{l S[table-format=2.2] S[table-format=2.0] S[table-format=1.1]}
        \toprule
        Methode & \tableSI{f_\text{gemessen}}{\centi\metre} & \tableSI{f_\text{Referenz}}{\centi\metre} & \tableSI{\Delta f}{\percent} \\
        \midrule
        direkt (Linse 1) & 5.37 & 5 & 7.4 \\
        direkt (Linse 2) & 9.99 & 10 & 0.1 \\
        grafisch, $x$ (Linse 1) & 5.41 & 5 & 8.2 \\
        grafisch, $y$ (Linse 1) & 5.24 & 5 & 4.8 \\
        grafisch, $x$ (Linse 2) & 10.08 & 10  & 0.8 \\
        grafisch, $y$ (Linse 2) & 9.76 & 10 & 2.4 \\
        Bessel & 9.97 & 10 & 0.3 \\
        Abbe, $g'$ & 14.7 & {$16 \pm 4$} & {im Fehlerbereich} \\
        Abbe, $b'$ & 18.8 & {$16 \pm 4$} & {im Fehlerbereich} \\
        \bottomrule
    \end{tabular}
    \label{tab:ergebnisse}
\end{table}

Insgesamt lässt sich sagen, dass alle hier verwendeten Messmethoden für die Brennweite genügend geringe Abweichungen vom Referenzwert aufzeigen.

Es ist zu beobachten, dass alle Messungen der Linse mit Brennweite von $\SI{50}{\milli\metre}$ eine größere Abweichung aufzeigen als die Vermessung der $\SI{100}{\milli\metre}$ Linse.
Dies könnte daran liegen, dass die verwendete Linse nicht exakt dem Referenzwert entspricht.
Eine genaue Quelle dieser erhöhten Abweichung lässt sich im Nachhinein nicht feststellen.

Allerdings sei anzumerken, dass alle Messungen eine gewisse Willkür besitzen, weil beim Messen schwer abzuschätzen war wann ein Bild scharf ist. 
Bei manchen Messungen erschien das Bild in einem Bereich von mehreren Zentimetern scharf und der Abstand wann es am schärfsten war musste geschätzt werden.

Die Methode von Abbe scheint auf den ersten Blick die ungenauste zu sein, allerdings ist diese die einzige hier verwendete Methode, die ein System aus mehreren Linsen untersucht hat.
Und dass der Wert im Fehlerbereich des Referenzwertes liegt, zeigt, dass auch diese Methode ausreichend genau ist.