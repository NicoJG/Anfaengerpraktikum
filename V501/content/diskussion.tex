\section{Diskussion}
\label{sec:Diskussion}

Die berechneten Empfindlichkeiten der verwendeten Kathodenstrahlröhre sind
\begin{align*}
    E_\text{gemessen} &= \SI{54.2+-3.3}{\centi\metre} \\
    E_\text{Vergleich} &= \SI{35.75}{\centi\metre} \, .
\end{align*}
Hier ist eine große Abweichung zu sehen.
Diese Abweichung wird mehrere Gründe haben. 

Der gravierendste Grund ist vermutlich, dass die verwendete Apperatur nicht mehr ordnungsgemäß zu funktionieren scheint.
Bei keiner angelegten X- oder Y-Spannung sollte der Leuchtfleck Punktförmig sein. Allerdings war es bei unserer Messung nicht möglich einen Punkt zu erzeugen, sondern es wurde immer eine horizontal breitgezogene, nicht zu definierende, Form abgebildet.

Aber selbst bei einem vollkommen intakten Gerät hätten sich Abweichungen gebildet, da einige Näherungen getroffen wurden, die eventuell einen großen Einfluss haben können. 
Z.B. wurden die Kondensatorplatten als gerade Platten angenommen, aber in der Kathodenstrahlröhre befanden sich Platten wie in \autoref{fig:abmessungen} zu sehen.

Zwar sind wir für die folgenden Messungen an ein neues, besser funktionierendes Gerät gewechselt,
aber das Benutzen der Kathodenstrahlröhre zum Untersuchen einer angelegten Spannung war trotzdem nur bedingt sinnvoll.
Die abgebildete Sinuskurve war selbst auf maximal eingestellter Spannung der Spannungsgeneratoren nur sehr klein.
Trotzdem war es möglich sowohl die Frequenz der Sägezahnspannung zu messen, als auch annähernd stehende Bilder zu erzeugen.
Die Frequenz der angelegten Sinusspannung wurde zu
\begin{equation*}
    \nu_\text{Sin} = \SI{47+-3}{\hertz}
\end{equation*}
ermittelt. 
Dies ist verwunderlich, da auf dem verwendeten Sinusspannungsgenerator eine Frequenz von $80-90 \: \si{\hertz}$ angegeben war.
Allerdings wurde nicht überprüft, ob diese Frequenz der tatsächlich erzeugten Frequenz entspricht.

Auch die berechnete Amplitude der Sinusspannung
\begin{equation*}
    U_\text{Sin} = \SI{5.329+-0.002}{\volt}
\end{equation*}
ist verwunderlich, da am Generator eine Spannung von $\~\SI{40}{\volt}$ eingestellt wurde.
Allerdings ist auch hier fragwürdig, ob dies die tatsächlich erzeugte Spannung war, da sich die Abbildung nur minimal bei Änderung der Spannung verändert hat.

Abschließend sei zu sagen, dass die bestimmten Werte aufgrund schlecht funktionierender Geräte nur als ganz grobe Schätzwerte angenommen werden können.
Trotzdem wurde gezeigt, dass eine Kathodenstrahlröhre dafür benutzt werden kann den zeitlichen Verlauf einer periodischen Spannung visuell darzustellen und zu untersuchen.