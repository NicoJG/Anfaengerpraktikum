\section{Auswertung}
\label{sec:Auswertung}

% Messwerte: Alle gemessenen physikalischen Größen sind übersichtlich darzustellen.

% Auswertung:
% Berechnung der geforderten Endergebnisse
% mit allen Zwischenrechnungen und Fehlerformeln, sodass die Rechnung nachvollziehbar ist.
% Eine kurze Erläuterung der Rechnungen (z.B. verwendete Programme)
% Graphische Darstellung der Ergebnisse

\subsection{Überprüfung der Bragg-Bedingung}
\label{ssec:1}

Die Messdaten aus \label{tab:bragg} werden nun geplottet.

\begin{figure}
    \centering
    \includegraphics[width=\textwidth]{build/plot_spektrum.pdf}
    \caption{Plot der Messungen für die Bragg-Bedingung}
    \label{fig:bragg}
\end{figure}

Aus dem Plot und den Daten kann entnommen werden, dass das Maximum der Intensität bei 

\begin{equation}
    \theta _\text{max} = \SI{28.2}{\degree}
\end{equation}

liegt.

\subsection{Analyse eines Emessionsspektrums der Kuper-Röntgenröhre}
\label{ssec:2}

Die Messwerte befinden sich im Anhang in \autoref{tab:werte_emissionCu}.
Dann werden die Ergebnisse geplottet.
Das Ergebnis ist in \autoref{fig:kupfer} dargestellt.

\begin{figure}
    \centering
    \includegraphics[width=\textwidth]{build/plot_emissionCu.pdf}
    \caption{Plot der Emission der Cu-Röntgenröhre}
    \label{fig:kupfer}
\end{figure}

Aus dem Spektrum lassen sich die beiden Winkel der K-Alpha und K-Beta Linien bestimmen, diese befinden sich bei den Winkeln

\begin{align}
    \theta _\text{beta} = \ang{20.2}\\
    \theta _\text{alpha} = \ang{22.5}.
\end{align}

Für die Berechnung der Halbwertsbreiten wurde eine Nullebene durch den niedrigsten N-Wert gelegt.
Die Halbwertsbreite kann nur grob berechnet werden, da in den notwendigen Bereichen nur wenige Werte vorliegen.
Es werden Geraden zwischen die beiden nächsten Punkte gelegt und von dort aus, die viel benötigten Winkel.
Damit ergeben sich je zwei Winkel für beide Linien

\begin{align*}
    \theta _\text{beta,1} &= \ang{20.0559}\\
    \theta _\text{beta,2} &= \ang{20.5605}.
\end{align*}

\begin{align*}
    \theta _\text{alpha,1} &= \ang{22.3542}\\
    \theta _\text{alpha,2} &= \ang{22.8505}.
\end{align*}

\eqref{eq:lambda} wird mit der Energie eines Photons gleichgesetzt und dann nach der Energie $E$ umgeformt.
Dann ergibt sich der Zusammenhang

\begin{equation}
    E = \frac{h \cdot c \cdot n}{2d \cdot \sin{\theta}}.
    \label{eq:energie}
\end{equation}

Über diesen werden die vier Energien berechnet und jeweils ihren entsprechenden Linien nach substrahiert,

\begin{align*}
    H_ \beta = \SI{1.4380e-15}{\joule} - \SI{1.4042e-15}{\joule} &= \SI{3.38e-17}{\joule}\\
    H_ \alpha = \SI{1.2966e-15}{\joule} - \SI{1.2699e-15}{\joule} &= \SI{2.67e-17}{\joule}.
\end{align*}

Aus diesen wiederum  wird über 

\begin{equation}
    A = \frac{E_\text{K}}{\Delta E_ \text{F\,W\,H\,M}}
    \label{eq:aufloes}
\end{equation}

das Auflösevermögen berechnet werden.
Dabei ist $E_\text{K}$ die Energie der jeweiligen Kante, 

\begin{align*}
    E_ \beta  = \SI{1.4265e-15}{\joule} = \SI{8.903}{\kilo\electronvolt}\\
    E_ \alpha = \SI{1.2877e-15}{\joule} = \SI{8.037}{\kilo\electronvolt}.
\end{align*}

Dann werden die jeweiligen Werte in \eqref{eq:aufloes} eingesetzt, daraus ergeben sich dann die beiden Auflösevermögen

\begin{align*}
    A_ \beta  = 42.18\\
    A_ \alpha = 48.23.
\end{align*}

Die drei Abschirmkonstanten $\sigma_1$, $\sigma_2$ und $\sigma_3$ berechnen sich über die folgenden drei Formeln

\begin{align*}
    \sigma_1 &= Z - \sqrt{\frac{E_\text{K,abs}}{R_\infty}}\\
    \sigma_2 &= Z - \sqrt{\frac{m^2}{n^2} \cdot \left(Z - \sigma_1\right)^2 - \frac{E_ \alpha ^2 m^2}{R_\infty}}\\
    \sigma_3 &= Z - \sqrt{\frac{l^2}{n^2} \cdot \left(Z - \sigma_1\right)^2 - \frac{E_ \beta ^2 m^2}{R_\infty}}.
\end{align*}

$Z = 29$ ist dabei die Ordnungszahl von Kupfer.
Die anderen Konstanten sind gegeben durch

\begin{align*}
    n &= 1\\
    m &= 2\\
    l &= 3\\
    R_\infty &= \SI{13.6}{\electronvolt}\\
    E_\text{K,abs} &= \SI{8.978}{\kilo\electronvolt}.
\end{align*}

$R_\infty$ ist die Rydbergenergie und $E_\text{K,abs}$ ist die Absorptionsenergie von Kupfer. \cite{V602} \cite{absorption}
Durch alle obigen Informationen lassen sich dann alle drei Werte als 

\begin{align}
    \sigma_1 &= 3.305\\
    \sigma_2 &= 12.537\\
    \sigma_3 &= 21.937
\end{align}

berechnen.

\subsection{Analyse der Absorptionsspektren}
\label{ssec:3}

Die Messwerte der sechs Messungen wurden alle im Anhang hinterlegt.
Zunächst werden alle Messreihen geplottet, dann ergibt sich folgender Plot in \autoref{fig:absorption}.

\begin{figure}
    \centering
    \includegraphics[width=\textwidth]{build/plot_absorption.pdf}
    \caption{Plot der Messwerte aller Absorptionsspektren}
    \label{fig:absorption}
\end{figure}

Die Bestimmung der Mitte der K-Kante, also dem Wert $I_\text{K}$, gelingt über 

\begin{equation}
    I_\text{K} = I^\text{min}_\text{K} + \frac{I^\text{max}_\text{K} - I^\text{min}_\text{K}}{2}.
    \label{eq:IK}
\end{equation}

Diese Rechnung wird jeweils für alle sechs Elemente durchgeführt, die Intensitätmaxima $I^{\text{max}}_\text{K}$ und -minima $I^{\text{min}}_\text{K}$ der  wurden in allen Fällen numerisch bestimmt.
Über \eqref{eq:IK} werden dann die benötigten Werte für $I_\text{K}$ berechnet.
$\theta$ kann dann im Plot oder aus den Tabellen dirkt zugeorndet werden.
Aus $\theta$ wiederum wird über \eqref{eq:energie} die Absorptionsenergie $E_\text{K,abs}$ bestimmt.


\begin{table}
    \centering
    \caption{Berechnete Werte für $I_\text{K}$ mit zugehörigen $\theta$ und $E_\text{K,abs}$.}
    \label{tab:IK}
    \begin{tabular}{c S[table-format=3.1] S[table-format=2.2] S[table-format=2.2]}
        \toprule
        \text{Element}  & \tableSI{I_\text{K}}{\frac{\text{Imp}}{\si{\second}}} & \tableSI{\theta}{\degree} & \tableSI{E_\text{K,abs}}{\kilo\electronvolt}\\
        \midrule
        Brom & 18.0 & 13.20 & 13.48\\
        Gallium & 94.0 & 17.35 & 10.32\\
        Rubidium & 37.0 & 11.80 & 15.05\\
        Strontium & 118.0 & 11.10 & 15.99\\
        Zink & 78.0 & 18.70 & 9.60\\
        Zirkonium & 206.5 & 9.95 & 17.81\\
        \bottomrule
    \end{tabular}
\end{table}

Dann wird aus all diesen bestimmten und gemessenen Werte über \eqref{eq:sigma} die Abschirmkonstante berechnet.
$Z$ ist die jeweilige Ordnungszahl des Elements, diese in der nachfolgend aufgelistet,

\begin{align*}
    Z_\text{Zn} &= 30\\
    Z_\text{Ga} &= 31\\
    Z_\text{Br} &= 35\\
    Z_\text{Rb} &= 37\\
    Z_\text{Sr} &= 38\\
    Z_\text{Zr} &= 40.
\end{align*}

Zusätzlich werden zwei Naturkonstanten benötigt, die Rydbergenergie $R_\infty$ und die Sommerfeldsche Feinstrukturkonstante $\alpha$. \cite{physics_constants}
Dann wird die Abschirmkonstante $\sigma _\text{K}$ für alle sechs Elemente berechnet.
Und die Berechnung ergibt die Werte

\begin{align*}
    \sigma_\text{K,Br} &= 3.84\\
    \sigma_\text{K,Ga} &= 3.67\\
    \sigma_\text{K,Rb} &= 4.11\\
    \sigma_\text{K,St} &= 4.12\\
    \sigma_\text{K,Zn} &= 3.63\\
    \sigma_\text{K,Zr} &= 4.28.
\end{align*}

Um schlussendlich die Rydbergkonstante zu erhalten, wird das Moseley'schen Gesetz \eqref{eq:moseley} in eine Geradengleichung umgeformt.
Damit erhält sie die Form 

\begin{equation}
    \sqrt{E_\text{K,abs}} = \sqrt{R \, h} \cdot Z - \sqrt{R \, h \, \sigma_\text{K}}.
\end{equation}

Um den Wert für die Rydberdfrequenz $R$ zu erhalten, wird zu nächst die Gerade geplottet.
Dafür werden die Werte für $Z$, $E_\text{K,abs}$ und $\sigma_\text{K}$ aus den vorherigen Berechnungen verwendet.