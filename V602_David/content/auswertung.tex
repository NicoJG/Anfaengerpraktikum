\section{Auswertung}
\label{sec:Auswertung}

% Messwerte: Alle gemessenen physikalischen Größen sind übersichtlich darzustellen.

% Auswertung:
% Berechnung der geforderten Endergebnisse
% mit allen Zwischenrechnungen und Fehlerformeln, sodass die Rechnung nachvollziehbar ist.
% Eine kurze Erläuterung der Rechnungen (z.B. verwendete Programme)
% Graphische Darstellung der Ergebnisse

\subsection{Überprüfung der Bragg-Bedingung}
\label{sec:1}

Die Messdaten aus \label{tab:bragg} werden nun geplottet.

\begin{figure}
    \centering
    \includegraphics[width=\textwidth]{build/plot_spektrum.pdf}
    \caption{Plot der Messungen für die Bragg-Bedingung}
    \label{fig:bragg}
\end{figure}

Aus dem Plot und den Daten kann entnommen werden, dass das Maximum der Intensität bei 

\begin{equation}
    \theta _\text{max} = \SI{28.2}{\degree}
\end{equation}

liegt.

\begin{figure}
    \centering
    \includegraphics[width=\textwidth]{build/plot_emissionCu.pdf}
    \caption{Plot der Emission der Cu-Röntgenröhre}
    \label{fig:kupfer}
\end{figure}