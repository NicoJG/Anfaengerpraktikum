\section{Diskussion}
\label{sec:Diskussion}

% Kurze Zusammenfassung der Ergebnisse
% -Vergleich mit Literaturwerten
% -Vergleich mit verschiedenen Messverfahren
% -bei Abweichungen mögliche Ursachen finden

\subsection{Vergleich der Ergebnisse mit den Literaturwerten}
\label{ssec:vergleich}

Das Maximum der Intensität bei \autoref{ssec:1} wurde bei

\begin{equation}
    \theta _\text{max} = \SI{28.2}{\degree}
\end{equation}

festgestellt.
Da der Kristallwinkel, also der Einfallswinkel, $\SI{14}{\degree}$ beträgt, ist das Maximum bei $\SI{28}{\degree}$ zu erwarten, da zustätzlich noch der Ausfallswinkel dazu addiert wird.
Beide Winkel sind nach dem Prinzip der Reflexion gleich groß.
Damit weicht der gemessene Wert um $0.71\%$ ab.
Auch wenn das Ergebnis sehr präszise ist, könnte diese geringe Abweichung an möglichen Ungenauigkeit beim justieren liegen, auch wenn es maschinell geschehen ist.
So könnte die Linse oder LiF-Kristall ebenfalls leicht versetzt montiert worden sein.

In \autoref{ssec:2} konnte die minimale Wellenlänge bzw. die maximale Energie des Bremsberges nicht berechnet werden, da das vorhandene Spektrum dafür nicht genügt.
Um diese Stelle zu finden wurde nicht weit genug gemessen, der entsprechende Teil ist gar nicht zu sehen, vermutlich ist er bei niedrigeren Winkeln zu finden.

In \autoref{ssec:3} wurden die sechs Absorptionsenergien und Abschirmkonstanten berechnet, auch diese werden mit den Literaturwerten verglichen.
 
\begin{table}
  \centering
  \caption{Vergleich der experimentellen und der theoretischen Absorptionsenergien. \cite{absorption}}
  \label{tab:e_lit}
  \begin{tabular}{c S[table-format=1.2] S[table-format=1.2] S[table-format=1.2]}
    \toprule 
   \text{Element} & \tableSI{E_\text{K}}{\kilo\electronvolt} & \tableSI{E_\text{K,Lit}}{\kilo\electronvolt} & \text{Abweichungen} \\ 
    \midrule 
    Brom & 13.48 & 13.47 & 0.07 \% \\
    Gallium & 10.32 & 10.37 & 0.48 \% \\
    Rubidium & 15.05 & 15.20 & 0.97 \% \\
    Strontium & 15.99 & 16.10 & 0.68 \% \\
    Zink & 9.60 & 9.65 & 0.52 \% \\
    Zirkonium & 17.81 & 18.00 & 1.06 \% \\
    \bottomrule
  \end{tabular}
\end{table} 

Die Abweichungen zeigen, dass die Ergenisse alle sehr genau bestimmt wurden.
Das ist durchaus überraschend, da im Laufe der Auswertung einige Näherungen getroffen wurden.
Bei der Bestimmung der Halbwertsbreite wurden durch recht ungenaue Geradennäherungen Werte ermittelt.
Außerdem wurde bei der Zuordnung der Intensitäten $I_\text{K}$ zu ihren zugehörigen Winkeln $\theta$ sehr stark gerundet.
Oft lagen Werte genau zwischen den diskreten Messwerten, die zu verfügung standen. 
Dort wurde es nötig nach Augenmaß einen Winkel festzulegen.  

\begin{table}
  \centering
  \caption{Vergleich der berechneten und theoretischen Abschirmkonstanten. \cite{absorption}}
  \label{tab:sigma_lit}
  \begin{tabular}{c S[table-format=1.2] S[table-format=1.2] S[table-format=1.2]}
    \toprule 
   \text{Element} & $\sigma _\text{K}$ & $\sigma _\text{K,Lit}$ & \text{Abweichungen} \\ 
    \midrule 
    Brom & 3.84 & 3.84 & 0.00 \% \\
    Gallium & 3.67 & 3.61 & 1.63 \% \\
    Rubidium & 4.11 & 3.94 & 4.13 \% \\
    Strontium & 4.12 & 3.99 & 3.15 \% \\
    Zink & 3.63 & 3.56 & 1.93 \% \\
    Zirkonium & 4.28 & 4.09 & 4.44 \% \\
    \bottomrule
  \end{tabular}
\end{table}

Die Literaturwerte von $\sigma$ wurden über die Literaturwerte der $E_\text{K}$-Kanten bestimmt. Daher sind die Verhältnisse gleich. 


Das Ziel der Auswertung war die Bestimmung der Rydbergenergie, dieser Wert wird ebenfalls mit dem Wert aus der Literatur verglichen.

\begin{table}
  \centering
  \caption{Vergleich der ausgerechneten Rydbergenergie mit dem Literaturwert. \cite{physics_constants}}
  \label{tab:rydberg_lit}
  \begin{tabular}{S[table-format=2.2,table-figures-uncertainty = 1] S[table-format=1.2] S[table-format=1.2]}
    \toprule 
   \tableSI{R_\infty}{\electronvolt} & \tableSI{R_{\infty, Lit}}{\electronvolt} & \text{Abweichungen} \\ 
    \midrule 
    12.56+-0.15 & 13.61 & 6.61 \% \\
  \end{tabular}
\end{table}

Obwohl die vorherigen Werte alle eine recht geringe Abweichung besitzen, weicht die Rydbergenergie etwas deutlicher ab, allerdings ist die as Ergebnis dennoch unter $10 \%$ Abweichung.
Eine Erklärung wäre, dass die Rydbergenergie über die Steigung der Ausgleichsgeraden berechnet wurde und diese schon durch einen Ausreißer im Plot leicht verändert werden kann.