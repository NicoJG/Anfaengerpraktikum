\section{Theorie}
\label{sec:Theorie}

% In knapper Form sind die physikalischen Grundlagen des Versuches, des Messverfahrens, sowie sämtliche für die Auswertung erforderlichen Gleichungen darzustellen. (Keine Herleitung)

% (eventuell die Aufgaben)

% Der Versuchsaufbau: Beschreibung des Versuchs und der Funktionsweise (mit Skizze/Bild/Foto)

Um die Absorption und Emission von Röntgenstrahlung zu untersuchen, wird zunächst die Strahlung an sich in ihre Bestandteile zerlegt.
Die Erzeugung von Röntgenstrahlung gelingt über eine Glühkathode, dabei werden Elektronen freigesetzt.
Diese werden bei ihrem Durchlauf in einer evakuierten Röhre auf eine Anode hin beschleunigt.
Die Elektronen werden im Coulomb-Feld des Atoms abgebremst und geben im Zuge dessen ein Röntenquant ab.
Dabei ist die Energie dieses Quants genau die Energie, die das Elektron verloren hat, dadurch entsteht das kontinuierliche Bremsspektrum.
Die kinetische Energie eines Elektrons lässt sich über 

\begin{equation}
    E_\text{kin} = U \cdot e_0
    \label{eq:ekin}
\end{equation}

berechnen. $U$ ist die angelegte Spannung und $e$ die Elementarladung.
Gibt ein Elektron seine gesamte Energie ab, entsteht ein Photon mit der minimalen Wellenlänge 

\begin{equation}
    \lambda _\text{min} = \frac{h \cdot c}{U \cdot e_0}.
    \label{eq:lambdamin}
\end{equation}

Der zweite Bestandteil der Strahlung ist das charakteristische Spektrum.
Wegen der Ionisation des Anodenmaterials entsteht eine Leerstelle, die von einem Elektron einer höheren Schale wieder besetzt wird.
Rückt ein Elektron nach so wird ein Röntenquant mit exakt der Energiedifferenz der beiden Schalen abgegeben.
Diese Energien sind dadurch diskret und sorgen für stark ausgeprägte Linien im Spektrum.
Es muss noch berücksichtigt werden, dass in einem Mehrelektronenatom die äußeren Elektronen leicht von den Inneren abgeschirmt werden, sie sind also weniger fest gebunden.
Die Energie eines solchen Elektrons auf der n-ten Schale kann über 

\begin{equation}
    E_\text{n} = - R_\infty \cdot z_\text{eff}^2 \cdot \frac{1}{n^2}
    \label{eq:en}
\end{equation}

berechnet werden.
$R_\infty$ ist dabei die Rydbergenergie und $z_\text{eff}$ die effektive Kernladung.
Wobei diese wiederum über

\begin{equation}
    z_\text{eff} = z - \sigma
    \label{eq:z}
\end{equation}

definiert ist.
$\sigma$ ist die Abschirmkonstante.

Streng genommen unterscheiden sich die Elektronen auf den Schalen durch verschiedene Bahndrehimpulse und Spins, daher kann die Bindungsenergie $E_\text{n,j}$ der Elektronen genau über die sommerfeldsche Strukturformel

\begin{equation}
    E_\text{n,j} = - R_\infty \left(z_\text{eff,1}^2 \cdot \frac{1}{n^2} + \alpha ^2 z_\text{eff,2}^4  \cdot \frac{1}{n^3} \left(\frac{1}{j + 0.5} - \frac{3}{4 n}   \right)\right)
    \label{eq:sommerfeld}
\end{equation}

berechnet werden.
$n$ steht hier für die Hauptquantenzahl, $\alpha$ für die Sommerfeldsche Feinstrukturkonstante und $j$ für den Gesamtdrehimpuls.
Für Elektronen auf der K-Schale ist die Abschirmkonstante über 

\begin{equation}
    \sigma _\text{K} = Z - \sqrt{\frac{E_\text{K}}{R_\infty} - \frac{\alpha ^2 z_\text{eff,2}^4}{4}}
    \label{eq:sigma}
\end{equation}

definiert.

Damit die Röntgenstrahlung hinsichtlich ihrer Energie und Wellenlänge untersucht werden kann wird die Bragg'sche Reflexion verwendet.
Photonen werden an einem LiF-Kristalls um den Winkel $\alpha$ gebrochen.
Daraus ergibt sich dann folgender Zusammenhang,

\begin{equation}
    2 d \sin{\alpha} = n \cdot \lambda.
    \label{eq:bragg}
\end{equation}

$n$ ist der Brechungsgrad des Kristalls und $d$ hier die Gitterkonstante.
Wird \eqref{eq:bragg} nach $\lambda$ umgestellt, ergibt sich 

\begin{equation}
   \lambda = \frac{2 d \sin{\alpha}}{n}.
   \label{eq:lambda}
\end{equation}