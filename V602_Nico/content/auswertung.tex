\newpage
\section{Auswertung}
\label{sec:Auswertung}

% Messwerte: Alle gemessenen physikalischen Größen sind übersichtlich darzustellen.

% Auswertung:
% Berechnung der geforderten Endergebnisse
% mit allen Zwischenrechnungen und Fehlerformeln, sodass die Rechnung nachvollziehbar ist.
% Eine kurze Erläuterung der Rechnungen (z.B. verwendete Programme)
% Graphische Darstellung der Ergebnisse

Auch die Auswertung ist, ähnlich wie die Durchführung, in 4 Teile aufgeteilt:
\begin{itemize}
    \item Überprüfung der Bragg Bedingung
    \item Untersuchung des Emissionsspektrums von Kuper
    \item Untersuchung der Absorptionsspektren verschiedener Materialien
    \item Bestimmung der Rydbergenergie
\end{itemize}



\subsection{Überprüfung der Bragg Bedingung}
\label{ssec:bragg_auswertung}

Die Messergebnisse aus Teil 1 der Durchführung sind in \autoref{tab:bragg} aufgelistet und in \autoref{fig:plot_bragg} dargestellt.
Die Maximale Intensität ergibt sich bei einem Winkel $\theta=\SI{28.2+-0.1}{\degree}$, wobei der Sollwinkel $\SI{28}{\degree}$ beträgt.

\begin{figure}
    \centering
    \includegraphics[width=\textwidth]{build/plot_bragg.pdf}
    \caption{Plot der Messergebnisse aus Abschnitt \ref{ssec:bragg}}
    \label{fig:plot_bragg}
\end{figure}



\subsection{Untersuchung des Emissionsspektrums von Kuper}
\label{ssec:emission_auswertung}

Die Messergebnisse aus Teil 2 der Durchführung sind in \autoref{tab:emission} aufgelistet und in \autoref{fig:plot_emission} dargestellt.
Hier lassen sich die $K$-Linien bei
\begin{align*}
    \theta(K_\alpha) &= \SI{22.5}{\degree} \\
    \theta(K_\beta) &= \SI{20.2}{\degree}
\end{align*}
ablesen und ergeben über \autoref{eq:bragg} und \ref{eq:energie} die Bindungsenergien und entsprechende Literaturwerte
\begin{align*}
    E(K_\alpha) &= \SI{8044}{\electronvolt} & E_\text{Lit}(K_\alpha) &= \SI{8048}{\electronvolt} \\
    E(K_\beta) &= \SI{8915}{\electronvolt} & E_\text{Lit}(K_\beta) &= \SI{8905}{\electronvolt} \,.\text{\cite{xray}}
\end{align*}

\begin{figure}
    \centering
    \includegraphics[width=\textwidth]{build/plot_emission.pdf}
    \caption{Plot der Messergebnisse aus Abschnitt \ref{ssec:emission}}
    \label{fig:plot_emission}
\end{figure}

Nun soll das Auflösungsvermögen $A(K)=E(K)/H(K)$ der Röntgenstrahlung bestimmt werden, wobei $H(K)$ die Halbwärtsbreite der $K$-Linie ist.
Diese Halbwärtsbreite wird aufgrund der beschränkten Anzahl an Messpunkten um die $K$-Linien folgendermaßen bestimmt.
Zuerst werden die Messwerte gesucht, dessen Intensitätswerte am nächsten an $N(K)/2$ liegen.
Zwischen diesen Punkten wird jeweils eine Gerade gelegt.
Für diese Geraden wird der passende $\theta$ Wert für $N(K)/2$ bestimmt und wie zuvor über \autoref{eq:bragg} und \autoref{eq:energie} in eine Energie umgerechnet.
Die Differenzen dieser Energien ergeben somit die Halbwärtsbreiten
\begin{align*}
    H(K_\alpha) &= \SI{165.8}{\electronvolt} \\
    H(K_\beta) &= \SI{207.0}{\electronvolt} \,.
\end{align*}
Damit ergeben sich die Auflösungsvermögen
\begin{align*}
    A(K_\alpha) &= \num{48.5} \\
    A(K_\beta) &= \num{43.1} \,.
\end{align*}

Außerdem werden aus den Energien $E(K)$ die Abschirmkonstanten $\sigma_1$ der Absorptionsenergie von Kupfer, $\sigma_2$ der $K_\alpha$-Linie und $\sigma_3$ der $K_\beta$-Linie bestimmt.
Da die Absorptionsenergie von Kuper hier nicht gemessen wurde, wird diese aus entsprechender Literatur zu $E_\text{abs}=\SI{8979}{\electronvolt}$ gewählt.\cite{absorption}
Abschätzungen der Abschirmkonstanten lassen sich über 
\begin{align}
    \sigma_1 &= Z - \sqrt{\frac{E_\text{abs}}{R_\infty}} \\
    \sigma_2 &= Z - \sqrt{4(Z-\sigma_1)^2-\frac{E(K_\alpha)}{R_\infty}} \\
    \sigma_3 &= Z - \sqrt{9(Z-\sigma_1)^2-\frac{E(K_\beta)}{R_\infty}}
\end{align}
ermitteln.\cite[Gleichungen (8),(9),(10)]{V602} 
Die Ordnungszahl von Kupfer ist $Z=29$.
Damit ergeben sich die Abschirmkonstanten zu
\begin{align*}
    \sigma_1 &= \num{3.3} \\
    \sigma_2 &= \num{12.4} \\
    \sigma_3 &= \num{22.5} \, .
\end{align*}



\subsection{Untersuchung der Absorptionsspektren verschiedener Materialien}
\label{ssec:absorption_auswertung}

Die Messergebnisse aus Teil 3 der Durchführung sind in den Tabellen \ref{tab:zink}-\ref{tab:zirconium} aufgelistet und in \autoref{fig:plot_absorption} dargestellt.

\begin{figure}
    \centering
    \includegraphics[width=\textwidth]{build/plot_absorption.pdf}
    \caption{Plot der Messergebnisse aus Abschnitt \ref{ssec:absorption}, mit Markierungen der bestimmten $K$-Kanten}
    \label{fig:plot_absorption}
\end{figure}

Aus den Messdaten kann nun jeweils die Absorptionsenergie bestimmt werden.
Dazu wird die $K$-Kante näherungsweise auf die Mitte der Kante festgelegt.
Also wird der nächstbeste $\theta$ Wert zu $N=N_\text{min} + \frac{1}{2}(N_\text{max}-N_\text{min})$ bestimmt bzw. die Mitte zweier Messwerte falls die gesuchte Intensität ungefähr in der Mitte dieser Messpunkte liegt.
Die so gefundenen $K$-Kanten sowie deren Energieäquivalent aus \autoref{eq:bragg} und \ref{eq:energie} sind in \autoref{tab:absorption} aufgelistet.

Nun können aus den Absorptionsenergien über \autoref{eq:abschirm} die Abschirmkonstanten der Absorbermaterialien bestimmt werden.

Um die bestimmten Werte vergleichen zu können, werden über Absorptionsenergien aus externer Literatur und die gleichen drei Gleichungen wie zuvor die Abschirmkonstanten und die Bragg-Winkel berechnet.
Auch diese Werte sind in \autoref{tab:absorption} aufgelistet.

\begin{table}
    \centering
    \caption{Ergebnisse und Literaturwerte der Absorptionsenergie, des Bragg-Winkels und der Abschirmkonstante.\cite{absorption}}
    \sisetup{
        table-format=2.2
    }
    \begin{tabular}{c S[table-format=1.0] S S S S S S}
        \toprule
        Element & $Z$ & \tableSI{E}{\kilo\electronvolt} & \tableSI{E_\text{Lit}}{\kilo\electronvolt} & \tableSI{\theta}{\degree} & \tableSI{\theta_\text{Lit}}{\degree} & $\sigma$ & $\sigma_\text{Lit}$ \\
        \midrule
        Zn & 30 & 9.60 & 9.65 & 18.70 & 18.60 & 3.63 & 3.57 \\
        Ga & 31 & 10.32 & 10.37 & 17.35 & 17.30 & 3.67 & 3.61 \\
        Br & 35 & 13.43 & 13.47 & 13.25 & 13.20 & 3.89 & 3.85 \\
        Rb & 37 & 15.05 & 15.20 & 11.80 & 11.70 & 4.11 & 3.94 \\
        Sr & 38 & 15.99 & 16.10 & 11.10 & 11.00 & 4.12 & 4.00 \\
        Zr & 40 & 17.73 & 18.00 & 10.00 & 9.80 & 4.37 & 4.09 \\
        \bottomrule
    \end{tabular}
    \label{tab:absorption}
\end{table}



\subsection{Bestimmung der Rydbergenergie}
\label{ssec:rydberg}

Nun lässt sich aus den im vorherigen Abschnitt berechneten Absorptionsenergien sowie Abschirmkonstanten die Rydbergenergie über \autoref{eq:moseley} ermitteln.
Dazu wird ein Plot aus den Werten der \autoref{tab:absorption} angefertigt und eine passende Ausgleichsgerade mit
\begin{equation*}
    \sqrt{E} = aZ+b
\end{equation*}
über die Python Funktion curve\_fit aus der Bibliothek Scipy bestimmt.
Somit ergeben sich die Parameter
\begin{align*}
    a &= \SI{3.52+-0.02}{\sqrt{\electronvolt}} \\
    b &= \SI{-7.6+-0.7}{\sqrt{\electronvolt}} \,.
\end{align*}

\begin{figure}
    \centering
    \includegraphics[width=\textwidth]{build/plot_rydberg.pdf}
    \caption{Plot der Ergebnisse aus \autoref{tab:absorption} mit passender Ausgleichsgerade.}
    \label{fig:plot_rydberg}
\end{figure}

Die Rydbergenergie wird über $R_\infty=a^2$ zu 
\begin{equation*}
    R_\infty = \SI{12.4+-0.1}{\electronvolt}
\end{equation*}
bestimmt.

Ein Literaturwert der Rydbergenergie lautet $R_\infty=\SI{13.6}{\electronvolt}$.\cite{V602}