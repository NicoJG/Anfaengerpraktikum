\section{Diskussion}
\label{sec:Diskussion}

Um eine Aussage über die Genauigkeit aller hier ermittelten Grö0en treffen zu können,
muss zunächst die Genauigkeit der Bragg Bedingung betrachtet werden.
Hier ergab die Messung zu Anfang des Versuchs eine Abweichung vom Sollwinkel von
\begin{equation*}
    \Delta\theta = \SI{0.7}{\percent} \, .
\end{equation*}
Diese Abweichung scheint zwar ausreichend gering, allerdings sind durch diese Ungenauigkeit alle weiteren ermittelten Werte auch ungenau.

Der Plot des Röntgenemissionsspektrums von Kupfer zeigt ein erwartetes Bild.
Die ermittelten $K_\alpha$ und $K_\beta$ Energien bestätigen dies und zeigen eine Abweichung 
\begin{align*}
    \Delta E(K_\alpha) &= \SI{0.05}{\percent} \\
    \Delta E(K_\beta) &= \SI{0.11}{\percent}
\end{align*}
von den angegebenen Literaturwerten.
Eine minimale bzw. maximale Wellenlänge des Bremsberges konnte nicht ermittelt werden, da hierfür zu wenig Messwerte aufgenommen wurden und der Apperat nicht gut von äußerer Strahlung abgeschirmt wurde.

%%%%%%%%%%%% Auflösungsvermögen und Abschirmkonstanten

Betrachtet man nun die Ergebnisse der Absorptionsuntersuchung stellt man ähnlich geringe Abweichungen fest.
Diese Abweichungen sind in \autoref{tab:absorption_abweichung} aufgelistet.
\begin{table}
    \centering
    \caption{Abweichungen der in Abschnitt \ref{ssec:absorption_auswertung} berechneten Werte zum Literaturwert}
    \begin{tabular}{c S S S}
        \toprule
        Element & \tableSI{\Delta E}{\percent} & \tableSI{\Delta\theta}{\percent} & \tableSI{\Delta\sigma}{\percent} \\
        \midrule
        Zn & 0.50 & 0.54 & 1.77 \\
        Ga & 0.45 & 0.29 & 1.75 \\
        Br & 0.29 & 0.38 & 1.14 \\
        Rb & 0.96 & 0.85 & 4.25 \\
        Sr & 0.68 & 0.91 & 2.96 \\
        Zr & 1.51 & 2.04 & 6.87 \\
        \bottomrule
    \end{tabular}
    \label{tab:absorption_auswertung}
\end{table}
Auch die Plots der Absorptionskurven in \autoref{fig:plot_absorption} zeigen das erwartete Bild.

Als letztes wurde die Rydbergenergie aus den Ergebnissen ermittelt und diese zeigt eine Abweichung
\begin{equation*}
    \Delta R_\infty = \SI{8.8}{\percent}
\end{equation*}
vom Vergleichswert. 
Diese Abweichung ist somit die Größte der hier ermittelten Wert, was allerdings nicht verwunderlich ist, da alle Abweichungen der vorherigen Messungen hier einen Beitrag leisten.

Alle Abweichungen scheinen angemessen, da viele systematische Fehlerquellen nicht beachtet wurden.
Z.B. wurde der Messapperat nicht sonderlich gut von äußerer Strahlung abgeschirmt und die Bragg Bedingung hätte noch besser erfüllt gewesen sein können.
Außerdem wurden die einzelnen Messwert nur mit einer Integrationszeit von 5 Sekunden und in Schritten von $\SI{0.1}{\degree}$ gemessen ohne dabei die Totzeit des Geiger-Müller Zählers zu beachten.
Zudem kommen die in der Theorie gemachten Näherungen hinzu.