\section{Auswertung}
\label{sec:Auswertung}

% Messwerte: Alle gemessenen physikalischen Größen sind übersichtlich darzustellen.

% Auswertung:
% Berechnung der geforderten Endergebnisse
% mit allen Zwischenrechnungen und Fehlerformeln, sodass die Rechnung nachvollziehbar ist.
% Eine kurze Erläuterung der Rechnungen (z.B. verwendete Programme)
% Graphische Darstellung der Ergebnisse

\subsection{Bestimmung des Emissionsspektrums}
\label{ssec:1}

Die gemessenen Impulse pro Sekunde werden in \autoref{tab:werte_emissionCu} aufgelistet.

\begin{table}
    \centering
    \caption{Messergebnisse für das Emissionsspektrums}
    \label{tab:werte_emissionCu}
    \begin{tabular}{S[table-format=1.1] S[table-format=4.1]}
    \toprule
    \tableSI{\theta}{\circ} & \tableSI{N}{\frac{\text{Imp}}{\second}} \\
    \midrule
    8.0 &	323.0\\
    8.1 &	316.0\\
    8.2	&	326.0\\
    8.3	&	340.0\\
    8.4	&	335.0\\
    8.5	&	343.0\\
    8.6	&	350.0\\
    8.7	&	350.0\\
    8.8	&	366.0\\
    8.9	&	357.0\\
    9.0	&	371.0\\
    9.1	&	371.0\\
    9.2	&	372.0\\
    9.3	&	364.0\\
    9.4	&	381.0\\
    9.5	&	379.0\\
    9.6	&	393.0\\
    9.7	&	375.0\\
    9.8	&	391.0\\
    9.9	&	395.0\\
    10.0&	402.0\\
    10.1&	405.0\\
    10.2&	390.0\\
    10.3&	398.0\\
    10.4&	400.0\\
    10.5&	418.0\\
    10.6&	401.0\\
    10.7&	410.0\\
    10.8&	408.0\\
    10.9&	409.0\\
    11.0&	414.0\\
    11.1&	420.0\\
    11.2&	417.0\\
    11.3&	417.0\\
    11.4&	409.0\\
    11.5&	406.0\\
    11.6&	404.0\\
    11.7&	405.0\\
    11.8&	400.0\\
    11.9&	383.0\\
    12.0&	389.0\\
    12.1&	382.0\\
    12.2&	372.0\\
    \bottomrule
    \end{tabular}
    \begin{tabular}{S[table-format=1.1] S[table-format=4.1]}
    \toprule
    \tableSI{\theta}{\circ} & \tableSI{N}{\frac{\text{Imp}}{\second}} \\
    \midrule
    12.3&	376.0\\
    12.4&	385.0\\
    12.5&	384.0\\
    12.6&	382.0\\
    12.7&	373.0\\
    12.8&	376.0\\
    12.9&	373.0\\
    13.0&	375.0\\
    13.1&	366.0\\
    13.2&	354.0\\
    13.3&	341.0\\
    13.4&	326.0\\
    13.5&	318.0\\
    13.6&	305.0\\
    13.7&	296.0\\
    13.8&	286.0\\
    13.9&	285.0\\
    14.0&	274.0\\
    14.1&	264.0\\
    14.2&	266.0\\
    14.3&	270.0\\
    14.4&	255.0\\
    14.5&	255.0\\
    14.6&	260.0\\
    14.7&	251.0\\
    14.8&	250.0\\
    14.9&	248.0\\
    15.0&	253.0\\
    15.1&	257.0\\
    15.2&	248.0\\
    15.3&	242.0\\
    15.4&	249.0\\
    15.5&	246.0\\
    15.6&	252.0\\
    15.7&	236.0\\
    15.8&	234.0\\
    15.9&	231.0\\
    16.0&	215.0\\
    16.1&	217.0\\
    16.2&	227.0\\
    16.3&	214.0\\
    16.4&	217.0\\
    16.5&	210.0\\
    \bottomrule
    \end{tabular}
    \begin{tabular}{S[table-format=1.1] S[table-format=4.1]}
    \toprule
    \tableSI{\theta}{\circ} & \tableSI{N}{\frac{\text{Imp}}{\second}} \\
    \midrule
    16.6&	211.0\\
    16.7&	206.0\\
    16.8&	205.0\\
    16.9&	198.0\\
    17.0&	203.0\\
    17.1&	199.0\\
    17.2&	198.0\\
    17.3&	191.0\\
    17.4&	192.0\\
    17.5&	184.0\\
    17.6&	191.0\\
    17.7&	188.0\\
    17.8&	181.0\\
    17.9&	185.0\\
    18.0&	184.0\\
    18.1&	179.0\\
    18.2&	180.0\\
    18.3&	166.0\\
    18.4&	173.0\\
    18.5&	167.0\\
    18.6&	169.0\\
    18.7&	160.0\\
    18.8&	159.0\\
    18.9&	157.0\\
    19.0&	149.0\\
    19.1&	153.0\\
    19.2&	150.0\\
    19.3&	147.0\\
    19.4&	150.0\\
    19.5&	148.0\\
    19.6&	149.0\\
    19.7&	143.0\\
    19.8&	153.0\\
    19.9&	182.0\\
    20.0&	291.0\\
    20.1&	1127.0\\
    20.2&	1599.0\\
    20.3&	1533.0\\
    20.4&	1430.0\\
    20.5&	1267.0\\
    20.6&	425.0\\
    20.7&	241.0\\
    20.8&	225.0\\
    \bottomrule
    \end{tabular}
    \begin{tabular}{S[table-format=1.1] S[table-format=4.1]}
    \toprule
    \tableSI{\theta}{\circ} & \tableSI{N}{\frac{\text{Imp}}{\second}} \\
    \midrule
    20.9&	192.0\\
    21.0&	188.0\\
    21.1&	172.0\\
    21.2&	168.0\\
    21.3&	169.0\\
    21.4&	166.0\\
    21.5&	170.0\\
    21.6&	174.0\\
    21.7&	164.0\\
    21.8&	180.0\\
    21.9&	179.0\\
    22.0&	191.0\\
    22.1&	232.0\\
    22.2&	300.0\\
    22.3&	536.0\\
    22.4&	4128.0\\
    22.5&	5050.0\\
    22.6&	4750.0\\
    22.7&	4571.0\\
    22.8&	4097.0\\
    22.9&	901.0\\
    23.0&	244.0\\
    23.1&	179.0\\
    23.2&	151.0\\
    23.3&	145.0\\
    23.4&	130.0\\
    23.5&	121.0\\
    23.6&	126.0\\
    23.7&	117.0\\
    23.8&	112.0\\
    23.9&	110.0\\
    24.0&	105.0\\
    24.1&	106.0\\
    24.2&	107.0\\
    24.3&	95.0\\
    24.4&	94.0\\
    24.5&	100.0\\
    24.6&	91.0\\
    24.7&	85.0\\
    24.8&	88.0\\
    24.9&	83.0\\
    25.0&	85.0\\
    &\\
        \bottomrule
    \end{tabular}
\end{table}

Die Werte aus \autoref{tab:werte_emissionCu} werden anschließend geplottet, dort werden zwei Winkel ganz besonders deutlich erkennbar.

\begin{figure}
    \centering
    \includegraphics[width=\textwidth]{build/plot_emissionCu.pdf}
    \caption{Emissionsspektrum der Messwerte aus \autoref{tab:werte_emissionCu}}
    \label{fig:emissionCu_plot}
\end{figure}

Bei den Winkeln 
\begin{equation}
    \theta _1 = 20.2 \circ
\end{equation}
\begin{equation}
    \theta _2 = 22.5 \circ
\end{equation}
sind die Maxima des Spektrums, aus diesen kann über die Formel 

\begin{equation}
    E = \frac{h \cdot c \cdot n}{2d \cdot \sin{\alpha}}
\end{equation}

die Energie $E$ berechnet werden.
Dabei ist $h$ das Planksche Wirkungsquantum und $c$ die Lichtgeschwindigkeit.
Die Gitterkonstante des verwendeten LiF-Kristalls beträgt $d = \SI{201.4}{\pico\meter}$ mit der Ordnung $n = 1$.

Dadurch betragen die beiden Energien für die $K_\alpha$ und $K_\beta$ Linien

\begin{equation}
    E_\alpha = \SI{1.2877e-15}{\joule}
\end{equation}
\begin{equation}
    E_\beta = \SI{1.4265e-15}{\joule}
\end{equation}

In der Literatur wird selten die Einheit $\si{joule}$ für diese Größenordnung verwendet, daher werden die Energien für den Vergleich in $\si{electronvolt}$ umgerechnen.
Dafür werden die Größen durch die Elektronenladung $e$ geteilt.

\begin{equation}
    E_\alpha = \SI{8.037}{\kilo\electronvolt}
\end{equation}
\begin{equation}
    E_\beta = \SI{8.903}{\kilo\electronvolt}
\end{equation}

Nun werden die berechneten Werte mit den Literaturwerten

\begin{equation}
    E_{\alpha,lit} = \SI{8.048}{\kilo\electronvolt}
\end{equation}
\begin{equation}
    E_{\beta,lit} = \SI{8.905}{\kilo\electronvolt}
\end{equation}

verglichen. \cite{emissionslinien}
Die Abweichungen sind dabei minimal.

\subsection{Bestimmung der Transmission als Funktion der Wellenlänge}
\label{ssec:2}

\begin{table}
    \centering
    \caption{Messwerte für die Bestimmung der Compton Wellenlänge}
    \label{tab:werte_2}
    \begin{tabular}{S[table-format=1.1] S[table-format=3.1] S[table-format=3.1]}
        \toprule
        \tableSI{\theta}{\circ} & \tableSI{N_0}{\frac{\text{Imp}}{\second}} \tableSI{N_\text{Al}}{\frac{\text{Imp}}{\second}} \\
        \midrule
        7.0 & 226.0 & 113.5\\
        7.1 & 232.0 & 112.0\\
        7.2 & 240.5 & 112.0\\
        7.3 & 248.0 & 113.5\\
        7.4 & 255.0 & 115.0\\   
        7.5 & 262.0 & 113.5\\
        7.6 & 269.0 & 113.0\\
        7.7 & 276.0 & 114.5\\
        7.8 & 281.0 & 114.0\\    
        7.9 & 289.5 & 112.0\\
        8.0 & 295.0 & 109.5\\
        8.1 & 300.0 & 109.0\\
        8.2 & 308.5 & 108.0\\
        8.3 & 311.0 & 106.0\\      
        8.4 & 317.0 & 104.5\\
        8.5 & 324.0 & 101.5\\
        \bottomrule
    \end{tabular}
    \begin{tabular}{S[table-format=1.1] S[table-format=3.1] S[table-format=3.1]}
        \toprule
        \tableSI{\theta}{\circ} & \tableSI{N_0}{\frac{\text{Imp}}{\second}} \tableSI{N_\text{Al}}{\frac{\text{Imp}}{\second}} \\
        \midrule
        8.6 & 328.5 & 100.0 \\
        8.7 & 332.5 & 100.5\\
        8.8 & 337.0 & 97.5\\
        8.9 & 340.5 & 95.0\\
        9.0 & 348.0 & 92.5\\
        9.1 & 350.0 & 89.5\\
        9.2 & 353.0 & 88.0\\
        9.3 & 356.5 & 84.5\\
        9.4 & 359.0 & 83.0\\
        9.5 & 363.5 & 81.0\\
        9.6 & 367.0 & 78.5\\
        9.7 & 369.0 & 76.0\\
        9.8 & 370.5 & 74.0\\
        9.9 & 375.0 & 72.0\\
        10.0 & 375.5 & 68.5\\
        & & \\
        \bottomrule
    \end{tabular}
\end{table}

