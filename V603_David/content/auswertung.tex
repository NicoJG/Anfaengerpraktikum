\section{Auswertung}
\label{sec:Auswertung}

% Messwerte: Alle gemessenen physikalischen Größen sind übersichtlich darzustellen.

% Auswertung:
% Berechnung der geforderten Endergebnisse
% mit allen Zwischenrechnungen und Fehlerformeln, sodass die Rechnung nachvollziehbar ist.
% Eine kurze Erläuterung der Rechnungen (z.B. verwendete Programme)
% Graphische Darstellung der Ergebnisse

\subsection{Bestimmung des Emissionsspektrums}
\label{ssec:1}

Die gemessenen Impulse pro Sekunde werden in \autoref{tab:werte_emissionCu} aufgelistet.
Die Werte werden anschließend geplottet, dort werden zwei Winkel ganz besonders deutlich erkennbar.

\begin{figure}
    \centering
    \includegraphics[width=\textwidth]{build/plot_emissionCu.pdf}
    \caption{Emissionsspektrum der Messwerte aus \autoref{tab:werte_emissionCu}}
    \label{fig:emissionCu_plot}
\end{figure}

Bei den Winkeln 

\begin{equation}
    \theta _1 = 20.2 \circ
\end{equation}
\begin{equation}
    \theta _2 = 22.5 \circ
\end{equation}

sind die Maxima des Spektrums, aus diesen kann über die Formel 

\begin{equation}
    E = \frac{h \cdot c \cdot n}{2d \cdot \sin{\alpha}}
\end{equation}

die Energie $E$ berechnet werden.
Dabei ist $h$ das Planksche Wirkungsquantum und $c$ die Lichtgeschwindigkeit. \cite{physics_constants}
Die Gitterkonstante des verwendeten LiF-Kristalls beträgt $d = \SI{201.4}{\pico\meter}$ mit der Ordnung $n = 1$.
Dadurch betragen die beiden Energien für die $K_\alpha$ und $K_\beta$ Linien

\begin{equation*}
    E_\alpha = \SI{1.2877e-15}{\joule} = \SI{8.037}{\kilo\electronvolt}
\end{equation*}
\begin{equation*}
    E_\beta = \SI{1.4265e-15}{\joule} = \SI{8.903}{\kilo\electronvolt}.
\end{equation*}

\subsection{Bestimmung der Transmission als Funktion der Wellenlänge}
\label{ssec:2}

Die Werte von $N_0$ und $N_\text{Al}$ geben jeweils nur die Impulse pro Sekunde an, daher werden sie mit der Messzeit $t = \SI{200}{\second}$ multipliziert, um die Gesamtanzahl der Impulse zu erhalten.
Nun wird noch die Totzeit des Geiger-Müller Zählers miteinberechnet. 
Über 

\begin{equation}
    I = \frac{N}{\left(1 - \tau \cdot N\right)}
    \label{eq:totzeit}
\end{equation}

Dabei ist $\tau = \SI{90}{\micro\second}$ die Totzeit des verwendeten Geiger-Müller Zählrohres.
Diese Rechnung wird für $N_0$ und $N_\text{Al}$ durchgeführt, dadurch ergeben sich $I_0$ und $I_\text{Al}$.
Aus diesen beiden Größen kann dann über 

\begin{equation}
    T = \frac{I_\text{Al}}{I_0}
\end{equation}

die Transmission $T$ berechnet werden, die für die Bestimmung der Compton-Wellenlänge benötigt wird.

Zuletzt werden aus den Winkeln $\alpha$ die Wellenlängen $\lambda$ bestimmt.
Das gelingt über \eqref{eq:lambda}.

Nun kann $\lambda$ gegen die Transmission $T$ geplottet werden, dann ergibt sich der Plot in \autoref{fig:compton_plot}.

\begin{figure}
    \centering
    \includegraphics[width=\textwidth]{build/plot_compton.pdf}
    \caption{Darstellung der Abhängigkeit von $\lambda$ und $T$}
    \label{fig:compton_plot}
\end{figure}

Die beiden Parameter für die Ausgleichsgerade sind mit 

\begin{align}
    a &= -15194732415 \pm \SI{239100069}{\per\meter} \label{eq:parameter1} \\
    b &= 1.23 \pm 0.01 \label{eq:parameter2}  
\end{align}

gegeben.

\subsection{Bestimmung der Compton-Wellenlänge}
\label{ssec:c}

Die drei Messungen für $I_0$, $I_1$ und $I_2$ ergeben folgende Werte

\begin{align*}
    I_0 &= 2731 \pm 52 \; \text{Impulse}\\
    I_1 &= 1180 \pm 34 \; \text{Impulse}\\
    I_2 &= 1024 \pm 32 \; \text{Impulse}.
\end{align*}

Verglichen mit den vorherigen Messungen ist hier keine Miteinberechnung der Totzeit notwendig, da vergeleichsweise wenig Impulse pro Sekunde gemessen werden.
Aus diesen Intensitäten können nun die beiden Transmissionen $T_1$ und $T_2$

\begin{align*}
    T_1 &= \frac{I_1}{I_0} &= 0.432 \pm 0.015\\
    T_2 &= \frac{I_2}{I_0} &= 0.375 \pm 0.014
\end{align*}

berechnet werden.
In \autoref{ssec:b} wird der linerare Zusammenhang zwischen $T$ und $\lambda$ deutlich.
Wird die Gleichung 

\begin{equation}
    T = a \cdot \lambda + b
\end{equation}

nach $\lambda$ umgestellt, können die entsprechenden Wellenlängen der Transmissionen berechnet werden.
Die Parameter $a$ und $b$ werden aus \eqref{eq:parameter1} und \eqref{eq:parameter2} genommen.
Damit ergeben sich durch die Formel 

\begin{equation}
    \lambda = \frac{T + b}{a}
\end{equation}

die beiden Wellenlängen

\begin{align*}
    \lambda _1 &= \SI{5.25(14)e-11}{\meter}\\
    \lambda _2 &= \SI{5.63(14)e-11}{\meter}.
\end{align*}

Mit \eqref{eq:deltalambda} kann dann die Compton-Wellenlänge des Elektrons

\begin{equation}
    \lambda _\text{C} = \SI{3.8(11)e-12}{\meter}
\end{equation}

berechnet werden.