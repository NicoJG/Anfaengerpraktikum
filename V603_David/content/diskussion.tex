\section{Diskussion}
\label{sec:Diskussion}

% Kurze Zusammenfassung der Ergebnisse
% -Vergleich mit Literaturwerten
% -Vergleich mit verschiedenen Messverfahren
% -bei Abweichungen mögliche Ursachen finden

Nun werden die berechneten Werte für $E$ mit den Literaturwerten verglichen.

\begin{table}
  \centering
  \caption{Vergleich der Energien des charakteristischen Spektrums. \cite{emissionslinien}}
  \label{tab:ergebnisse}
  \begin{tabular}{c S c S}
    \toprule 
   \tableSI{E_\text{gemessen}}{\kilo\electronvolt} & \tableSI{E_\text{Literatur}}{\kilo\electronvolt} & \text{Abweichungen} \\ 
    \midrule 
    8.037 & 8.048 & 0.137 \% \\
    8.903 & 8.905 & 0.022 \% \\
    \bottomrule
  \end{tabular}
\end{table}

Die Abweichungen sind dabei minimal.

