\section{Diskussion}
\label{sec:Diskussion}

% Kurze Zusammenfassung der Ergebnisse
% -Vergleich mit Literaturwerten
% -Vergleich mit verschiedenen Messverfahren
% -bei Abweichungen mögliche Ursachen finden

Nun werden die berechneten Werte für $E$ mit den Literaturwerten verglichen.

\begin{table}
  \centering
  \caption{Vergleich der Energien des charakteristischen Spektrums. \cite{emissionslinien}}
  \label{tab:ergebnisse}
  \begin{tabular}{S[table-format=1.3] S[table-format=1.3] S}
    \toprule 
   \tableSI{E_\text{gemessen}}{\kilo\electronvolt} & \tableSI{E_\text{Literatur}}{\kilo\electronvolt} & \text{Abweichungen} \\ 
    \midrule 
    8.037 & 8.048 & 0.137 \% \\
    8.903 & 8.905 & 0.022 \% \\
    \bottomrule
  \end{tabular}
\end{table}

Die Abweichungen sind dabei minimal.

Die Berechnung der Compton-Wellenlänge benötigte viele Zwischenschritte und kleinere Rundungen.

\begin{table}
  \centering
  \caption{Vergleich der gemessenen mit der theoretischen Compton-Wellenlänge. \cite{physics_constants}}
  \label{tab:ergebnisse2}
  \begin{tabular}{S[table-format=1.2,table-figures-uncertainty = 1] S[table-format=1.3] S}
    \toprule 
   \tableSI{\lambda _C / \text{gemessen}}{\pico\meter} & \tableSI{\lambda _C / \text{Literatur}}{\pico\meter} & \text{Abweichungen} \\ 
    \midrule 
    3.8 +- 1.1 & 2.43 & 10.00 \% \\
    \bottomrule
  \end{tabular}
\end{table}

Die Abweichung ist damit um einiges größer als bei den Energien zuvor.
Im Zuge der Rechnungen wurde einige male gerundet, was das Ergebnis abfälschen kann.
Zudem basiert ein Großteil der Rechnung auf einer linearen Regression, die ebenfalls sehr stark von Messffehler beeinflusst wird.
Bei der Absorption wurden außerdem die zusätzlichen Absorptionskoeffizienzen vernachlässigt.
Es wurden also andere Effekte, die ebenfalls die Absorption beeinflusen nicht beachetet, wie zum Beispiel der Paarbilungs- und Photoeffektkoeffizient.
