\section{Auswertung}
\label{sec:Auswertung}

% Messwerte: Alle gemessenen physikalischen Größen sind übersichtlich darzustellen.

% Auswertung:
% Berechnung der geforderten Endergebnisse
% mit allen Zwischenrechnungen und Fehlerformeln, sodass die Rechnung nachvollziehbar ist.
% Eine kurze Erläuterung der Rechnungen (z.B. verwendete Programme)
% Graphische Darstellung der Ergebnisse

\subsection{Bestimmung des Emissionsspektrums der Cu-Röntgenröhre}
\label{ss:Emission}

Die erste in \autoref{sec:Durchführung} beschriebene Messung ergibt die Messwerte in \autoref{tab:emission}.
In \autoref{fig:plot_emission} werden die Messwerte nun als $\lambda-N$ Diagramm dargestellt indem \autoref{eq:glanzwinkel} verwendet wird.
Dabei wird die Gitterkonstante $d=\SI{201.4e-12}{\metre}$ des LiF-Kristall und die Beugungsordnung $n=1$ angenommen.

\begin{figure}
    \centering
    \includegraphics[width=0.7\textwidth]{build/plot_emission.pdf}
    \caption{Plot des Emissionspektrums der Röntgenröhre}
    \label{fig:plot_emission}
\end{figure}

Hier werden die Stellen der relativen Maxima abgelesen und über \autoref{eq:photonenenergie} in eine Photonenenergie übertragen.
Als Naturkonstanten werden hier
\begin{align*}
    c &= \SI{2.998e8}{\metre\per\second} \\
    h &= \SI{4.136e-15}{\electronvolt\second}
\end{align*}
verwendet. \cite{physics_constants}
Damit ergeben sich 
\begin{align*}
    E(K_\alpha) &= \SI{8044}{\electronvolt} \\
    E(K_\beta) &= \SI{8915}{\electronvolt} .
\end{align*}

\subsection{Bestimmung der Transmission vom Aluminiumabsorber}
\label{ssec:transmission}

Die zweite in \autoref{sec:Durchführung} beschriebene Messung ergibt die Messwerte in \autoref{tab:transmission}.
Ziel ist es nun aus diesen Ergebnissen eine Funktion $T(\lambda)$ zu Ermitteln, die die Transmission der Röntgenstrahlung durch die Aluminiumplatte beschreibt.
Dafür wird zuerst mithilfe von \autoref{eq:glanzwinkel}, \autoref{eq:totzeit} und $T=\frac{I_\text{Al}}{I_\text{Ohne}}$ ein $\lambda-T$ Diagramm erstellt.
Dieses ist in \autoref{fig:plot_transmission} zu sehen.
Die Totzeit des Geiger-Müller Zählers wird als $\tau = \SI{90e-6}{\second}$ angenommen und die Integrationszeit der einzelnen Messungen beträgt $t=\SI{200}{\second}$.
Hierbei wird für $N$ ein Fehler $\Delta N = \sqrt{N*t}/t$ verwendet und es wird mit der Python Bibliothek uncertainties gerechnet.\cite{uncertainties}

\begin{figure}
    \centering
    \includegraphics[width=0.7\textwidth]{build/plot_transmission.pdf}
    \caption{Plot der Transmission von Röntgenstrahlung durch einen Aluminiumabsorber}
    \label{fig:plot_transmission}
\end{figure}

Eine Ausgleichsgerade wird über die Python Funktion curve\_fit der Bibliothek SciPy erstellt. \cite{scipy}
Als Gleichung wird hierfür
\begin{equation}
    T(\lambda) = a \cdot \lambda + b
\end{equation}
verwendet und die Ausgleichsrechnung ergibt die Parameter
\begin{align*}
    a &= \SI{-1.461+-0.025e10}{\per\metre} \\
    b &= \SI{1.189+-0.016}{} \,.
\end{align*}

\subsection{Bestimmung der Compton-Wellenlänge}
\label{ssec:compton-wellenlänge}

Der letzte Teil der Messung ergibt die insgesamt gemessenen Impulse in einer Integrationszeit von $t=\SI{300}{\second}$ für drei verschiedene Szenarien
\begin{align*}
    I_0 &= \SI{2731}{Impulse} \\
    I_1 &= \SI{1180}{Impulse} \\
    I_2 &= \SI{1024}{Impulse} \,.
\end{align*}
Hier steht $I_0$ für die Messung ohne Aluminiumabsorber, $I_1$ für die Messung mit Aluminiumabsorber zwischen Blende und Plexiglasstreuer und $I_2$ für die Messung mit Aluminiumabsorber zwischen Plexiglasstreuer und Geiger-Müller Zählrohr.

Mithilfe der Ausgleichsrechnung aus \autoref{ssec:transmission}, $T_i=I_i/I_0$ und
\begin{equation}
    \lambda_i = \frac{T_i - b}{a}
\end{equation}
ergeben sich die Wellenlängen
\begin{align*}
    \lambda_1 &= \SI{5.18+-0.14e-11}{\metre} \\
    \lambda_2 &= \SI{5.57+-0.14e-11}{\metre} \,.
\end{align*}
Damit ergibt sich die Compton-Wellenlänge zu
\begin{equation*}
    \lambda_\text{C} = \lambda_2 - \lambda_1 = \SI{3.910+-0.067e-12}{\metre} \,.
\end{equation*}