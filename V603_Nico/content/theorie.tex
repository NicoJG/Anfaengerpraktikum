\section{Theorie}
\label{sec:Theorie}

% In knapper Form sind die physikalischen Grundlagen des Versuches, des Messverfahrens, sowie sämtliche für die Auswertung erforderlichen Gleichungen darzustellen. (Keine Herleitung)

% (eventuell die Aufgaben)

% Der Versuchsaufbau: Beschreibung des Versuchs und der Funktionsweise (mit Skizze/Bild/Foto)

Der Compton Effekt beschreibt die Streuung eines Photons an einem Elektron.
Beim Stoß mit dem Elektron gibt das Photon Energie ab und dessen Wellenlänge wird gestreckt.
Die Größe dieser Veränderung hängt vom Streuwinkel $\Theta$ ab. 
Wobei $\Theta=\SI{0}{\degree}$ bedeutet, das die Bahn des Photons nicht verändert wurde.
Wenn $\lambda_1$ die Wellenlänge vor dem Stoß und $\lambda_2$ die Wellenlänge nach dem Stoß beschreibt, lässt sich die Differenz über
\begin{equation}
    \Delta \lambda = \lambda_2 - \lambda_1 = \frac{h}{m_e c}(1-\cos \Theta)
    \label{eq:differenz}
\end{equation}
berechnen.
Der Vorfaktor ist proportional zu den Naturkonstanten der Lichtgeschwindigkeit $c$, dem Planckschen Wirkungsquantum $h$ und der Elektronenmasse $m_e$.
Dieser Konstante Vorfaktor 
\begin{equation}
    \lambda_c = \frac{h}{m_e c}
    \label{eq:compton-wellenlänge}
\end{equation}
heißt Compton Wellenlänge.

Um den Effekt beobachten zu können wird hier Röntgenstrahlung an Plexiglas gestreut und die ausfallende Strahlung wird über ein Geiger-Müller-Zählrohr mit der einfallenden Strahlung verglichen.
Diese Röntgenstrahlung wird in einer Röntgenröhre erzeugt, in der Elektronen von einer Glühkathode auf eine Anode beschleunigt werden und der Zusammenstoß auf der Anode erzeugt $\gamma$-Strahlung.

Das Spektrum dieser Strahlung hängt vom Material der Anode ab.
Dieses ist zwar kontinuierlich, allerdings Maxima in der Intensität deutlich zu erkennen, da diese Photonenenergien gerade der Energiedifferenz der Energieniveaus eines Anodenatoms entspricht.

Um das charakteristische Spektrum der hier verwendeten Kupferanode zu bestimmen, wird die Röntgenstrahlung an einem LiF-Kristall gebeugt.
Hierbei entsteht beim Glanzwinkel $\alpha$ eine konstruktive Interferenz, welche von der Wellenlänge der Röntgenstrahlung abhängt. 
Die zugehörige Wellenlänge zum Winkel $\alpha$ kann über 
\begin{equation}
    \lambda = \frac{2d}{n}\sin\alpha
    \label{eq:glanzwinkel}
\end{equation}
bestimmt werden. 
Hier ist $d$ die Gitterkonstante des Kristalls und $n$ die Beugungsordnung.