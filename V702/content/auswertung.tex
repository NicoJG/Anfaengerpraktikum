\section{Auswertung}
\label{sec:Auswertung}

\subsection{Bestimmung der Untergrundrate}
\label{ssec:auswertung_untergrund}

Die Messung der einfallenden Strahlung ohne eine vor den Zähler gestellte Probe ergab die Messwerte
\begin{equation*}
    N_0 = \{129, 143, 144, 136, 139, 126, 158\} \si{\frac{Imp}{300\second}} \, .
\end{equation*}
Nun wird der Mittelwert und der Fehler des Mittelwerts berechnet und ergibt eine durchschnittliche Untergrundrate von
\begin{align*}
    \bar{N}_0 &= \SI{139+-4}{\frac{Imp}{300\second}} \\
              &= \SI{0.46+-0.01}{\frac{Imp}{\second}} \, .
\end{align*}

\subsection{Bestimmung der Halbwertszeit von Vanadium}
\label{ssec:auswertung_vanadium}

Die Messreihe der Intensität der zerfallenden Vanadiumprobe ($ ^{52}_{23}V$) ergab die Messwerte in \autoref{tab:vanadium}. 
Von diesen Werten wird für alle folgenden Berechnungen die Untergrundrate $N_0=\SI{13.9+-0.4}{\frac{Imp}{30\second}}$ abgezogen.
Diese korrigierten Werte sind außerdem in \autoref{fig:plot_vanadium} logarithmisch dargestellt.
Als Fehler wird hier aufgrund der Poissonverteilung $\Delta N = \sqrt{N}$ angenommen.

\begin{table}
    \centering
    \caption{Messergebnisse für die Vanadiumprobe}
    \resizebox{0.5\textwidth}{!}{%
    \begin{tabular}{S[table-format=3.0] S[table-format=3.0(2)]}
        \toprule
        \tableSI{t}{\second} & \tableSI{N}{\frac{Imp}{30\second}} \\
        \midrule
        30 & 189+-13 \\
        60 & 197+-14 \\
        90 & 150+-12 \\
        120 & 159+-12 \\
        150 & 155+-12 \\
        180 & 132+-11 \\
        210 & 117+-10 \\
        240 & 107+-10 \\
        270 & 94+-9 \\
        300 & 100+-10 \\
        330 & 79+-8 \\
        360 & 69+-8 \\
        390 & 81+-9 \\
        420 & 46+-6 \\
        450 & 49+-7 \\
        \bottomrule
    \end{tabular}
    \begin{tabular}{S[table-format=3.0] S[table-format=3.0(2)]}
        \toprule
        \tableSI{t}{\second} & \tableSI{N}{\frac{Imp}{30\second}} \\
        \midrule
        480 & 61+-7 \\
        510 & 56+-7 \\
        540 & 40+-6 \\
        570 & 45+-6 \\
        600 & 32+-5 \\
        630 & 27+-5 \\
        660 & 43+-6 \\
        690 & 35+-5 \\
        720 & 19+-4 \\
        750 & 28+-5 \\
        780 & 27+-5 \\
        810 & 36+-6 \\
        840 & 25+-5 \\
        870 & 29+-5 \\
        900 & 18+-4 \\
        \bottomrule
    \end{tabular}
    \begin{tabular}{S[table-format=3.0] S[table-format=3.0(2)]}
        \toprule
        \tableSI{t}{\second} & \tableSI{N}{\frac{Imp}{30\second}} \\
        \midrule
        930 & 17+-4 \\
        960 & 24+-4 \\
        990 & 21+-4 \\
        1020 & 25+-5 \\
        1050 & 21+-4 \\
        1080 & 24+-4 \\
        1110 & 25+-5 \\
        1140 & 17+-4 \\
        1170 & 20+-4 \\
        1200 & 19+-4 \\
        1230 & 20+-4 \\
        1260 & 18+-4 \\
        1290 & 16+-4 \\
        1320 & 17+-4 \\
         & \\
        \bottomrule
    \end{tabular}
    }%
    \label{tab:vanadium}
\end{table}

Nun wird nach \autoref{eq:zerfall} eine Ausgleichsgerade mit 
\begin{equation}
    \ln(N(t)) = \lambda*t+b
    \label{eq:ausgleichsgerade}
\end{equation}
über die Funktion curve\_fit der Python Bibliothek SciPy bestimmt und ebenfalls in \autoref{fig:plot_vanadium} dargestellt.\cite{scipy}
Die Bestimmung der Parameter ergibt
\begin{align*}
    \lambda &= \SI{-0.0032+-0.0001}{\per\second} \\
    b &= \num{5.32+-0.04} \, .
\end{align*}
Über \autoref{eq:lambda} lässt sich nun die Halbwertszeit zu
\begin{equation*}
    T = \SI{215+-7}{\second}
\end{equation*}
berechnen.
Ein Vergleichswert aus der Literatur ist
\begin{equation*}
    T_\text{lit} = \SI{224.58}{\second} \, .\text{\cite{periodic_table}}
\end{equation*}

\begin{figure}
    \centering
    \includegraphics[width=0.8\textwidth]{build/plot_Vanadium.pdf}
    \caption{Plot der Messwerte für Vanadium}
    \label{fig:plot_vanadium}
\end{figure}


\subsection{Bestimmung der Halbwertszeit von Rhodium}
\label{ssec:auswertung_rhodium}

Die Messreihe der Intensität der zerfallenden Rhodiumprobe (ca. $\SI{10}{\percent}$ $ ^{104i}_{45}Rh$ und $\SI{90}{\percent}$ $ ^{104}_{45}Rh$) ergab die Messwerte in \autoref{tab:rhodium}.
Auch hier wird für die weiteren Berechnungen die Untergrundrate $N_0=\SI{6.96+-0.20}{\frac{Imp}{15\second}}$ abgezogen und der Fehler $\Delta N = \sqrt{N}$ angenommen.
Diese korrigierten Werte sind außerdem in \autoref{fig:plot_rhodium} logarithmisch dargestellt.

\begin{table}
    \centering
    \caption{Messergebnisse für die Rhodiumprobe}
    \resizebox{0.5\textwidth}{!}{%
    \begin{tabular}{S[table-format=3.0] S[table-format=3.0(2)]}
        \toprule
        \tableSI{t}{\second} & \tableSI{N}{\frac{Imp}{15\second}} \\
        \midrule
        15 & 667+-25 \\
        30 & 585+-24 \\
        45 & 474+-21 \\
        60 & 399+-19 \\
        75 & 304+-17 \\
        90 & 253+-15 \\
        105 & 213+-14 \\
        120 & 173+-13 \\
        135 & 152+-12 \\
        150 & 126+-11 \\
        165 & 111+-10 \\
        180 & 92+-9 \\
        195 & 79+-8 \\
        210 & 74+-8 \\
        225 & 60+-7 \\
        \bottomrule
    \end{tabular}
    \begin{tabular}{S[table-format=3.0] S[table-format=3.0(2)]}
        \toprule
        \tableSI{t}{\second} & \tableSI{N}{\frac{Imp}{15\second}} \\
        \midrule
        240 & 52+-7 \\
        255 & 56+-7 \\
        270 & 53+-7 \\
        285 & 41+-6 \\
        300 & 36+-6 \\
        315 & 37+-6 \\
        330 & 32+-5 \\
        345 & 36+-6 \\
        360 & 38+-6 \\
        375 & 34+-5 \\
        390 & 40+-6 \\
        405 & 21+-4 \\
        420 & 35+-5 \\
        435 & 33+-5 \\
        450 & 36+-6 \\
        \bottomrule
    \end{tabular}
    \begin{tabular}{S[table-format=3.0] S[table-format=3.0(2)]}
        \toprule
        \tableSI{t}{\second} & \tableSI{N}{\frac{Imp}{15\second}} \\
        \midrule
        465 & 20+-4 \\
        480 & 24+-4 \\
        495 & 30+-5 \\
        510 & 30+-5 \\
        525 & 26+-5 \\
        540 & 28+-5 \\
        555 & 23+-4 \\
        570 & 20+-4 \\
        585 & 28+-5 \\
        600 & 17+-4 \\
        615 & 26+-5 \\
        630 & 19+-4 \\
        645 & 13+-3 \\
        660 & 17+-4 \\
         & \\
        \bottomrule
    \end{tabular}
    }
    \label{tab:rhodium}
\end{table}

Hier sind allerdings im Vergleich zu Vanadium zwei Zerfallsprozesse zu beobachten. 
Die Halbwertszeit von $ ^{104i}_{45}Rh$ ist deutlich geringer und liefert wie in \autoref{fig:plot_rhodium} zu sehen ab einem gewissen Zeitpunkt nahezu keinen Beitrag zur Intensität mehr.
Um also die Halbwertszeit vom langlebigeren $ ^{104}_{45}Rh$ zu bestimmen, wird eine Ausgleichsgerade durch alle Messwerte ab $t=\SI{270}{\second}$ gelegt, da hier der Übergang vom kurzlebigen zum langlebigen Zerfallsprozess nahezu vorüber ist.
Somit ergeben sich nach \autoref{eq:ausgleichsgerade} die Parameter zu 
\begin{align*}
    \lambda_\text{lang} &= \SI{-0.0027+-0.0004}{\per\second} \\
    b_\text{lang} &= \num{4.35+-0.17}
\end{align*}
und über \autoref{eq:lambda} eine Halbwertszeit von 
\begin{equation*}
    T_\text{lang} = \SI{257+-37}{\second} \, .
\end{equation*}
Der entsprechende Vergleichswert aus der Literatur ist 
\begin{equation*}
    T_\text{lang,lit} = \SI{260.4}{\second} \, .\text{\cite{periodic_table}}
\end{equation*}

\begin{figure}
    \centering
    \includegraphics[width=0.8\textwidth]{build/plot_Rhodium.pdf}
    \caption{Plot der Messwerte für Rhodium}
    \label{fig:plot_rhodium}
\end{figure}

In der Zeit bis $t=\SI{210}{\second}$ ist analog nahezu nur der kurzlebige Zerfall zu beobachten, allerdings müssen hier die Messwerte über die extrapolierte Ausgleichsgerade des langlebigen Prozesses korrigiert werden.
Also werden die in blau dargestellten Messwerte über
\begin{equation}
    N_\text{korrigiert}(t) = N(t) - \exp(\lambda_\text{lang} * t + b_\text{lang})
\end{equation}
berechnet und für die Ausgleichsgerade des kurzlebigen Prozesses verwendet.
Damit ergeben sich die Parameter zu
\begin{align*}
    \lambda_\text{kurz} &= \SI{-0.0161+-0.0004}{\per\second} \\
    b_\text{kurz} &= \num{6.68+-0.03}
\end{align*}
und über \autoref{eq:lambda} eine Halbwertszeit von 
\begin{equation*}
    T_\text{kurz} = \SI{43+-1}{\second} \, .
\end{equation*}
Der entsprechende Vergleichswert aus der Literatur ist 
\begin{equation*}
    T_\text{kurz,lit} = \SI{42.3}{\second} \, .\text{\cite{periodic_table}}
\end{equation*}