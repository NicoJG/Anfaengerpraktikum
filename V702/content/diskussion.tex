\section{Diskussion}
\label{sec:Diskussion}

Die Messung der Untergrundrate zeigt ausreichend geringe Abweichungen in der gemessenen Intensität, deswegen scheint der Mittelwert $\bar{N}_0 = \SI{139}{\frac{Imp}{300\second}}$ geeignet um damit zu rechnen.
Allerdings muss bemerkt werden, dass dies eine Näherung ist und zu weiteren Abweichungen in allen nachfolgenden Berechnungen führen wird.

Die Messung der Vanadiumprobe zeigt den erwarteten exponentiellen Verlauf, allerdings sind die Messwert am Ende der Messzeit sehr ungenau, da die Intensität fast so niedrig ist wie der Untergrund.
Die Halbwertszeit wurde zu
\begin{equation*}
    T_V = \SI{215+-7}{\second}
\end{equation*}
ermittelt und ergibt somit eine Abweichung vom Literaturwert von
\begin{equation*}
    \Delta T_V = \SI{4.27}{\percent} \, .
\end{equation*}

Die Messung der Rhodiumprobe zeigt ebenfalls das erwartete Bild, wobei auch hier die Messwerte am Ende der Messreihe hohe Ungenauigkeiten zeigen.
Zuerst wurde die Halbwertszeit vom langlebigen Zerfalls zu
\begin{equation*}
    T_\text{Rh,lang} = \SI{257+-37}{\second}
\end{equation*}
ermittelt und ergibt eine Abweichung vom Literaturwert von 
\begin{equation*}
    \Delta T_\text{Rh,lang} = \SI{1.31}{\percent} \, .
\end{equation*}
Allerdings muss hier erwähnt werden dass der statistische Fehler sehr hoch ist.

Dann wurde mithilfe der extrapolierten Ausgleichsrechnung des langlebigen Zerfalls der kurzlebige Zerfall untersucht.
Hier ergibt sich eine Halbwertszeit von
\begin{equation*}
    T_\text{Rh,kurz} = \SI{43+-1}{\second}
\end{equation*}
und eine Abweichung vom Literaturwert von 
\begin{equation*}
    \Delta T_\text{Rh,kurz} = \SI{1.65}{\percent} \, .
\end{equation*}

Insgesamt konnten trotz der gemachten Näherungen und der statistischen Natur des Versuchs gute Schätzwerte für die Halbwertszeiten berechnet werden.