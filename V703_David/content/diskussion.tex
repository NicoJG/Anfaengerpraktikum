\section{Diskussion}
\label{sec:Diskussion}

% Kurze Zusammenfassung der Ergebnisse
% -Vergleich mit Literaturwerten
% -Vergleich mit verschiedenen Messverfahren
% -bei Abweichungen mögliche Ursachen finden

Mit einer Unsicherheit von $\SI{1.21(23)}{\%\per100\volt}$ ist das Zählrohr recht ungenau, besonders bei höheren Spannungen.
Der Wert ist an sich sehr ungenau, da hier mit Augenmaß bestimmt werden musste, von wo bis wo das Plateau festgelegt wird.
Daher ist die ausgerechnete Steigung nicht sehr zuverlässig.
Die Totzeit $\tau$ liegt mit etwa $\SI{115(4)}{\micro\second}$ etwa im Bereich, in dem andere Zählrohrtotzeiten liegen. In einem vorherigen Versuch lag diese beispielsweise bei $\SI{90}{\micro\second}$. \cite{V603}
Über das Oszilloskop kann nur eine sehr ungenaue Aussage über die Totzeit getroffen werden, allerdings liegt sie mit ihrer Unsicherheit im gleichen Bereich, wie die berechnete, daher kann der Wert als gute Näherung angesehen werden.
Es ist zu sehen, dass die Werte für $Z$ auf einer gut genäherten Gerade liegen, die berechneten Werte weisen nur wenig Abweichung auf.