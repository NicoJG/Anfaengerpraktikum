\section{Diskussion}
\label{sec:Diskussion}

Die Charakteristik in \autoref{fig:plot_charakteristik} zeigt das erwartete Bild und ergibt einen Plateauanstieg von $\SI{1.15}{\percent}$.
Der Anstieg ist gering genug um eine effektive Messung der Intensität zu ermöglichen, da die Steigung deutlich geringer zu der Steigung vor oder nach dem Plateau ist.
Allerdings sind deutliche statistische Ungenauigkeiten zu sehen und manche Messwerte sind schon bei geringer Spannung so hoch wie die Messpunkte am rechten Rand des Plateaus.
Das Zählrohr scheint geeignet für eine Messung der Intensität von hochenergetischer Strahlung zu sein, aber die Präzision zeigt deutliche Defizite.

Auch die bestimmte Totzeit $T=\SI{115}{\micro\second}$ hat die erwartete Größenordnung.
Allerdings ist eine Bestimmung dieser Totzeit über die visuelle Analyse des Oszillograms zu ungenau um eine gute Aussage über die Totzeit machen zu können.

Die in \autoref{tab:strom} aufgelisteten Anzahlen der freigesetzten Ladungen pro einfallendem Teilchen zeigen, dass ein einfallendes Teilchen tatsächlich sehr viele Townsend-Lawinen auslöst.
Außerdem scheint der Zusammenhang zwischen Spannung und freigesetzer Ladungen linear steigend zu sein.

Aufgrund großer Schwankungen in der Intensitätsmessung und Strommessung können alle bestimmten Werte nur als Schätzung angesehen werden und die tatsächliche Bestimmung der Kenngrößen des Zählrohrs, würde ein deutlich genaueres Messverfahren verlangen.