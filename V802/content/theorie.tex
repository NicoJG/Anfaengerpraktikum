\section{Theorie}
\label{sec:Theorie}

Jede periodische Funktion lässt sich mit einer Fouriersynthese in einer Reihe von $\sin(x)$ und $\cos(x)$ darstellen. Die allgemeine Darstellung ist dann 
\begin{equation}
    \label{eq:Fouriersynthese}
    f (t) = \sum_{k=0}^\infty \left(A_k \cos \left( \omega _k t \right) + B_k \sin \left( \omega _k t \right) \right).
\end{equation}
Wobei $\omega _k = \frac {2 \pi k}{T}$ ist und die Koeffizienten $A_k$ und $B_k$ mit 
\begin{align}
    A_k &= \frac{2}{T} \int_{\frac{-T}{2}}^{\frac{T}{2}} f(t) \, \cos \left( \omega _k t \right)  \symup{d}t & A_0 &= \frac{1}{T} \int f (t) \, \symup{d}t \label{eq:A_k} \\
    B_k &= \frac{2}{T} \int_{\frac{-T}{2}}^{\frac{T}{2}} f(t) \, \sin \left( \omega _k t \right)  \symup{d}t & B_0 &= 0 \label{eq:B_k}
\end{align}
berechnet werden. Wobei $T$ die Periodendauer ist. \cite{V802}