\section{Auswertung}
\label{sec:Auswertung}

Die Federkraft und Auslenkung hängen nach dem Hook'schen Gesetz über
\begin{equation}
  \label{eq:Hook}
  F = - D \cdot \Delta x
\end{equation}
zusammen.
Die Federkonstante $D$ kann nun aus den Messwerten über zwei Methoden bestimmt werden.

\subsection{Mittelwertsbildung}

\begin{equation}
  \left< x \right> = \frac{1}{n} \cdot \sum_{i=1}^{n} x_i
\end{equation}

\subsection{Ausgleichsrechnung}

Da \eqref{eq:Hook} ein linearer Zusammenhang ohne y-Achsenabschnitt ist kann für die Ausgleichsrechnung
\begin{equation}
  \label{eq:Lin-Ausgleich}
  D = - \frac
  {\left< F \Delta x \right> - \left< F \, \right> \left< \Delta x \right>}
  {\left< \Delta x^2 \right> - \left< \Delta x \right> ^2}
\end{equation}

\begin{figure}
  \centering
  \includegraphics{plot.pdf}
  \caption{Plot.}
  \label{fig:plot}
\end{figure}


Siehe \autoref{fig:plot}!
