\section{Auswertung}
\label{sec:Auswertung}
\begin{table}
  \centering
  \caption{Messwerte: Auslenkung $\Delta x $ und Federkraft $F$}
  \label{tab:data}
  \begin{tabular}{c c}
    \toprule 
    $\Delta x \:/\: \si{\centi\meter}$ & $F \:/\: \si{\newton}$ \\ 
    \midrule 
    2 & 0.06 \\
    6 & 0.18 \\
    10 & 0.29 \\
    16 & 0.47 \\
    21 & 0.62 \\
    24 & 0.71 \\
    33.5 & 1.00 \\
    38 & 1.13 \\
    50 & 1.49 \\
    58 & 1.73 \\ 
    \bottomrule
  \end{tabular}
\end{table} 
\noindent Die Federkraft und Auslenkung hängen nach dem Hookeschen Gesetz über
\begin{equation}
  \label{eq:Hook}
  F = - D \cdot \Delta x
\end{equation}
zusammen.
Die Federkonstante $D$ kann nun aus den Messwerten (siehe \autoref{tab:data}) über zwei Methoden bestimmt werden.

\newpage

\subsection{Mittelwertsbildung}
Aus den zehn aufgenommenen Messwerten wird nun jeweils $D_i = F_i / \Delta x_i$ berechnet und davon der Mittelwert mit der Formel
\begin{equation}
  \label{eq:Mittelwert}
  \left< D \, \right> = \frac{1}{n} \cdot \sum_{i=1}^{n} D_i
\end{equation}
bestimmt. Damit ergibt sich eine Federkonstante $D = \SI{0.03}{\newton\per\meter}$.

\subsection{Ausgleichsrechnung}

Da \autoref{eq:Hook} ein linearer Zusammenhang ohne y-Achsenabschnitt ist kann für die Ausgleichsrechnung
\begin{equation}
  \label{eq:Lin-Ausgleich}
  D = - \frac
  {\left< F \Delta x \right> - \left< F \, \right> \left< \Delta x \right>}
  {\left< \Delta x^2 \right> - \left< \Delta x \right> ^2}
\end{equation}
verwendet werden.
Aus den gemessenen Werten ergeben sich nun die Mittelwerte $\left< F \Delta x \right> = \SI{29.296}{\newton\meter}$, $\left< F \, \right> = \SI{0.768}{\newton}$, $\left< x \right> = \SI{25.85}{\meter}$, $\left< x^2 \right> = \SI{984.325}{\meter\squared}$ und $\left< x \right> ^2 = \SI{668.223}{\meter\squared}$.
Damit ergibt sich eine Federkonstante $D=\SI{0.03}{\newton\per\meter}$.
Dies kann man in \autoref{fig:plot} sehen.

\begin{figure}
  \centering
  \includegraphics{plot.pdf}
  \caption{Die Messwerte aus \autoref{tab:data} und die dazugehörige Ausgleichsgerade}
  \label{fig:plot}
\end{figure}

