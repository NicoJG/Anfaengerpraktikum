\section{Auswertung}
\label{sec:Auswertung}

Die Federkraft und Auslenkung hängen nach dem Hook'schen Gesetz über
\begin{equation}
F=-D\cdot\delta x
\end{equation}
zusammen.
Die Federkonstante $D$ kann nun aus den Messwerten über zwei Methoden bestimmt werden.

\subsection{Mittelwertsbildung}
Aus den zehn aufgenommenen Messwerten für die Federkonstente D wird nun der Mittelwert mit der Formel

\begin{equation}
  \label{eq:Mittelwert}
  \bar{x} = \frac{1}{n} \cdot \sum_{i=1}^{n} x_i
\end{equation}
bestimmt. 

\subsection{Ausgleichsrechnung}


\begin{figure}
  \centering
  \includegraphics{plot.pdf}
  \caption{Plot.}
  \label{fig:plot}
\end{figure}


Siehe \autoref{fig:plot}!
