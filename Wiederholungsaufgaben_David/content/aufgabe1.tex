\section{Begrifflichkeiten}
\label{sec:1}

1) Fassen Sie die Bedeutung folgender Begriffe in IHRE Worte und notieren sie eine Möglichkeit der Berechnung.

1.1) Was bezeichnet der Mittelwert?

Ein Mittelwert wird aus einer vorgegebenen Anzahl an Werten ermittelt und gibt den durchschnittlichen Wert dieser Größen an. In den meisten Fällen wird vom Arithmetischen Mittel gesprochen, also der Summation aller Werte, die dann durch die Anzahl der Werte geteilt wird. Bei $x_n$ Werten ist der Mittelwert $\bar{x}$ durch

\begin{equation}
    \bar{x} = \frac{1}{n} \cdot \sum_{i = 1}^{n} x_i
\end{equation}

bestimmbar.

1.2) Welche Bedeutung hat die Standartabweichung?

Die Standartabweichung $\sigma$ ist ein Maß für die Abweichung vom Erwartungswert $\mu$. 
Der Erwartungswert ist dabei das Ergebnis, das sich auf lange Sicht einstellen wird, also der Wert, der gemittelt am wahrscheinlichsten passieren wird.
Mit der Standartabweichung kann ermittelt werden, wie genau die Werte um diesen Erwartungswert verteilt sind.
Eine kleine Standartabweichung bedeutet, die Werte liegen wie erwartet.
Sie wird nach 

\begin{equation}
    \sigma = \sqrt{\frac{1}{n} \cdot \sum_{i = 1}^{n} \left(x_i - \mu \right)^2}
\end{equation}

berechnet.

1.3) Worin unterscheidet sich die Streuung der Messwerte und der Fehler des
Mittelwertes?

Die Streuung der Messwerte ist ein Maß dafür, wie genau die Werte um den Erwartungswert verteilt sind.
Der Fehler des Mittelwerts wird verwendet, um zu beschreiben, wie weit der berechnete Mittelwert vom tatsächlichen Mittelwert der Gesamtheit abweicht.

