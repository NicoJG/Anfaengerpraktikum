\section{Hohlzylinder}
\label{sec:2}

2) Berechnen Sie das Volumen eines Hohlzylinders, mit $R_\text{innen} = \SI{10 \pm 1}{\centi\meter}$, $R_\text{außen} = \SI{15 \pm 1}{\centi\meter}$ und $h = \SI{20 \pm 1}{\centi\meter}$.

Das Volumen eines Hohlzylinders wird über 

\begin{equation}
    V = \pi \cdot h \cdot \left(R_\text{außen}^2 - R_\text{innen}^2\right)
\end{equation}

bestimmt. 

Zunächst wird diese Rechnung ohne Beachten der Unsicherheit durchgeführt.

\begin{equation}
    \pi \cdot \SI{20 \pm 1}{\centi\meter} \cdot \left((\SI{15 \pm 1}{\centi\meter})^2 - (\SI{10 \pm 1}{\centi\meter})^2\right) = \SI{2500 \pi}{\cubic\centi\meter}.
\end{equation}

Mithilfe der Gaußschen Fehlerfortpflanzungsformel

\begin{equation}
    \Delta f = \sqrt{\sum_{i = 1}^{n} \left(\frac{\dif{f}}{\dif{y_\text{i}}} \cdot \Delta y_\text{i}\right)^2}, 
\end{equation}

wird die Unsicherheit dieses berechneten Ergebnisses bestimmt.
Wobei $y_\text{i}$ die Anzahl der verschiedenen Größen mit Untersicherheit ist.

\begin{equation*}
    \Delta V = \sqrt{\left(2 \pi \cdot h \cdot R_\text{außen} \cdot \Delta R_\text{außen}\right)^2 + \left(-2 \pi \cdot h \cdot R_\text{innen} \cdot \Delta R_\text{innen}\right)^2 + \left(\pi \left(R_\text{außen}^2 - R_\text{innen}^2\right) \cdot \Delta h\right)^2} 
\end{equation*}
\begin{equation*}
    \Delta V = \sqrt{\left(2 \pi \cdot 20 \cdot 15 \cdot 1 \right)^2 + \left(-2 \pi \cdot 20 \cdot 10 \cdot 1\right)^2 + \left(\pi \left(15^2 - 10^2\right) \cdot 1\right)^2} \si{\cubic\centi\meter}
\end{equation*}
\begin{equation*}
    \Delta V = \SI{2299}{\cubic\centi\meter} 
\end{equation*}

Damit beträgt das Volumen des Hohlzylinders 

\begin{equation*}
    V = 2500 \pi \pm \SI{2299}{\cubic\centi\meter}.
\end{equation*}
