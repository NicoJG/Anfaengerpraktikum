\section{Lineare Regression}
\label{sec:3}

3) Führen Sie eine lineare Regression durch. In der Anlage Regression.pdf befinden sich Daten und eine Anleitung, wie die Daten ausgewertet werden sollen.
Die Graphik und die Auswertung kann natürlich mit jedem beliebigen Programm
durchgeführt werden. Für diejenigen, die Python verwenden oder die Auswertung
mit Python ausprobieren wollen ist eine detaillierte Arbeitsanweisung beigefügt.

In diesem Versuch wurden an verschiedenen Linien $N$ Spannungen $U$ gemessen, die Ergebnisse dieser Messung wurden in der nachfolgenen Tabelle notiert.
Für die lineare Regression wird die Linienzahl $N$ zunächst in einen Abstand $D$ umgerechnet.
Dafür wird die Formel

\begin{equation}
    D = (N_\text{Linie} - 1) \cdot \SI{6}{\milli\meter}
\end{equation}

verwendet.

\begin{table}
    \centering
    \label{tab:werte}
    \begin{tabular}{S S S}
        \toprule
        {$N_{\text{Linie}}$} & \tableSI{D}{\milli\metre} & \tableSI{U}{\volt} \\
        \midrule
        1 & 0 & -19.5 \\
        2 & 6 & -16.1 \\
        3 & 12 & -12.4 \\
        4 & 18 & -9.6 \\
        5 & 24 & -6.2 \\
        6 & 30 & -2.4 \\
        7 & 36 & 1.2 \\
        8 & 42 & 5.1 \\
        9 & 48 & 8.3 \\
        \bottomrule
    \end{tabular}
    \caption{Gegebene Messwerte $N_\text{Linie}$, $U$ und berechnete Abstände $D$}
\end{table}

In einem Diagramm wird der Abstand $D$ gegen die Spannung $U$ aufgetragen, dadurch wird ein linearer Zusammenhang deutlich.
Deswegen wird die Regression mit einer linearen Funktion der Form

\begin{equation}
    f = ax + b
\end{equation}

durchgeführt. 
Es entsteht der Plot aus \autoref{fig:werte_plot}.

\begin{figure}
    \centering
    \includegraphics[width=\textwidth]{build/plot_werte.pdf}
    \caption{Berechnete Werte von $D$ gegen $U$ aufgetragen mit Ausgleichsgeraden.}
    \label{fig:werte_plot}
\end{figure}

Der Plot wurde mit den beiden Parametern $a$ und $b$ gebildet, dessen Werte und ihre Unsicherheit sind durch

\begin{equation}
    a = 0.581 \pm \SI{0.006}{\volt\per\milli\meter}
\end{equation}
\begin{equation}
    b = -19.444 \pm \SI{0.173}{\volt}
\end{equation}

gegeben.