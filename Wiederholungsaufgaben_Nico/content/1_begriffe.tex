\section{Begriffserklärungen}
\label{sec:Begriffe}

\subsection{Was bezeichnet der Mittelwert?}
\label{ssec:Mittelwert}

Der Mittelwert $\bar{x}$ einer Reihe von $N$ Messwerten $x_i$ kann durch
\begin{equation}
    \bar{x} = \frac{1}{N} \sum_{i=1}^N x_i 
\end{equation}
berechnet werden. 
Dieser bezeichnet eine Art Schwerpunkt der Messwerte oder Durchschnittswert. 
Der Mittelwert einer rein zufälligen Messreihe ist außerdem der Erwartungswert $\lambda$ dieser Reihe.

\subsection{Welche Bedeutung hat die Standardabweichung?}
\label{ssec:Standardabweichung}

Die Standardabweichung $\sigma$ kann durch
\begin{equation}
    \sigma = \sqrt{\frac{1}{N} \sum_{i=1}^N (x_i - \lambda)^2}
\end{equation}
berechnet werden. 
Dieser Wert bezeichnet die mittlere Distanz der Messwerte zum Erwartungswert und ist somit ein Maß der Streuung der Messwerte.

\subsection{Worin unterscheidet sich die Streuung der Messwerte und der Fehler des Mittelwerts?}
\label{ssec:Streuung-Fehler}

Die Streuung der Messwerte, welche über die Standardabweichung beschrieben wird, beschreibt wie weit die realen Messwerte vom Erwartungswert abweichen.

Der Fehler des Mittelwerts wird benutzt um bei einer therotisch unendlich großen Messreihe die Streubreite angeben zu können.
Also beschreibt er wie weit die Mittelwerte vieler Messungen zueinander streuen werden.