\section{Volumen eines Hohlzylinders}
\label{sec:Hohlzylinder}

Im Folgenden wird das Volumen eines Hohlzylinders mit den Maßen
\begin{align*}
    R_\text{innen} = R_\text{i} &= \SI{10+-1}{\centi\metre} \\
    R_\text{außen} = R_\text{a} &= \SI{15+-1}{\centi\metre} \\
    h &= \SI{20+-1}{\centi\metre}
\end{align*}
berechnet. Dieses lässt sich durch Subtraktion des inneren Zylinders vom äußeren Zylinder berechnen.
Das Volumen eines Zylinders wird über
\begin{equation}
    V_\text{Zylinder} = \pi h R^2
\end{equation}
berechnen.
Also kann das Volumen des Hohlzylinders über
\begin{align*}
    V &= V_\text{außen} - V_\text{innen} \\
    &= \pi h (R_\text{a}^2 - R_\text{i}^2) 
\end{align*}
berechnet werden.
Die zugehörige Gaußsche Fehlerfortpflanzung ergibt
\begin{align*}
    \Delta V &= \sqrt{\left(\frac{\partial V}{\partial h} \Delta h \right)^2 + \left(\frac{\partial V}{\partial R_\text{a}} \Delta R_\text{a} \right)^2 + \left(\frac{\partial V}{\partial R_\text{i}} \Delta R_\text{i} \right)^2} \\
    &= \pi \sqrt{ (R_\text{a}^2 - R_\text{i}^2)^2 \Delta h^2 + 4 h^2 R_\text{a}^2 \Delta R_\text{a}^2 + 4 h^2 R_\text{i}^2 \Delta R_\text{i}^2 } \; .
\end{align*}
Somit ergibt sich das Volumen des Hohlzylinders zu
\begin{equation}
    V = \SI{7854+-2299}{\centi\metre\cubed} \; .
\end{equation}