\section{Lineare Regression}
\label{sec:Regression}

Gegeben waren die Werte $U$ gemessen an einer Linie $N_\text{Linie}$ aus \autoref{tab:regression}. 
Über 
\begin{equation}
    D = (N_\text{Linie}-1) \cdot \SI{6}{\milli\metre}
\end{equation}
werden den Linien ein Abstand zugeordnet.

\begin{table}
    \centering
    \begin{tabular}{S S S}
        \toprule
        {$N_\text{Linie}$} & \tableSI{D}{\milli\metre} & \tableSI{U}{\volt} \\
        \midrule
        1 & 0.000 & -19.5 \\
        2 & 6.000 & -16.1 \\
        3 & 12.000 & -12.4 \\
        4 & 18.000 & -9.6 \\
        5 & 24.000 & -6.2 \\
        6 & 30.000 & -2.4 \\
        7 & 36.000 & 1.2 \\
        8 & 42.000 & 5.1 \\
        9 & 48.000 & 8.3 \\
        \bottomrule
    \end{tabular}
    \caption{Gegebene Messwerte $N_\text{Linie}$, $U$ und berechnete Abstände $D$}
\end{table}

Nun wird mithilfe von
\begin{equation}
    U = a D + b
\end{equation}
eine lineare Regression über die Funktion curve\_fit aus der Python Bibliothek SciPy erstellt.
Somit ergeben sich die Parameter
\begin{align}
    a &= \SI{0.581+-0.007}{\volt\per\milli\metre} \\
    b &= \SI{-19.7+-0.2}{\volt} \; .
\end{align}
Diese Ausgleichsrechnung ist in \autoref{fig:plot_regression} dargestellt.

\begin{figure}[]
    \centering
    \includegraphics[width=\textwidth]{build/plot_regression.pdf}
    \caption{Plot der beschriebenen Daten und Ausgleichsrechnung}
    \label{fig:plot_regression}
\end{figure}